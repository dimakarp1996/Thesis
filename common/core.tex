\newcommand{\pseudolabel}[1]{%
    \ifthenelse{\equal{#1}{valid}}    {
    \begin{table}[htbp]
\centering
\caption {Лучшая точность на валидационных данных (при лучшей скорости обучения из выбираемых, среднее по 3 запускам)}
\label{tab:ps1}% label всегда желательно идти после caption
\resizebox{\textwidth}{!}{%
\begin{tabular}{|c||c|c|c|c|c|}
\hline
\multirow{2}{*}{Эксперимент} & \multicolumn{5}{c|}{Задача}  \\
\cline{2-6}
& Среднее& RTE & QQP & MNLI-m & SST-2 \\
\hline
\hline
Базовый (воспроизведённый) & 81.3 & 64.6 & 90.8 & 77.3 & 92.7   \\
\hline
Независимые метки    & 82.8  & \textbf{78.3} & 90.6 & 75.8 & 92.0  \\
\hline
Независимые метки, замороженная голова  & 82.5 & 76.1 & 90.5 & 75.7 & 91.4  \\
\hline
Мягкие независимые метки     & 82.2 & 69.7 & 89.5 & 75.9 & \textbf{92.6} \\
\hline
Мягкие независимые метки, замороженная голова & 82.6 & 74.4 & 90.4 & \textbf{76.7} & 91.2  \\
\hline
Дополненные независимые метки   & 81.4 & 68.1 & 90.5 & 75.6 & 92.4 \\
\hline
Мягкое вероятностное предположение  & \textbf{84.2} & 78.2 & \textbf{90.7} & 76.2 & 91.9   \\
\hline
Мягкие предсказанные метки  & 83.2 & 76.3 & 90.5 & 76.0 & 92.2  \\
\hline
Жесткие предсказанные метки & 82.9 & 77.4 & 90.6 & 75.3 & 90.7 \\
\hline
\end{tabular}
}
\end{table}
    }
    \ifthenelse{\equal{#1}{test}}    {
    \begin{table}[htbp]
\centering
\caption {Лучшая точность на тестовых данных (при лучшей скорости обучения из выбираемых, среднее по 3 запускам)}
\label{tab:ps2}% label всегда желательно идти после caption
\resizebox{\textwidth}{!}
{%
\begin{tabular}{|c||c|c|c|c|c|c|}
\hline
\multirow{2}{*}{Эксперимент} & \multicolumn{6}{c|}{Задача}  \\
\cline{2-7}
& Среднее& RTE& QQP& MNLI-m &MNLI-mm& SST-2\\
\hline
Базовый (оригинальная статья) & 78.8 & 66.4  & 71.2 & 84.6 & 83.4 & 93.5 \\
\hline
Базовый (воспроизведённый) & 77.6 & 62.7 & 71.0 & 83.1 & 82.7 & 93.5 \\
\hline
Независимые метки &79.0 & 71.5 & 70.9 & 82.7 & 81.7 & 91.3 \\
\hline
Мягкие независимые метки  &78.9 & 69.3 & 71.3 & 82.8 & 82.1 & 92.6 \\
\hline
Дополненные независимые метки &77.6 & 64.2&\textbf{ 71.8} & 81.2 & 80.7 & \textbf{93.2} \\
\hline
Мягкое вероятностное предположение  &\textbf{79.7} & \textbf{72.7} & 70.7 &\textbf{ 83.4} &\textbf{82.3} & 92.5 \\
\hline
Мягкие предсказанные метки  & 78.8 & 70.3 & 70.7 & 81.7 & 81.7 & 92.5 \\
\hline
Жесткие предсказанные метки & 79.1 & 71.3 &71.1 & 81.7 & 81.4 & 92.6 \\
\hline
\begin{tabular}[c]{@{}l@{}}Независимые метки,\\замороженная голова\end{tabular}   & 78.2 & 66.9& \textbf{71.8} & 82.6 & 81.8 & 91.9 \\
\hline
\begin{tabular}[c]{@{}l@{}}Мягкие независимые метки,\\замороженная голова\end{tabular}  &79.1 & 70.0 & 71.5 & 83.0 &\textbf{ 82.3} & 92.4 \\
\hline
\end{tabular}
}
\end{table}
}
}

\newcommand{\rutopics}[1]{%
    \ifthenelse{\equal{#1}{data}}    {
\begin{table*}
\centering
\scalebox{0.7}{
\begin{tabular}{|c||c|c|c|c|c|} \hline
\textbf{тип данных}  & \multicolumn{2}{c|}{\textbf{однометочные}} & \multicolumn{2}{c|}{\textbf{многометочные}} & \multirow{2}{*}{\textbf{равноразмерные}}\\
\cline{1-5}
\textbf{класс}  & \multicolumn{1}{c|}{все} & \multicolumn{1}{c|}{отвеченные} & \multicolumn{1}{c|}{все} & \multicolumn{1}{c|}{отвеченные} & \\\hline \hline
\textit{Размер набора данных} & 361650 & 266597 & 170930 & 137341 & 264786\\ \hline
\textit{Размер 6классового поднабора данных} & 18864 & 15912 & 27191 & 20569 & 15830 \\ \hline
\textit{Музыка} & 9514 & 5809 & 4456 & 3287 & 5797 \\ \hline
\textit{Еда, напитки и кулинария} & 5750 & 4758 & 14096 & 11084 & 4723 \\ \hline
\textit{Медиа и коммуникации} & 4505 & 2637 & 5577 & 3948 & 2619 \\ \hline
\textit{Транспорт} & 2435 & 1625 & 1933 & 1387 & 1613 \\ \hline
\textit{Новости} & 945 & 602 & 912 & 720 & 600\\ \hline
\textit{Погода} & 890 & 481 & 217 & 143 & 478 \\ \hline
\end{tabular}
}
\caption{Размеры набора данных {RuQTopics} по классу и части}
%\centering
\label{tab:RuQTopics:sizes}
\end{table*}
    }
   \ifthenelse{\equal{#1}{languages}}    {
\begin{table*}[!htbp]
\caption{Метрики модели \textit{bert-base-multilingual-cased} на объединенном тестовом наборе данных {MASSIVE} для всех языков. Модель обучалась на версии \textbf{Q} набора данных {RuQTopics}. \textbf{Код} означает код языка(ISO 639-1), \textbf{N} означает число статей в Википедии на этом языке на 11 октября 2018 года, \textbf{Дист} означает лингвистическую дистанцию между этим языком и русским, посчитанную в соответствии с работой~\cite{lang_sim}. Усреднено по трем запускам.}
\label{tab:rutopics:crosslingual}
% \centering
% \scalebox{0.7}{
\begin{minipage}{0.5\textwidth}
\scalebox{0.55}{
\begin{tabular}[baseline={(0,2.1)}]{|c|c|c|c|c|c|} \hline
\multirow{2}{*}{\textbf{Язык}}  & \multirow{2}{*}{\textbf{Код}} & \multirow{2}{*}{\textbf{Дист}} & \multirow{2}{*}{\textbf{N}}  &  \multicolumn{2}{c|}{\textbf{Метрики}} \\ %\hline
\cline{5-6}
& & & & Точность & Макро-F1 \\ \hline \hline
русский & ru & 0 & 1,501,878 & 80.8 & 79.8\\
китайский (Тайвань) & zh-TW & 92.2 & 1,025,366 & 79.6 & 79.1\\
китайский & zh & 92.2 & 1,025,366 & 78.0 & 77.7\\
английский & en & 60.3 & 5,731,625 & 75.2 & 75.6\\
японский & ja & 93.3 & 1,124,097 & 72.4 & 70.5\\
словенский & sl & 4.2 & 162,453 & 70.3 & 69.0\\
шведский & sv & 59.5 & 3,763,579 & 70.2 & 69.6\\
малайский & ms & n/c & 320,631 & 68.9 & 67.7\\
итальянский & it & 45.8 & 1,466,064 & 68.8 & 68.0\\
индонезийский & id & 91.2 & 440,952 & 68.7 & 67.5\\
нидерландский & nl & 64.6 & 1,944,129 & 68.7 & 68.5\\
португальский & pt & 61.6 & 1,007,323 & 68.6 & 68.7\\
испанский & es & 51.7 & 1,480,965 & 68.2 & 68.0\\
датский & da & 66.2 & 240,436 & 67.8 & 66.7\\
французский & fr & 61.0 & 2,046,793 & 65.5 & 65.5\\
персидский & fa & 72.4 & 643,750 & 65.2 & 64.2\\
турецкий & tr & 86.2 & 316,969 & 64.5 & 62.4\\
вьетнамский & vi & 95.0 & 1,190,187 & 64.3 & 65.1\\
норвежский букмол & nb & 67.2 & 495,395 & 64.3 & 64.0\\
польский & pl & 5.1 & 1,303,297 & 64.2 & 62.2\\
азербайджанский & az & 87.7 & 138,538 & 63.9 & 63.1\\
каталанский & ca & 60.3 & 591,783 & 61.4 & 60.4\\
венгерский & hu & 87.2 & 437,984 & 61.3 & 60.0\\
иврит & he & 88.9 & 231,868 & 60.9 & 59.5\\
хинди & hi & 69.8 & 127,044 & 60.7 & 58.7\\
корейский & ko & 89.5 & 429,369 & 60.4 & 59.6\\
\hline
\end{tabular}
}
\end{minipage}
\begin{minipage}{0.5\textwidth}
\scalebox{0.55}{
\begin{tabular}[baseline={(0,2.1)}]{|c|c|c|c|c|c|} \hline
\multirow{2}{*}{\textbf{Язык}}  & \multirow{2}{*}{\textbf{Код}} & \multirow{2}{*}{\textbf{Дист}} & \multirow{2}{*}{\textbf{N}}  &  \multicolumn{2}{c|}{\textbf{Метрики}} \\ %\hline
\cline{5-6}
& & & & Точность & Макро-F1 \\ \hline \hline
румынский & ro & 55.0 & 388,896 & 57.1 & 53.9\\
урду & ur & 66.7 & 140,939 & 56.4 & 55.9\\
арабский & ar & 86.5 & 619,692 & 56.2 & 55.7\\
каннада & kn & 90.8 & 23,844 & 56.1 & 53.0\\
филиппинский & tl & 91.9 & 80,992 & 55.0 & 51.3\\
телугу & te & 96.7 & 69,354 & 53.7 & 49.3\\
финский & fi & 88.9 & 445,606 & 53.3 & 51.3\\
бирманский & my & 86.0 & 39,823 & 52.5 & 49.7\\
африкаанс & af & 64.8 & 62,963 & 52.4 & 50.3\\
тамильский & ta & 94.7 & 118,119 & 52.4 & 50.1\\
немецкий & de & 64.5 & 2,227,483 & 52.2 & 51.6\\
албанский & sq & 69.4 & 74,871 & 51.5 & 47.2\\
латышский & lv & 49.1 & 88,189 & 49.6 & 48.4\\
малаялам & ml & 96.7 & 59,305 & 48.7 & 46.3\\
армянский & hy & 77.8 & 246,571 & 48.1 & 47.5\\
бенгальский & bn & 66.3 & 61,294 & 47.3 & 45.3\\
тайский & th & 89.5 & 127,010 & 46.5 & 44.9\\
греческий & el & 75.3 & 153,855 & 46.3 & 44.8\\
грузинский & ka & 96.0 & 124,694 & 39.2 & 38.1\\
яванский & jv & 95.4 & 54,964 & 38.7 & 37.1\\
монгольский & mn & 86.2 & 18,353 & 36.6 & 33.7\\
исландский & is & 68.9 & 45,873 & 32.6 & 29.9\\
суахили & sw & 95.1 & 45,806 & 31.0 & 28.0\\
валлийский & cy & 75.5 & 101,472 & 28.5 & 25.3\\
кхмерский & km & 97.1 & 6,741 & 16.1 & 8.6\\
амхарский & am & 86.6 & 14,375 & 12.1 & 5.0\\ \hline
\hline
\end{tabular}
}
\end{minipage}
\end{table*}
}
}

\newcommand{\mtlEN}[1]{%
    \ifthenelse{\equal{#1}{en}}    {
\begin{table*}
 \caption{Метрики англоязычных моделей (точность/f1 macro) для пяти англоязычных диалоговых задач.Режим S означает однозадачные модели, режим M означает многозадачные модели. Усреднено по трем запускам.}
 \label{tab:tr-ag:en_results}
\centering
\resizebox{\textwidth}{!}{%
\begin{tabular}{|c|c|c|c|c|c|c|c|c|}
\hline
\multirow{2}{*}{Модель} & \multirow{2}{*}{Режим} & \multirow{2}{*}{Среднее} & Эмоции & Тональность & Токсичность & Интенты & Темы & Число \\
& & & 39.4k & 80.5k & 127.6k & 11.5k & 11.5k & батчей \\ \hline \hline
\textit{\multirow{2}{*}{distilbert-base-cased}} & S & \textbf{82.9/78.4} & \textbf{70.3/63.1} & 74.7/74.3 & 91.5/81.2 & \textbf{87.4/82.7} & \textbf{91.0/90.6} & 11390 \\ 
 & M  & 82.1/77.2 & 67.7/60.7 & \textbf{75.2/75.0} & 90.6/79.8 & 86.3/80.4 & 90.8/90.1 & 14000 \\ \hline
\textit{\multirow{2}{*}{bert-base-cased}} & S & \textbf{83.9/79.7} & \textbf{71.2/64.2} & 76.1/75.8 & \textbf{93.2/83.5} & \textbf{87.9/84.2} & \textbf{91.3/90.7} & 9470 \\
 & M &  83.0/78.4 & 69.0/63.1 & \textbf{76.5/76.4} & 91.4/80.8 & 87.1/81.2 & 91.2/90.6 & 11760 \\ \hline
\textit{\multirow{2}{*}{bert-large-cased}} & S &  \textbf{84.7/80.5} & \textbf{70.9/64.4} & \textbf{80.5/80.4} & \textbf{92.1/82.2} & \textbf{88.4/84.9} & 91.3/90.7 & 8526 \\ 
 & M  & 83.6/78.7 & 69.0/61.8 & 79.0/78.9 & 91.3/80.9 & 87.3/80.9 & \textbf{91.3/90.8} & 11200 \\ \hline
 \end{tabular}
 }
\end{table*}
}
\ifthenelse{\equal{#1}{ensmall}}{
\begin{table*}
\caption{Точность/ F1 для запусков на части тренировочных данных.  Режим M означает многозадачные модели, режим S означает однозадачные модели, и Доля означает долю использованных тренировочных данных. Базовая модель \textit{distilbert-base-cased}. Усреднено по пяти запускам. }
\label{tab:tr-ag:en_dialog_part}
\resizebox{\textwidth}{!}{%
\begin{tabular}{|c|c|c|c|c|c|c|c|c|}
\hline
 \multirow{2}{*}{Режим} &  \multirow{2}{*}{Доля} &  \multirow{2}{*}{Среднее} & Эмоции & Тональность & Токсичность & Интенты & Темы & Число \\
& & & 39.4k & 80.5k & 127.6k & 11.5k & 11.5k & батчей \\
\hline \hline
S & 15\% & 78.5/70.9 & 65.8/50.6 & 69.3/68.8 & 92.2/81.2 & 78.7/68.8 & 86.3/85.1 & 2173 \\
M & 15\% & 77.4/70.6 & 64.0/55.0 & 68.3/67.7 & 91.6/80.0 & 76.9/64.6 & 86.4/85.4 & 4741 \\ \hline
S & 10\% & 77.1/68.4 & 64.6/45.0 & 68.3/67.8 & 92.2/81.0 & 75.5/64.7 & 84.8/83.3 & 1579 \\
M & 10\% & 75.9/69.1 & 62.6/53.6 & 66.6/65.8 & 91.5/79.7 & 74.3/63.2 & 84.6/83.3 & 4295 \\ \hline
S & 9\% & 76.7/67.4 & 64.6/43.9 & 68.2/67.7 & 91.8/80.4 & 74.4/62.7 & 84.2/82.4 & 1457 \\
M & 9\% & 75.4/67.8 & 62.1/52.4 & 66.5/65.7 & 91.4/79.5 & 72.4/58.5 & 84.4/83.0 & 3695 \\ \hline
%S & 8\% & 73.8/64.6 & 63.7/43.5 & 67.6/67.0 & 92.0/80.5 & 62.0/49.7 & 83.9/82.1 & 1381 \\
%M & 8\% & 74.9/67.2 & 61.7/51.4 & 66.7/66.0 & 91.5/79.5 & 71.1/57.3 & 83.6/81.7 & 3511 \\ \hline
%S & 7.5\% & 73.5/64.0 & 63.8/42.5 & 67.4/66.9 & 91.6/80.1 & 61.3/48.6 & 83.6/81.7 & 1293 \\
%M & 7.5\% & 74.6/66.9 & 61.0/50.2 & 67.0/66.4 & 91.5/79.5 & 70.3/56.9 & 83.2/81.4 & 2995 \\ \hline
S & 7\% & 73.5/64.0 & 63.3/42.1 & 67.9/67.4 & 91.8/80.1 & 61.4/49.4 & 83.3/81.1 & 1251 \\
M & 7\% & 74.2/66.4 & 61.1/50.4 & 65.8/65.1 & 91.0/78.9 & 70.0/56.3 & 83.1/81.3 & 2882 \\ \hline
S & 5\% & 69.1/59.0 & 62.5/38.9 & 66.9/66.3 & 91.8/79.9 & 42.7/30.8 & 81.6/78.8 & 901 \\
M & 5\% & 71.7/62.4 & 60.5/48.6 & 64.4/63.4 & 90.8/78.5 & 62.4/44.4 & 80.2/77.3 & 2381 \\ \hline
S & 3\% & 59.6/49.0 & 60.6/37.7 & 65.2/64.5 & 91.8/79.3 & 26.5/16.9 & 54.0/46.5 & 584 \\
M & 3\% & 68.8/58.1 & 58.6/42.7 & 62.5/61.3 & 91.0/78.3 & 55.5/37.1 & 76.4/71.0 & 1566 \\ \hline
S & 2\% & 44.9/31.5 & 48.7/21.7 & 39.6/26.5 & 91.8/79.0 & 2.6/0.2 & 41.4/30.1 & 274 \\
M & 2\% & 64.8/52.1 & 57.6/38.4 & 61.4/60.1 & 90.8/78.0 & 44.2/23.5 & 69.9/60.4 & 923 \\ \hline
\end{tabular}
}
\end{table*}
}
}


\newcommand{\mtlRU}[1]{
\ifthenelse{\equal{#1}{ru}}    {
\begin{table*}
 \caption{Метрики русскоязычных моделей(точность/f1 macro) для пяти диалоговых задач. Режим S означает однозадачные модели, режим M означает многозадачные модели. Усреднено по трем запускам.}
 \label{tab:tr-ag:ru_results}
\centering
\resizebox{\textwidth}{!}{%
\begin{tabular}{|c|c|c|c|c|c|c|c|c|}
\hline
\multirow{2}{*}{Модель} & \multirow{2}{*}{Режим} & \multirow{2}{*}{Среднее} & Эмоции & Тональность & Токсичность & Интенты & Темы & Число \\
& & & 6.5k & 82.6k & 93.3k & 11.5k & 11.5k & батчей \\ \hline \hline
\textit{\multirow{2}{*}{DeepPavlov/distilrubert-base-cased-conversational}} & S & 86.9/84.1 & 82.2/76.1 & 77.9/78.2 & 97.1/95.4 & 86.7/81.6 & 90.4/89.5 & 8472 \\
 & M & 86.3/82.6 & 81.0/74.6 & 77.7/77.7 & 96.9/95.0 & 85.2/75.9 & 90.7/89.9 & 8540 \\ \hline
\textit{\multirow{2}{*}{DeepPavlov/rubert-base-cased-conversational}} & S & 86.5/83.4 & 80.9/75.3 & 78.0/78.2 & 97.2/95.6 & 86.2/79.1 & 90.0/89.0 & 7999 \\ 
 & M & 86.2/82.6 & 80.5/73.8 & 77.6/77.6 & 96.8/95.0 & 85.3/76.9 & 90.5/89.8 & 8113 \\ \hline
 \end{tabular}
 }
 \end{table*}
}
\ifthenelse{\equal{#1}{rusmall}}{ \begin{table*}
\caption{ Влияние добавления англоязычных данных. Точность/ f1-макро на русскоязычных данных для трансформер-инвариантного \textit{distilbert-base-multilingual-cased}. Число батчей означает число батчей, которые видела модель в процессе своего обучения, режим S означает однозадачное обучение, режим M -- многозадачное обучение. RU доля -- доля тренировочных примеров, взятых из каждого русскоязычного тренировочного набора данных, и EN  доля -- доля примеров, взятых из каждого англоязычного тренировочного набора данных. Усреднено по 3 запускам, кроме указанных выше экспериментов, для которых усреднение проводилось по 5 запускам.}
\label{tab:mult_smalldata_results}
\resizebox{\textwidth}{!}{\begin{tabular}{|c|c|c|c||c|c|c|c|c|c||c|} \hline
Режим & \begin{tabular}[c]{@{}l@{}}RU\\доля\end{tabular} & \begin{tabular}[c]{@{}l@{}}EN\\доля\end{tabular} &\begin{tabular}[c]{@{}l@{}}Валидация\\на\end{tabular} & Среднее & \begin{tabular}[c]{@{}l@{}}Эмоции\end{tabular} & \begin{tabular}[c]{@{}l@{}}Тональность\end{tabular} & \begin{tabular}[c]{@{}l@{}}Токсичность\end{tabular} & \begin{tabular}[c]{@{}l@{}}Интенты\end{tabular} & \begin{tabular}[c]{@{}l@{}}Темы\end{tabular} & \begin{tabular}[c]{@{}l@{}}Число\\ батчей\end{tabular} \\\hline \hline
S & 100\% & 100\% & EN & 85.2/82.1 & 79.0/70.8 & 77.2/77.4 & 96.5/94.5 & 84.5/80.6 & 88.4/87.4 & 15946 \\
M & 100\% & 100\% & EN & 84.4/80.9 & 77.2/70.5 & 75.8/75.8 & 96.4/94.4 & 83.5/76.3 & 88.9/87.8 & 20737 \\ \hline
S & 50\% & 100\% & EN & 83.2/79.5 & 75.6/65.8 & 75.6/75.7 & 96.1/93.9 & 82.2/76.5 & 86.8/85.5 & 16672 \\
M & 50\% & 100\% & EN & 82.8/78.1 & 76.2/64.5 & 74.0/73.4 & 95.9/93.5 & 80.9/72.7 & 87.2/86.1 & 19336 \\ \hline
S & 25\% & 100\% & EN & 81.4/76.7 & 73.7/61.4 & 73.7/73.9 & 95.5/92.7 & 78.8/71.9 & 85.1/83.6 & 16589 \\
M & 25\% & 100\% & EN & 80.9/76.4 & 73.1/63.9 & 73.7/73.7 & 95.1/92.2 & 77.5/68.1 & 85.3/83.9 & 16665 \\ \hline
S & 20\% & 100\% & EN & 80.6/76.0 & 71.8/60.3 & 74.0/74.0 & 95.1/92.1 & 78.0/71.1 & 83.9/82.4 & 12951 \\
M & 20\% & 100\% & EN & 80.1/75.0 & 71.9/61.2 & 73.5/73.5 & 94.9/91.9 & 76.1/65.5 & 84.2/82.8 & 17429 \\ \hline
S & 15\% & 100\% & EN & 79.7/74.7 & 70.8/57.8 & 72.6/72.7 & 94.6/91.3 & 77.3/70.1 & 83.1/81.6 & 13037 \\
M & 15\% & 100\% & EN & 78.9/73.5 & 70.0/58.4 & 71.9/71.5 & 94.5/91.2 & 74.7/65.0 & 83.5/81.8 & 15599 \\ \hline
S & 10\% & 100\% & EN & 77.9/72.0 & 68.3/52.1 & 72.3/72.7 & 93.9/90.0 & 73.9/65.8 & 81.2/79.4 & 13545 \\
M & 10\% & 100\% & EN & 77.4/70.9 & 67.9/51.2 & 71.7/71.7 & 93.7/90.1 & 72.3/61.5 & 81.6/79.9 & 14471 \\ \hline
S & 5\% & 100\% & EN & 75.0/67.9 & 64.1/45.0 & 70.2/70.4 & 92.7/87.8 & 69.9/60.5 & 77.9/75.8 & 12567 \\
M & 5\% & 100\% & EN & 74.2/66.3 & 63.4/41.2 & 70.1/70.2 & 92.3/87.6 & 67.6/56.6 & 77.7/75.9 & 12779 \\ \hline
S & 3\% & 100\% & EN & 71.8/64.6 & 59.1/38.8 & 68.4/68.6 & 91.0/85.6 & 65.9/57.4 & 74.6/72.6 & 12065 \\
M & 3\% & 100\% & EN & 70.7/64.2 & 58.5/44.8 & 67.8/67.7 & 90.9/85.5 & 62.4/51.3 & 74.0/71.6 & 14896 \\ \hline
S & 0\% & 100\% & EN & 52.4/42.0 & 48.3/26.6 & 43.8/43.1 & 80.0/58.6 & 37.5/31.5 & 52.3/50.4 & 15469 \\
M & 0\% & 100\% & EN & 51.0/41.5 & 42.6/23.8 & 45.4/42.8 & 78.6/61.6 & 38.0/30.6 & 50.0/48.4 & 14000 \\ \hline
S & 100\% & 0\% & RU & 84.4/80.4 & 76.5/66.5 & 77.2/77.3 & 96.7/94.7 & 83.5/76.4 & 88.2/87.0 & 11199 \\
M & 100\% & 0\% & RU & 84.3/80.4 & 77.9/70.4 & 76.4/76.5 & 96.5/94.4 & 82.3/73.3 & 88.4/87.4 & 11956 \\ \hline
S & 50\% & 0\% & RU & 82.5/78.0 & 74.0/63.2 & 76.4/76.4 & 96.1/93.8 & 80.0/71.9 & 86.1/84.8 & 5878 \\
M & 50\% & 0\% & RU & 82.3/78.0 & 75.0/66.5 & 74.6/74.7 & 96.0/93.7 & 79.5/69.8 & 86.4/85.2 & 8090 \\ \hline
S & 25\% & 0\% & RU & 79.5/72.5 & 67.0/45.0 & 75.1/75.4 & 95.4/92.5 & 76.6/68.1 & 83.6/81.5 & 3496 \\
M & 25\% & 0\% & RU & 79.6/74.3 & 72.3/62.1 & 72.7/72.8 & 95.3/92.6 & 73.8/61.5 & 83.7/82.1 & 5830 \\ \hline
S & 20\% & 0\% & RU & 78.4/70.3 & 64.3/36.2 & 74.4/74.7 & 95.1/92.0 & 75.3/67.5 & 82.9/81.0 & 2796 \\
M & 20\% & 0\% & RU & 79.0/74.2 & 71.4/61.4 & 73.0/73.2 & 95.0/91.9 & 73.5/63.8 & 82.3/80.8 & 5773 \\ \hline
S & 15\% & 0\% & RU & 77.7/70.1 & 66.1/44.5 & 74.0/74.0 & 94.8/91.5 & 72.2/61.2 & 81.6/79.5 & 1997 \\
M & 15\% & 0\% & RU & 77.2/71.3 & 70.7/59.6 & 71.7/72.0 & 94.6/91.4 & 68.6/54.9 & 80.6/78.7 & 5320 \\ \hline
S & 10\% & 0\% & RU & 75.7/67.1 & 64.5/41.1 & 73.3/73.5 & 93.9/90.0 & 67.7/54.7 & 78.8/76.2 & 1469 \\
M & 10\% & 0\% & RU & 75.2/68.2 & 68.7/55.3 & 71.5/71.7 & 94.0/90.2 & 64.0/48.4 & 77.8/75.5 & 2836 \\ \hline
S & 5\% & 0\% & RU & 58.4/47.9 & 48.3/20.3 & 71.0/71.1 & 92.7/87.9 & 29.9/18.2 & 50.1/41.8 & 739 \\
M & 5\% & 0\% & RU & 70.3/61.6 & 64.8/48.3 & 70.1/70.3 & 92.6/88.0 & 53.0/35.0 & 71.2/66.3 & 2095 \\ \hline
S & 3\% & 0\% & RU & 57.0/45.2 & 49.1/20.5 & 69.5/69.6 & 91.5/85.8 & 38.9/24.7 & 36.2/25.6 & 521 \\
M & 3\% & 0\% & RU & 65.9/55.1 & 62.6/41.3 & 69.0/69.2 & 91.2/85.6 & 42.6/24.2 & 63.9/55.1 & 1132 \\ \hline
\end{tabular}
}
\end{table*}}
}

\newcommand{\mtlPLOT}[1]{
\ifthenelse{\equal{#1}{ruplot}}{
\label{fig:thresholds_acc_ru}
\begin{figure}[!htbp]
\begin{minipage}{0.55\textwidth}
\begin{tikzpicture}[baseline={(0,2.1)}]%[scale=2]
\begin{axis}[xlabel = RU доля,
ylabel = Средняя точность,
legend pos= south east,
width=0.9\textwidth,
xtick={0,1,2,3,4,5,6,7,8},
xticklabels={2,3,5,10,15,20,25,50,100},
ymin=50,ymax=90,
legend cell align={left},
legend style={nodes={scale=0.5, transform shape}}
]
\addplot[color=orange,dotted, mark=*] coordinates {
(1, 57.02)
(2, 58.379999999999995)
(3, 75.66666666666667)
(4, 77.7)
(5, 78.36666666666667)
(6, 79.53333333333333)
(7, 82.5)
(8, 84.39999999999999)
};
\addlegendentry{S точность (RU)}
\addplot[color=red,solid,mark=*] coordinates {
(1, 65.88)
(2, 70.3)
(3, 75.23333333333333)
(4, 77.23333333333333)
(5, 79.0)
(6, 79.60000000000001)
(7, 82.3)
(8, 84.33333333333333)
};
\addlegendentry{M точность (RU)}
\addplot[color=cyan,dashed, mark=*] coordinates {
(1, 70.73333333333333)
(2, 74.23333333333333)
(3, 77.39999999999999)
(4, 78.89999999999999)
(5, 80.13333333333334)
(6, 80.93333333333332)
(7, 82.83333333333333)
(8, 84.36666666666667)
};
\addlegendentry{S точность (RU+EN)}
\addplot[color=blue,dashed,mark=*] coordinates {
(1, 71.8)
(2, 74.96666666666667)
(3, 77.93333333333334)
(4, 79.66666666666667)
(5, 80.56666666666666)
(6, 81.36666666666666)
(7, 83.23333333333333)
(8, 85.16666666666667)
};
\addlegendentry{M точность (RU+EN)}
\end{axis}%
\end{tikzpicture}
% \hspace{3mm}
% \captionof{figure}
% \caption{Av}
\end{minipage}
\begin{minipage}{0.45\textwidth}
{
\scalebox{0.8}{
\begin{tabular}[baseline={(0,2.1)}]{|l||c|c|c|c|}
\hline
RU & S & M & S & M \\
доля & RU & RU & RU+EN & RU+EN \\ % Sizes are different for RU and RU+EN, so I don't give them here
\hline \hline
3 & 71.8 & 70.7 & 57.0 & 65.9 \\ \hline
5 & 75.0 & 74.2 & 58.4 & 70.3 \\ \hline
10 & 77.9 & 77.4 & 75.7 & 75.2 \\ \hline
15 & 79.7 & 78.9 & 77.7 & 77.2 \\ \hline
20 & 80.6 & 80.1 & 78.4 & 79.0 \\ \hline
25 & 81.4 & 80.9 & 79.5 & 79.6 \\ \hline
50 & 83.2 & 82.8 & 82.5 & 82.3 \\ \hline
100 & 85.2 & 84.4 & 84.4 & 84.3 \\ \hline
\end{tabular}}
}
\end{minipage}
\caption{Средняя точность на русскоязычных данных для \textit{distilbert-base-multilingual-cased}. S означает однозадачный режим, M означает многозадачный режим, RU доля означает долю русскоязычных обучающих данных, использованных при обучении, RU означает только обучение на этой доле русскоязычных данных, RU+EN означает обучение на этой доле русскоязычных обучающих данных плюс на 100 процентах англоязычных обучающих данных.}
\label{fig:tr-ag:ru_dialog_part}
\end{figure}}
\ifthenelse{\equal{#1}{enplot}}{
\begin{figure}
\centering
\caption{Средняя точность для англоязычных диалоговых задач. Эффект уменьшения тренировочных данных. Результаты усреднены по пяти запускам.}
\begin{tikzpicture}
\label{fig:tr-ag:en_dialog_part}
\begin{axis}[xlabel = Процент используемых тренировочных данных,
ylabel = Средняя точность,
legend pos= south east,
% width=10cm,
% height=10cm,
% xmin=2,
% xmax=100,
xtick={2,3,5,7,9,10,15},
ymin=40,ymax=85,
legend cell align={left},
legend style={nodes={scale=0.7, transform shape}}
]
\addplot[color=blue,solid, mark=*] coordinates {
(2, 64.8)%50
(3, 68.8)%50
(5, 71.7)%100
(7,74.2)
%(7.5,74.6)
% (8,74.9)
(9,75.4)
(10, 75.9)%300
(15, 77.4)
};
\addlegendentry{Многозадачные модели}
\addplot[color=green,dashed,mark=*] coordinates {
(2, 44.9)%50
(3, 59.6)%100
(5, 69.1)%300
(7,73.5)
%(7.5,73.5)
% (8,73.8)%500
(9,76.7)
(10,77.1)
(15, 78.5)
};
\addlegendentry{Однозадачные модели}
\end{axis}%
\end{tikzpicture}%
\end{figure}
}
}

\newcommand{\mtlPLOTS}[1]{
\ifthenelse{\equal{#1}{enplots}}{
\pgfplotsset{every tick label/.append style={font=\tiny}}
\begin{figure}[!ht]
\begin{subfigure}{0.48\textwidth}
\begin{tikzpicture}[scale=1]
\begin{axis}[xlabel = Число тренировочных примеров,
ylabel = Точность,
title=Средняя точность,
legend pos= south east,
width=\textwidth,
% width=10cm,
% height=10cm,
% xmin=2,
% xmax=100,
xtick={0,1,2,3,4,5,6,7},
xticklabels={5k, 8k, 14k, 19k, 24k, 27k, 41k, 271k},
ymin=40,ymax=90,
legend cell align={left},
legend style={nodes={scale=0.5, transform shape}}
]
\addplot[color=orange,dotted, mark=*] coordinates {
(0, 44.9)
(1, 59.6)
(2, 69.1)
(3, 73.5)
(4, 76.7)
(5, 77.1)
(6, 78.5)
(7, 82.9)
};
\addlegendentry{Однозадачная точность (EN)}
\addplot[color=red,solid,mark=*] coordinates {
(0, 64.8)
(1, 68.8)
(2, 71.7)
(3, 74.2)
(4, 75.4)
(5, 75.9)
(6, 77.4)
(7, 82.1)
};
\addlegendentry{Многозадачная точность (EN)}
\end{axis}
\end{tikzpicture}
\end{subfigure}
\begin{subfigure}{0.48\textwidth}
\begin{tikzpicture}[scale=1]
\begin{axis}[xlabel = Число обучающих примеров,
ylabel = Точность,
title=Классификация эмоций,
legend pos= south east,
width=\textwidth,
% width=10cm,
% height=10cm,
% xmin=2,
% xmax=100,
xtick={0,1,2,3,4,5,6,7},
xticklabels={0.8k, 1.2k, 2k, 2.7k, 3.6k, 3.9k, 5.9k, 39.5k},
ymin=45,ymax=80,
legend cell align={left},
legend style={nodes={scale=0.5, transform shape}}
]

\addplot[color=orange,dotted, mark=*] coordinates {
(0, 48.7)
(1, 60.6)
(2, 62.5)
(3, 63.3)
(4, 64.6)
(5, 64.6)
(6, 65.8)
(7, 70.3)
};
\addlegendentry{Однозадачная точность (EN)}
\addplot[color=red,solid,mark=*] coordinates {
(0, 57.6)
(1, 58.6)
(2, 60.5)
(3, 61.1)
(4, 62.1)
(5, 62.6)
(6, 64.0)
(7, 67.7)};
\addlegendentry{Многозадачная точность (EN)}
\end{axis}%
\end{tikzpicture}
\end{subfigure}

\medskip

\begin{subfigure}{0.48\textwidth}
\begin{tikzpicture}[scale=1]
\begin{axis}[xlabel = Число обучающих примеров,
ylabel = Точность,
title=Классификация тональности,
legend pos= south east,
width=\textwidth,
% width=10cm,
% height=10cm,
% xmin=2,
% xmax=100,
xtick={0,1,2,3,4,5,6,7},
xticklabels={2.6k, 3.8k, 6.4k, 8.9k, 11.5k, 12.7k, 19.1k, 127.7k},
ymin=35,ymax=80,
legend cell align={left},
legend style={nodes={scale=0.5, transform shape}}
]
\addplot[color=orange,dotted, mark=*] coordinates {
(0, 39.6)
(1, 65.2)
(2, 66.9)
(3, 67.9)
(4, 68.2)
(5, 68.3)
(6, 69.3)
(7, 74.7)
};
\addlegendentry{Однозадачная точность (EN)}
\addplot[color=red,solid,mark=*] coordinates {
(0, 61.4)
(1, 62.5)
(2, 64.4)
(3, 65.8)
(4, 66.5)
(5, 66.6)
(6, 68.3)
(7, 75.2)};
\addlegendentry{Многозадачная точность (EN)}
\end{axis}%
\end{tikzpicture}
\end{subfigure}
\begin{subfigure}{0.48\textwidth}
\begin{tikzpicture}[scale=1]
\begin{axis}[xlabel = Число обучающих примеров,
ylabel = Точность,
title=Классификация токсичности,
legend pos= south east,
width=\textwidth,
% width=10cm,
% height=10cm,
% xmin=2,
% xmax=100,
xtick={0,1,2,3,4,5,6,7},
xticklabels={1.6k, 2.4k, 4.0k, 5.6k, 7.2k, 8k, 12k, 80k},
ymin=85,ymax=95,
legend cell align={left},
legend style={nodes={scale=0.5, transform shape}}
]
\addplot[color=orange,dotted, mark=*] coordinates {
(0, 91.8)
(1, 91.8)
(2, 91.8)
(3, 91.8)
(4, 91.8)
(5, 92.2)
(6, 92.2)
(7, 91.5)
};
\addlegendentry{Однозадачная точность (EN)}
\addplot[color=red,solid,mark=*] coordinates {
(0, 90.8)
(1, 91.0)
(2, 90.8)
(3, 91.0)
(4, 91.4)
(5, 91.5)
(6, 91.6)
(7, 90.6)};
\addlegendentry{Многозадачная точность (EN)}
\end{axis}%
\end{tikzpicture}
\end{subfigure}

\medskip

\begin{subfigure}{0.48\textwidth}
\begin{tikzpicture}
\begin{axis}[xlabel = Число обучающих примеров,
ylabel = Точность,
title=Классификация интентов,
legend pos= south east,
width=\textwidth,
% width=10cm,
% height=10cm,
% xmin=2,
% xmax=100,
xtick={1,2,3,4,5,6,7},
xticklabels={0.3k, 0.6k, 1.2k, 1.7k, 2.3k, 2.9k, 5.8k, 11.5k},
ymin=25,ymax=85,
legend cell align={left},
legend style={nodes={scale=0.5, transform shape}}
]
\addplot[color=orange,dotted, mark=*] coordinates {
%(0, 2.6)
(1, 26.5)
(2, 42.7)
(3, 61.4)
(4, 74.4)
(5, 75.5)
(6, 78.7)
(7, 87.4)
};
\addlegendentry{Однозадачная точность (EN)}
\addplot[color=red,solid,mark=*] coordinates {
%(0, 44.2)
(1, 55.5)
(2, 62.4)
(3, 70.0)
(4, 72.4)
(5, 74.3)
(6, 76.9)
(7, 86.3)};
\addlegendentry{Многозадачная точность (EN)}
\end{axis}
\end{tikzpicture}
\end{subfigure}
\begin{subfigure}{0.48\textwidth}
\begin{tikzpicture}
\begin{axis}[xlabel = Число обучающих примеров,
ylabel = Точность,
title=Тематическая классификация,
legend pos= south east,
width=\textwidth,
% width=10cm,
% height=10cm,
% xmin=2,
% xmax=100,
xtick={0,1,2,3,4,5,6,7},
xticklabels={0.3k, 0.6k, 1.2k, 1.7k, 2.3k, 2.9k, 5.8k, 11.5k},
ymin=35,ymax=90,
legend cell align={left},
legend style={nodes={scale=0.5, transform shape}}
]
\addplot[color=orange,dotted, mark=*] coordinates {
(0, 41.4)
(1, 54.0)
(2, 81.6)
(3, 83.3)
(4, 84.2)
(5, 84.8)
(6, 86.3)
(7, 91.0)
};
\addlegendentry{Однозадачная точность (EN)}
\addplot[color=red,solid,mark=*] coordinates {
(0, 69.9)
(1, 76.4)
(2, 80.2)
(3, 83.1)
(4, 84.4)
(5, 84.6)
(6, 86.4)
(7, 90.8)
};
\addlegendentry{Многозадачная точность (EN)}
\end{axis}%
\end{tikzpicture}
\end{subfigure}
\caption{Однозадачная и многозадачная точность для каждой задачи. Английский язык.}
\label{fig:tr-ag:en_dialog_part_n_samples}
\end{figure}
}
\ifthenelse{\equal{#1}{ruplots}}{
\pgfplotsset{every tick label/.append style={font=\tiny}}

\begin{figure}[!ht]
\begin{subfigure}{0.48\textwidth}
\begin{tikzpicture}[scale=1]
\begin{axis}[xlabel = Число тренировочных примеров,
ylabel = Точность,
title=Средняя точность,
legend pos= south east,
width=\textwidth,
% width=10cm,
% height=10cm,
% xmin=2,
% xmax=100,
xtick={0,1,2,3,4,5,6,7},
xticklabels={6k,10k,21k,31k,41k,51k,103k,205k},
ymin=50,ymax=90,
legend cell align={left},
legend style={nodes={scale=0.5, transform shape}}
]
\addplot[color=orange,dotted, mark=*] coordinates {
(0, 57.02)
(1, 58.379999999999995)
(2, 75.66666666666667)
(3, 77.7)
(4, 78.36666666666667)
(5, 79.53333333333333)
(6, 82.5)
(7, 84.39999999999999)
};
\addlegendentry{Однозадачная точность (RU)}
\addplot[color=red,solid,mark=*] coordinates {
(0, 65.88)
(1, 70.3)
(2, 75.23333333333333)
(3, 77.23333333333333)
(4, 79.0)
(5, 79.60000000000001)
(6, 82.3)
(7, 84.33333333333333)};
\addlegendentry{Многозадачная точность (RU)}
\end{axis}
\end{tikzpicture}
\end{subfigure}
\begin{subfigure}{0.48\textwidth}
\begin{tikzpicture}[scale=1]
\begin{axis}[xlabel = Число обучающих примеров,
ylabel = Точность,
title=Классификация эмоций,
legend pos= south east,
width=\textwidth,
% width=10cm,
% height=10cm,
% xmin=2,
% xmax=100,
xtick={0,1,2,3,4,5,6,7},
xticklabels={196, 327, 655, 983, 1311, 1639, 3278, 6557},
ymin=45,ymax=80,
legend cell align={left},
legend style={nodes={scale=0.5, transform shape}}
]
\addplot[color=orange,dotted, mark=*] coordinates {
(0, 49.14)
(1, 48.260000000000005)
(2, 64.5)
(3, 66.10000000000001)
(4, 64.26666666666667)
(5, 66.96666666666665)
(6, 73.96666666666667)
(7, 76.46666666666667)
};
\addlegendentry{Однозадачная точность (RU)}
\addplot[color=red,solid,mark=*] coordinates {
(0, 62.61999999999999)
(1, 64.82000000000001)
(2, 68.73333333333333)
(3, 70.73333333333333)
(4, 71.36666666666667)
(5, 72.3)
(6, 75.0)
(7, 78.13333333333334)};
\addlegendentry{Многозадачная точность (RU)}
\end{axis}%
\end{tikzpicture}
\end{subfigure}

\medskip

\begin{subfigure}{0.48\textwidth}
\begin{tikzpicture}[scale=1]
\begin{axis}[xlabel = Число обучающих примеров,
ylabel = Точность,
title=Классификация тональности,
legend pos= south east,
width=\textwidth,
% width=10cm,
% height=10cm,
% xmin=2,
% xmax=100,
xtick={0,1,2,3,4,5,6,7},
xticklabels={2k, 4k, 8k, 12k, 17k, 21k, 41k, 83k},
ymin=65,ymax=80,
legend cell align={left},
legend style={nodes={scale=0.5, transform shape}}
]
\addplot[color=orange,dotted, mark=*] coordinates {
(0, 69.46000000000001)
(1, 71.02000000000001)
(2, 73.3)
(3, 73.96666666666665)
(4, 74.36666666666666)
(5, 75.13333333333334)
(6, 76.36666666666667)
(7, 77.16666666666667)
};
\addlegendentry{Однозадачная точность (RU)}
\addplot[color=red,solid,mark=*] coordinates {
(0, 68.99999999999999)
(1, 70.14)
(2, 71.46666666666665)
(3, 71.7)
(4, 72.96666666666667)
(5, 72.66666666666667)
(6, 74.56666666666666)
(7, 76.23333333333333)};
\addlegendentry{Многозадачная точность (RU)}
\end{axis}%
\end{tikzpicture}
\end{subfigure}
\begin{subfigure}{0.48\textwidth}
\begin{tikzpicture}[scale=1]
\begin{axis}[xlabel = Число обучающих примеров,
ylabel = Точность,
title=Классификация токсичности,
legend pos= south east,
width=\textwidth,
% width=10cm,
% height=10cm,
% xmin=2,
% xmax=100,
xtick={0,1,2,3,4,5,6,7},
xticklabels={3k, 5k, 9k, 14k, 19k, 23k, 47k, 93k},
ymin=90,ymax=100,
legend cell align={left},
legend style={nodes={scale=0.5, transform shape}}
]
\addplot[color=orange,dotted, mark=*] coordinates {
(0, 91.50000000000001)
(1, 92.70000000000002)
(2, 93.93333333333334)
(3, 94.80000000000001)
(4, 95.06666666666666)
(5, 95.36666666666667)
(6, 96.13333333333333)
(7, 96.7)
};
\addlegendentry{Однозадачная точность (RU)}
\addplot[color=red,solid,mark=*] coordinates {
(0, 91.24000000000001)
(1, 92.6)
(2, 93.96666666666665)
(3, 94.60000000000001)
(4, 95.0)
(5, 95.33333333333333)
(6, 96.0)
(7, 96.5)};
\addlegendentry{Многозадачная точность (RU)}
\end{axis}%
\end{tikzpicture}
\end{subfigure}

\medskip

\begin{subfigure}{0.48\textwidth}
\begin{tikzpicture}
\begin{axis}[xlabel = Число обучающих примеров,
ylabel = Точность,
title=Классификация интентов,
legend pos= south east,
width=\textwidth,
% width=10cm,
% height=10cm,
% xmin=2,
% xmax=100,
xtick={0,1,2,3,4,5,6,7},
xticklabels={0.3k, 0.6k, 1.2k, 1.7k, 2.3k, 2.9k, 5.8k, 11.5k},
ymin=25,ymax=85,
legend cell align={left},
legend style={nodes={scale=0.5, transform shape}}
]
\addplot[color=orange,dotted, mark=*] coordinates {
(0, 38.9)
(1, 29.920000000000005)
(2, 67.7)
(3, 72.16666666666667)
(4, 75.33333333333333)
(5, 76.60000000000001)
(6, 79.96666666666668)
(7, 83.46666666666665)
};
\addlegendentry{Однозадачная точность (RU)}
\addplot[color=red,solid,mark=*] coordinates {
(0, 42.64)
(1, 52.98)
(2, 64.03333333333335)
(3, 68.6)
(4, 73.5)
(5, 73.83333333333333)
(6, 79.5)
(7, 82.36666666666667)};
\addlegendentry{Многозадачная точность (RU)}
\end{axis}
\end{tikzpicture}
\end{subfigure}
\begin{subfigure}{0.48\textwidth}
\begin{tikzpicture}
\begin{axis}[xlabel = Число обучающих примеров,
ylabel = Точность,
title=Тематическая классификация,
legend pos= south east,
width=\textwidth,
% width=10cm,
% height=10cm,
% xmin=2,
% xmax=100,
xtick={0,1,2,3,4,5,6,7},
xticklabels={0.3k, 0.6k, 1.2k, 1.7k, 2.3k, 2.9k, 5.8k, 11.5k},
ymin=35,ymax=90,
legend cell align={left},
legend style={nodes={scale=0.5, transform shape}}
]
\addplot[color=orange,dotted, mark=*] coordinates {
(0, 36.16)
(1, 50.06)
(2, 78.80000000000001)
(3, 81.56666666666666)
(4, 82.86666666666666)
(5, 83.60000000000001)
(6, 86.13333333333334)
(7, 88.2)
};
\addlegendentry{Однозадачная точность (RU)}
\addplot[color=red,solid,mark=*] coordinates {
(0, 42.64)
(1, 52.98)
(2, 64.03333333333335)
(3, 68.6)
(4, 73.5)
(5, 73.83333333333333)
(6, 79.5)
(7, 82.36666666666667)};
\addlegendentry{Многозадачная точность (RU)}
\end{axis}%
\end{tikzpicture}
\end{subfigure}


\caption{Однозадачная и многозадачная точность для каждой задачи. Русский язык.}
\label{fig:tr-ag:ru_dialog_part_n_samples}
\end{figure}
}
}

\newcommand{\mtlFEATS}[1]{
\ifthenelse{\equal{#1}{englue}}    {
\begin{table*}
\caption{Метрики многозадачной трансформер-инвариантной модели для набора задач GLUE. M.Corr означает корреляцию Мэттью, P/S означает корреляцию Пирсона-Спирмена, Acc точность, F1 -- f1 метрику. Режим S означает однозадачные модели, режим M означает многозадачные модели. Размер означает размер тренировочного набора данных}
\label{tab:tr-ag:mtl_glue}
\centering
\resizebox{\textwidth}{!}{%
\begin{tabular}{|c|c|c|c|c|c|c|c|c|c|c|c|c|}
\hline
\multirow{3}{*}{Модель} & \multirow{3}{*}{Режим}  & Среднее & CoLA & SST-2 & MRPC &STS-B &QQP&MNLI & QNLI & RTE & AX & Число \\
\cline{3-11}
   &   Размер & & 8.6k & 67.3k & 2.5k & 5.7k & 363.8k & 392.7k & 104.7k & 2.5k & как у MNLI & батчей \\ 
\cline{3-11}   
   &  Метрика & & M.Corr & Acc & F1/Acc & P/S Corr & F1/Acc & Acc(m/mm) & Acc & Acc & M.Corr &  \\ \hline \hline
Человек & -- & 87.1 & 66.4 & 97.8 & 86.3/80.8 & 92.7/92.6 & 59.5/80.4 & 92.0/92.8 & 91.2 & 93.6 & - & -\\ \hline
%\textit{\multirow{2}{*}{distilbert-base-cased}} & S & 73.1 & \textbf{42.4} & \textbf{92.1} & 85.6/\textbf{80.3} & 78.8/76.8 & \textbf{69.5/88.5} & \textbf{81.3/80.8} & \textbf{87.5} & 49.8 & 29.9 & 70846 \\ 
\textit{\multirow{2}{*}{distilbert-base-cased}} & S & 73.3 & \textbf{42.4} & \textbf{92.1} & 85.6/\textbf{80.3} & 78.8/76.8 & \textbf{69.5/88.5} & \textbf{81.3/80.8} & \textbf{87.5} & 52.1 & 29.9 & 70861 \\ 
 & M & \textbf{74.5} & 36.0 & 91.0 & \textbf{85.7}/79.9 & \textbf{82.6/81.6} & 68.4/87.4 & 80.4/80.3 & 86.0 & \textbf{69.5} & \textbf{30.1} & 88905 \\  \hline
\textit{\multirow{2}{*}{bert-base-cased}} & S & 77.3 & \textbf{53.7} & \textbf{93.2} & \textbf{87.7/82.8} & 83.8/82.2 & \textbf{70.3/88.9} & \textbf{83.8/83.1} & \textbf{90.6} & 62.1 & 32.1 & 42722\\ 
 & M & \textbf{77.8} & 45.8 & 92.9 & 86.8/82.2 & \textbf{85.3/84.7} & 70.2/88.6 & 83.5/82.6 & 90.1 & \textbf{74.5} & \textbf{32.8} & 112613\\  \hline
\textit{\multirow{2}{*}{bert-large-cased}} & S & \textbf{79.5} & \textbf{59.2} & \textbf{94.9} & 85.0/80.6 & \textbf{85.8/84.5} & 70.5/89.1 & \textbf{86.7/85.6} & 92.2 & 70.1 & \textbf{39.4} & 37290 \\ 
 & M & \textbf{79.5} & 50.8 & 94.1 & \textbf{87.3/82.8} & 83.8/83.9 & \textbf{71.0/89.2} & 85.9/85.0 & \textbf{92.4} & \textbf{78.5} & 38.5 & 53343 \\  \hline
\end{tabular}
}
\end{table*}
}
\ifthenelse{\equal{#1}{rutransfer}}{
\begin{table*}
\caption{Точность/f1 macro на русскоязычных данных для многоязычных моделей. Режим S означает однозадачные модели, режим M -- многозадачные модели. RU означает русскоязычные данные, EN означает англоязычные данные. Объединенные означает, что русскоязычные и англоязычные данные объединены по задаче, Отдельные означает, что русскоязычные и англоязычными задачи считаются отдельными задачами. Усреднено по трем запускам. }
\label{mult_results}
%\begin{tabular}{|c|c|c||c|c|c|c|c|c|} \hline
\resizebox{\textwidth}{!}{
\begin{tabular}{|c|c|c||c|c|c|c|c|c||c|} \hline
Модель & \begin{tabular}[c]{@{}l@{}}Тренировочные\\данные\end{tabular} & Режим & Среднее & \begin{tabular}[c]{@{}l@{}}Эмоции\end{tabular} & \begin{tabular}[c]{@{}l@{}}Тональность\end{tabular} & \begin{tabular}[c]{@{}l@{}}Токсичность\end{tabular} & \begin{tabular}[c]{@{}l@{}}Интенты\end{tabular} & \begin{tabular}[c]{@{}l@{}}Темы\end{tabular} &\begin{tabular}[c]{@{}l@{}}Число\\батчей\end{tabular} \\
\hline \hline
\textit{distilbert-base-multilingual-cased} & RU & S & 84.7/81.0 & 77.4/69.1 & 77.7/77.9 & 96.7/94.8 & 83.5/76.6 & 88.1/86.9 & 10058 \\ %\hline
\textit{distilbert-base-multilingual-cased} & RU & M & 84.3/80.2 & 78.1/70.5 & 76.8/76.7 & 96.5/94.4 & 81.9/72.3 & 88.2/87.1 & 9821 \\ \hline
\textit{distilbert-base-multilingual-cased} & \begin{tabular}[c]{@{}l@{}}RU+EN,\\объединенные\end{tabular} & S & 85.2/81.8 & 78.9/70.2 & 77.4/77.3 & 96.8/94.9 & 84.7/79.1 & 88.4/87.4 & 31843 \\ %\hline
\textit{distilbert-base-multilingual-cased} & \begin{tabular}[c]{@{}l@{}}RU+EN,\\объединенные\end{tabular} & M & 84.5/81.1 & 77.9/70.7 & 76.6/76.7 & 96.5/94.5 & 82.9/76.5 & 88.4/87.2 & 17790 \\ \hline
\textit{distilbert-base-multilingual-cased} & \begin{tabular}[c]{@{}l@{}}RU+EN,\\отдельные\end{tabular} & M & 84.4/80.6 & 77.6/70.0 & 76.8/77.1 & 96.5/94.5 & 82.4/73.9 & 88.3/87.2 & 23688 \\ \hline
\textit{bert-base-multilingual-cased} & RU & S & 84.7/80.2 & 76.6/64.2 & 77.8/78.2 & 96.9/95.1 & 83.9/76.3 & 88.4/87.0 & 10884 \\ %\hline
\textit{bert-base-multilingual-cased} & RU & M & 84.8/81.4 & 78.4/71.4 & 76.3/76.3 & 96.8/94.8 & 83.7/76.6 & 89.0/87.8 & 12810 \\ \hline
\textit{bert-base-multilingual-cased} & \begin{tabular}[c]{@{}l@{}}RU+EN,\\объединенные\end{tabular} & S & 85.6/82.3 & 78.9/70.1 & 77.6/77.8 & 96.9/94.9 & 85.0/80.4 & 89.4/88.5 & 23752 \\ %\hline
\textit{bert-base-multilingual-cased} & \begin{tabular}[c]{@{}l@{}}RU+EN,\\объединенные\end{tabular} & M & 85.2/82.3 & 79.2/72.7 & 76.4/76.6 & 96.7/94.8 & 84.3/79.3 & 89.4/88.3 & 20755 \\ \hline
\textit{bert-base-multilingual-cased} & \begin{tabular}[c]{@{}l@{}}RU+EN,\\отдельные\end{tabular} & M & 85.0/81.6 & 78.3/71.4 & 77.1/77.0 & 96.7/94.7 & 84.0/76.7 & 89.1/88.0 & 22701 \\ \hline
\end{tabular}
}
\end{table*}
}
}


\newcommand{\mtldream}[1]{
    \ifthenelse{\equal{#1}{first}}{\begin{table}[htbp]
    \caption{Точность (взвешенный-F1) для многозадачной классификации для различных моделей. «1 в 1» означает оригинальные модели, «6 в 1» -- многозадачную модель с одним линейным слоем, обученную на аннотациях всех упомянутых в таблице классификаторов, «3 в 1 (cobot)» -- многозадачную модель с одним линейным слоем, обученную только на аннотациях классификаторов cobot topics, cobot dialogact topics и cobot dialogact intents, «3 в 1 (не cobot)» -- многозадачную модель с одним линейным слоем, обученную только на аннотациях остальных классификаторов(классификаторы эмоций, тональности и токсичности).}
    \label{mtldream:1}
    \centering
    \scalebox{0.65}{
    \begin{tabular}{|c|c|c|c|c|} 
    \hline
    \multirow{2}{*}{3адача} & \multicolumn{4}{c|}{Модели} \\
    \cline{2-5}
     & \textbf{1 в 1} & \textbf{6 в 1} & \textbf{3 в 1 (cobot)} & \textbf{3 в 1 (не cobot)}\\ 
    \hline
    cobot topics   & --- & 84~(83) & 82~(84) & --- \\
    \hline
    cobot dialogact topics  & --- & 76~(64) & 78~(66) & --- \\ 
    \hline
    cobot dialogact intents & --- & 69~(65) & 70~(67) & --- \\ 
    \hline
    Эмоции  & 92~(75) & 82~(60) & --- & 85~(67) \\
    \hline
    Тональность & 72~(68) & 60~(57) & --- & 66~(62) \\ 
    \hline
    Токсичность & 92~(60) & 92~(59) & --- & 93~(60)\\ 
    \hline
    \end{tabular}}
\end{table}}
\ifthenelse{\equal{#1}{final}}{\begin{table}[htbp]
\centering
\caption {Точность/взвешенный-F1) для оценки моделей в экспериментах с трансформер-инвариантными моделями. Для не-Коботовских задач при оценке используются оригинальные тестовые наборы данных, для коботовских -- тестовая часть разбиения данных. Как distilbert обозначается модель \textit{distilbert-base-uncased}, как bert модель \textit{bert-base-uncased}. «С историей» означает использование диалоговой истории только в задаче MIDAS, «Без истории» означает, что деалоговая история не использовалась ни в одной задаче «Размер» означает размер обучающей выборки. Режим S означает, что обучались однозадачные модели, M означает, что обучалась многозадачная модель. }
\label{tab:tr-ag-dream}% label всегда желательно идти после caption
\resizebox{\textwidth}{!}{
\begin{tabular}{|c||c|c|c|c|c|c|} \hline

Задача & Размер &\begin{tabular}[c]{@{}l@{}}distilbert, S\\с историей\end{tabular} & \begin{tabular}[c]{@{}l@{}}distilbert, M\\с историей\end{tabular}  & \begin{tabular}[c]{@{}l@{}}distilbert, M\\без истории\end{tabular} & \begin{tabular}[c]{@{}l@{}}bert, S\\с историей\end{tabular} & \begin{tabular}[c]{@{}l@{}}bert, M\\с историей\end{tabular}\\ \hline \hline
       
Эмоции              & 39.5k & 70.47/70.30 & 68.18/67.86 & 67.59/67.32         & 71.48/71.16 & 67.27/67.23 \\ \hline
Токсичность            & 1.62M & 94.53/93.64 & 93.84/93.5  & 93.86/93.41         & 94.54/93.15 & 93.94/93.4 \\ \hline
Тональность            & 94k  & 74.75/74.63 & 72.55/72.21 & 72.22/71.9          & 75.95/75.88 & 75.65/75.62 \\ \hline
Интенты MIDAS          & 7.1k & 80.53/79.81 & 72.73/71.56~ & 73.69/73.26 & 82.3/82.03  & 77.01/76.38 \\ \hline
DeepPavlov Topics & 1.8M & 87.48/87.43 & 86.98/86.9  & 87.01/87.05         & 88.09/88.1  & 87.43/87.47 \\ \hline
cobot topics ~                  & 216k & 79.88/79.9  & 77.31/77.36 & 77.45/77.35         & 80.68/80.67 & 78.21/78.22 \\ \hline
\begin{tabular}[c]{@{}l@{}}cobot dialogact\\ topics \end{tabular}            & 127k & 76.81/76.71 & 76.92/76.79 & 76.8/76.7          & 77.02/76.97 & 76.86/76.74 \\ \hline
\begin{tabular}[c]{@{}l@{}}cobot dialogact \\intents \end{tabular}           & 318k & 77.07/77.7  & 76.83/76.76 & 76.65/76.57         & 77.28/77.72 & 76.96/76.89 \\ \hline
Средее для 9 задач                   & 4.22M & 80.36/80.20    & 78.48/78.22 & 78.36/78.15         & 81.31/81.12  & 79.3/79.11 \\ \hline
\begin{tabular}[c]{@{}l@{}} Видеопамяти \\ использовано, Мб    \end{tabular}            &    & 2418*9=21762 & 2420     & 2420             & 3499*9=31491 & 3501    \\ \hline
\end{tabular}
}
\end{table}}
}
