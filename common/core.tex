
\newcommand{\pseudolabelA}{
\begin{table}[htbp]
\centering
\caption {Лучшая точность на тестовых данных (при лучшей скорости обучения из выбираемых, среднее по 3 запускам)}
\label{tab:ps2}% label всегда желательно идти после caption
\resizebox{\textwidth}{!}
{%
\begin{tabular}{|c||c|c|c|c|c|c|}
\hline
\multirow{2}{*}{Эксперимент} & \multicolumn{6}{c|}{Задача}  \\
\cline{2-7}
& Среднее& RTE& QQP& MNLI-m &MNLI-mm& SST-2\\
\hline
Базовый (оригинальная статья) & 78.8 & 66.4  & 71.2 & 84.6 & 83.4 & 93.5 \\
\hline
Базовый (воспроизведённый) & 77.6 & 62.7 & 71.0 & 83.1 & 82.7 & 93.5 \\
\hline
Независимые метки &79.0 & 71.5 & 70.9 & 82.7 & 81.7 & 91.3 \\
\hline
Мягкие независимые метки  &78.9 & 69.3 & 71.3 & 82.8 & 82.1 & 92.6 \\
\hline
Дополненные независимые метки &77.6 & 64.2&\textbf{ 71.8} & 81.2 & 80.7 & \textbf{93.2} \\
\hline
Мягкое вероятностное предположение  &\textbf{79.7} & \textbf{72.7} & 70.7 &\textbf{ 83.4} &\textbf{82.3} & 92.5 \\
\hline
Мягкие предсказанные метки  & 78.8 & 70.3 & 70.7 & 81.7 & 81.7 & 92.5 \\
\hline
Жесткие предсказанные метки & 79.1 & 71.3 &71.1 & 81.7 & 81.4 & 92.6 \\
\hline
\begin{tabular}[c]{@{}l@{}}Независимые метки,\\замороженная голова\end{tabular}   & 78.2 & 66.9& \textbf{71.8} & 82.6 & 81.8 & 91.9 \\
\hline
\begin{tabular}[c]{@{}l@{}}Мягкие независимые метки,\\замороженная голова\end{tabular}  &79.1 & 70.0 & 71.5 & 83.0 &\textbf{ 82.3} & 92.4 \\
\hline
\end{tabular}
}
\end{table}}

\newcommand{\enmtlA}{
\begin{table*}
 \caption{Метрики англоязычных моделей (точность/f1 macro) для пяти англоязычных диалоговых задач.Режим S означает однозадачные модели, режим M означает многозадачные модели. Усреднено по трем запускам.}
 \label{tab:tr-ag:en_results}
\centering
\resizebox{\textwidth}{!}{%
\begin{tabular}{|c|c|c|c|c|c|c|c|c|}
\hline
\multirow{2}{*}{Модель} & \multirow{2}{*}{Режим} & \multirow{2}{*}{Среднее} & Эмоции & Тональность & Токсичность & Интенты & Темы & Число \\
& & & 39.4k & 80.5k & 127.6k & 11.5k & 11.5k & батчей \\ \hline \hline
\textit{\multirow{2}{*}{distilbert-base-cased}} & S & \textbf{82.9/78.4} & \textbf{70.3/63.1} & 74.7/74.3 & 91.5/81.2 & \textbf{87.4/82.7} & \textbf{91.0/90.6} & 11390 \\ 
 & M  & 82.1/77.2 & 67.7/60.7 & \textbf{75.2/75.0} & 90.6/79.8 & 86.3/80.4 & 90.8/90.1 & 14000 \\ \hline
\textit{\multirow{2}{*}{bert-base-cased}} & S & \textbf{83.9/79.7} & \textbf{71.2/64.2} & 76.1/75.8 & \textbf{93.2/83.5} & \textbf{87.9/84.2} & \textbf{91.3/90.7} & 9470 \\
 & M &  83.0/78.4 & 69.0/63.1 & \textbf{76.5/76.4} & 91.4/80.8 & 87.1/81.2 & 91.2/90.6 & 11760 \\ \hline
\textit{\multirow{2}{*}{bert-large-cased}} & S &  \textbf{84.7/80.5} & \textbf{70.9/64.4} & \textbf{80.5/80.4} & \textbf{92.1/82.2} & \textbf{88.4/84.9} & 91.3/90.7 & 8526 \\ 
 & M  & 83.6/78.7 & 69.0/61.8 & 79.0/78.9 & 91.3/80.9 & 87.3/80.9 & \textbf{91.3/90.8} & 11200 \\ \hline
 \end{tabular}
 }
\end{table*}}

\newcommand{\rumtlA}{ 
\begin{table*}
 \caption{Метрики русскоязычных моделей(точность/f1 macro) для пяти диалоговых задач. Режим S означает однозадачные модели, режим M означает многозадачные модели. Усреднено по трем запускам.}
 \label{tab:tr-ag:ru_results}
\centering
\resizebox{\textwidth}{!}{%
\begin{tabular}{|c|c|c|c|c|c|c|c|c|}
\hline
\multirow{2}{*}{Модель} & \multirow{2}{*}{Режим} & \multirow{2}{*}{Среднее} & Эмоции & Тональность & Токсичность & Интенты & Темы & Число \\
& & & 6.5k & 82.6k & 93.3k & 11.5k & 11.5k & батчей \\ \hline \hline
\textit{\multirow{2}{*}{DeepPavlov/distilrubert-base-cased-conversational}} & S & 86.9/84.1 & 82.2/76.1 & 77.9/78.2 & 97.1/95.4 & 86.7/81.6 & 90.4/89.5 & 8472 \\
 & M & 86.3/82.6 & 81.0/74.6 & 77.7/77.7 & 96.9/95.0 & 85.2/75.9 & 90.7/89.9 & 8540 \\ \hline
\textit{\multirow{2}{*}{DeepPavlov/rubert-base-cased-conversational}} & S & 86.5/83.4 & 80.9/75.3 & 78.0/78.2 & 97.2/95.6 & 86.2/79.1 & 90.0/89.0 & 7999 \\ 
 & M & 86.2/82.6 & 80.5/73.8 & 77.6/77.6 & 96.8/95.0 & 85.3/76.9 & 90.5/89.8 & 8113 \\ \hline
 \end{tabular}
 }
 \end{table*}}

\newcommand{\enmtlB}{ 
\begin{table*}
\caption{Метрики многозадачной трансформер-инвариантной модели для набора задач GLUE. M.Corr означает корреляцию Мэттью, P/S означает корреляцию Пирсона-Спирмена, Acc точность, F1 -- f1 метрику. Режим S означает однозадачные модели, режим M означает многозадачные модели. Размер означает размер тренировочного набора данных}
\label{tab:tr-ag:mtl_glue}
\centering
\resizebox{\textwidth}{!}{%
\begin{tabular}{|c|c|c|c|c|c|c|c|c|c|c|c|c|}
\hline
\multirow{3}{*}{Модель} & \multirow{3}{*}{Режим}  & Среднее & CoLA & SST-2 & MRPC &STS-B &QQP&MNLI & QNLI & RTE & AX & Число \\
\cline{3-11}
   &   Размер & & 8.6k & 67.3k & 2.5k & 5.7k & 363.8k & 392.7k & 104.7k & 2.5k & как у MNLI & батчей \\ 
\cline{3-11}   
   &  Метрика & & M.Corr & Acc & F1/Acc & P/S Corr & F1/Acc & Acc(m/mm) & Acc & Acc & M.Corr &  \\ \hline \hline
Человек & -- & 87.1 & 66.4 & 97.8 & 86.3/80.8 & 92.7/92.6 & 59.5/80.4 & 92.0/92.8 & 91.2 & 93.6 & - & -\\ \hline
%\textit{\multirow{2}{*}{distilbert-base-cased}} & S & 73.1 & \textbf{42.4} & \textbf{92.1} & 85.6/\textbf{80.3} & 78.8/76.8 & \textbf{69.5/88.5} & \textbf{81.3/80.8} & \textbf{87.5} & 49.8 & 29.9 & 70846 \\ 
\textit{\multirow{2}{*}{distilbert-base-cased}} & S & 73.3 & \textbf{42.4} & \textbf{92.1} & 85.6/\textbf{80.3} & 78.8/76.8 & \textbf{69.5/88.5} & \textbf{81.3/80.8} & \textbf{87.5} & 52.1 & 29.9 & 70861 \\ 
 & M & \textbf{74.5} & 36.0 & 91.0 & \textbf{85.7}/79.9 & \textbf{82.6/81.6} & 68.4/87.4 & 80.4/80.3 & 86.0 & \textbf{69.5} & \textbf{30.1} & 88905 \\  \hline
\textit{\multirow{2}{*}{bert-base-cased}} & S & 77.3 & \textbf{53.7} & \textbf{93.2} & \textbf{87.7/82.8} & 83.8/82.2 & \textbf{70.3/88.9} & \textbf{83.8/83.1} & \textbf{90.6} & 62.1 & 32.1 & 42722\\ 
 & M & \textbf{77.8} & 45.8 & 92.9 & 86.8/82.2 & \textbf{85.3/84.7} & 70.2/88.6 & 83.5/82.6 & 90.1 & \textbf{74.5} & \textbf{32.8} & 112613\\  \hline
\textit{\multirow{2}{*}{bert-large-cased}} & S & \textbf{79.5} & \textbf{59.2} & \textbf{94.9} & 85.0/80.6 & \textbf{85.8/84.5} & 70.5/89.1 & \textbf{86.7/85.6} & 92.2 & 70.1 & \textbf{39.4} & 37290 \\ 
 & M & \textbf{79.5} & 50.8 & 94.1 & \textbf{87.3/82.8} & 83.8/83.9 & \textbf{71.0/89.2} & 85.9/85.0 & \textbf{92.4} & \textbf{78.5} & 38.5 & 53343 \\  \hline
\end{tabular}
}
\end{table*}}


\newcommand{\enmtlC}{
\begin{table*}
\caption{Точность/ F1 для запусков на части тренировочных данных.  Режим M означает многозадачные модели, режим S означает однозадачные модели, и Доля означает долю использованных тренировочных данных. Базовая модель \textit{distilbert-base-cased}. Усреднено по пяти запускам. }
\label{tab:tr-ag:en_dialog_part}
\resizebox{\textwidth}{!}{%
\begin{tabular}{|c|c|c|c|c|c|c|c|c|}
\hline
 \multirow{2}{*}{Режим} &  \multirow{2}{*}{Доля} &  \multirow{2}{*}{Среднее} & Эмоции & Тональность & Токсичность & Интенты & Темы & Число \\
& & & 39.4k & 80.5k & 127.6k & 11.5k & 11.5k & батчей \\
\hline \hline
S & 15\% & 78.5/70.9 & 65.8/50.6 & 69.3/68.8 & 92.2/81.2 & 78.7/68.8 & 86.3/85.1 & 2173 \\
M & 15\% & 77.4/70.6 & 64.0/55.0 & 68.3/67.7 & 91.6/80.0 & 76.9/64.6 & 86.4/85.4 & 4741 \\ \hline
S & 10\% & 77.1/68.4 & 64.6/45.0 & 68.3/67.8 & 92.2/81.0 & 75.5/64.7 & 84.8/83.3 & 1579 \\
M & 10\% & 75.9/69.1 & 62.6/53.6 & 66.6/65.8 & 91.5/79.7 & 74.3/63.2 & 84.6/83.3 & 4295 \\ \hline
S & 9\% & 76.7/67.4 & 64.6/43.9 & 68.2/67.7 & 91.8/80.4 & 74.4/62.7 & 84.2/82.4 & 1457 \\
M & 9\% & 75.4/67.8 & 62.1/52.4 & 66.5/65.7 & 91.4/79.5 & 72.4/58.5 & 84.4/83.0 & 3695 \\ \hline
%S & 8\% & 73.8/64.6 & 63.7/43.5 & 67.6/67.0 & 92.0/80.5 & 62.0/49.7 & 83.9/82.1 & 1381 \\
%M & 8\% & 74.9/67.2 & 61.7/51.4 & 66.7/66.0 & 91.5/79.5 & 71.1/57.3 & 83.6/81.7 & 3511 \\ \hline
%S & 7.5\% & 73.5/64.0 & 63.8/42.5 & 67.4/66.9 & 91.6/80.1 & 61.3/48.6 & 83.6/81.7 & 1293 \\
%M & 7.5\% & 74.6/66.9 & 61.0/50.2 & 67.0/66.4 & 91.5/79.5 & 70.3/56.9 & 83.2/81.4 & 2995 \\ \hline
S & 7\% & 73.5/64.0 & 63.3/42.1 & 67.9/67.4 & 91.8/80.1 & 61.4/49.4 & 83.3/81.1 & 1251 \\
M & 7\% & 74.2/66.4 & 61.1/50.4 & 65.8/65.1 & 91.0/78.9 & 70.0/56.3 & 83.1/81.3 & 2882 \\ \hline
S & 5\% & 69.1/59.0 & 62.5/38.9 & 66.9/66.3 & 91.8/79.9 & 42.7/30.8 & 81.6/78.8 & 901 \\
M & 5\% & 71.7/62.4 & 60.5/48.6 & 64.4/63.4 & 90.8/78.5 & 62.4/44.4 & 80.2/77.3 & 2381 \\ \hline
S & 3\% & 59.6/49.0 & 60.6/37.7 & 65.2/64.5 & 91.8/79.3 & 26.5/16.9 & 54.0/46.5 & 584 \\
M & 3\% & 68.8/58.1 & 58.6/42.7 & 62.5/61.3 & 91.0/78.3 & 55.5/37.1 & 76.4/71.0 & 1566 \\ \hline
S & 2\% & 44.9/31.5 & 48.7/21.7 & 39.6/26.5 & 91.8/79.0 & 2.6/0.2 & 41.4/30.1 & 274 \\
M & 2\% & 64.8/52.1 & 57.6/38.4 & 61.4/60.1 & 90.8/78.0 & 44.2/23.5 & 69.9/60.4 & 923 \\ \hline
\end{tabular}
}
\end{table*}}


\newcommand{\rumtlB}{
\label{fig:thresholds_acc_ru}
\begin{figure}[!htbp]
\begin{minipage}{0.55\textwidth}
\begin{tikzpicture}[baseline={(0,2.1)}]%[scale=2]
\begin{axis}[xlabel = RU доля,
ylabel = Средняя точность,
legend pos= south east,
width=0.9\textwidth,
xtick={0,1,2,3,4,5,6,7,8},
xticklabels={2,3,5,10,15,20,25,50,100},
ymin=50,ymax=90,
legend cell align={left},
legend style={nodes={scale=0.5, transform shape}}
]
\addplot[color=orange,dotted, mark=*] coordinates {
(1, 57.02)
(2, 58.379999999999995)
(3, 75.66666666666667)
(4, 77.7)
(5, 78.36666666666667)
(6, 79.53333333333333)
(7, 82.5)
(8, 84.39999999999999)
};
\addlegendentry{S точность (RU)}
\addplot[color=red,solid,mark=*] coordinates {
(1, 65.88)
(2, 70.3)
(3, 75.23333333333333)
(4, 77.23333333333333)
(5, 79.0)
(6, 79.60000000000001)
(7, 82.3)
(8, 84.33333333333333)
};
\addlegendentry{M точность (RU)}
\addplot[color=cyan,dashed, mark=*] coordinates {
(1, 70.73333333333333)
(2, 74.23333333333333)
(3, 77.39999999999999)
(4, 78.89999999999999)
(5, 80.13333333333334)
(6, 80.93333333333332)
(7, 82.83333333333333)
(8, 84.36666666666667)
};
\addlegendentry{S точность (RU+EN)}
\addplot[color=blue,dashed,mark=*] coordinates {
(1, 71.8)
(2, 74.96666666666667)
(3, 77.93333333333334)
(4, 79.66666666666667)
(5, 80.56666666666666)
(6, 81.36666666666666)
(7, 83.23333333333333)
(8, 85.16666666666667)
};
\addlegendentry{M точность (RU+EN)}
\end{axis}%
\end{tikzpicture}
% \hspace{3mm}
% \captionof{figure}
% \caption{Av}
\end{minipage}
\begin{minipage}{0.45\textwidth}
{
\scalebox{0.8}{
\begin{tabular}[baseline={(0,2.1)}]{|l||c|c|c|c|}
\hline
RU & S & M & S & M \\
доля & RU & RU & RU+EN & RU+EN \\ % Sizes are different for RU and RU+EN, so I don't give them here
\hline \hline
3 & 71.8 & 70.7 & 57.0 & 65.9 \\ \hline
5 & 75.0 & 74.2 & 58.4 & 70.3 \\ \hline
10 & 77.9 & 77.4 & 75.7 & 75.2 \\ \hline
15 & 79.7 & 78.9 & 77.7 & 77.2 \\ \hline
20 & 80.6 & 80.1 & 78.4 & 79.0 \\ \hline
25 & 81.4 & 80.9 & 79.5 & 79.6 \\ \hline
50 & 83.2 & 82.8 & 82.5 & 82.3 \\ \hline
100 & 85.2 & 84.4 & 84.4 & 84.3 \\ \hline
\end{tabular}}
}
\end{minipage}
\caption{Средняя точность на русскоязычных данных для \textit{distilbert-base-multilingual-cased}. S означает однозадачный режим, M означает многозадачный режим, RU доля означает долю русскоязычных обучающих данных, использованных при обучении, RU означает только обучение на этой доле русскоязычных данных, RU+EN означает обучение на этой доле русскоязычных обучающих данных плюс на 100 процентах англоязычных обучающих данных.}
\label{fig:tr-ag:ru_dialog_part}
\end{figure}}

\newcommand{rutopicsA}{
\begin{table*}
\centering
\scalebox{0.7}{
\begin{tabular}{|c||c|c|c|c|c|} \hline
\textbf{тип данных}  & \multicolumn{2}{c|}{\textbf{однометочные}} & \multicolumn{2}{c|}{\textbf{многометочные}} & \multirow{2}{*}{\textbf{равноразмерные}}\\
\cline{1-5}
\textbf{класс}  & \multicolumn{1}{c|}{все} & \multicolumn{1}{c|}{отвеченные} & \multicolumn{1}{c|}{все} & \multicolumn{1}{c|}{отвеченные} & \\\hline \hline
\textit{Размер набора данных} & 361650 & 266597 & 170930 & 137341 & 264786\\ \hline
\textit{Размер 6классового поднабора данных} & 18864 & 15912 & 27191 & 20569 & 15830 \\ \hline
\textit{Музыка} & 9514 & 5809 & 4456 & 3287 & 5797 \\ \hline
\textit{Еда, напитки и кулинария} & 5750 & 4758 & 14096 & 11084 & 4723 \\ \hline
\textit{Медиа и коммуникации} & 4505 & 2637 & 5577 & 3948 & 2619 \\ \hline
\textit{Транспорт} & 2435 & 1625 & 1933 & 1387 & 1613 \\ \hline
\textit{Новости} & 945 & 602 & 912 & 720 & 600\\ \hline
\textit{Погода} & 890 & 481 & 217 & 143 & 478 \\ \hline
\end{tabular}
}
\caption{Размеры набора данных {RuQTopics} по классу и части}
%\centering
\label{tab:RuQTopics:sizes}
\end{table*}}

\newcommand{rutopicsB}{
\begin{table*}[!htbp]
\caption{Метрики модели \textit{bert-base-multilingual-cased} на объединенном тестовом наборе данных {MASSIVE} для всех языков. Модель обучалась на версии \textbf{Q} набора данных {RuQTopics}. \textbf{Код} означает код языка(ISO 639-1), \textbf{N} означает число статей в Википедии на этом языке на 11 октября 2018 года, \textbf{Дист} означает лингвистическую дистанцию между этим языком и русским, посчитанную в соответствии с работой~\cite{lang_sim}. Усреднено по трем запускам.}
\label{tab:rutopics:crosslingual}
% \centering
% \scalebox{0.7}{
\begin{minipage}{0.5\textwidth}
\scalebox{0.55}{
\begin{tabular}[baseline={(0,2.1)}]{|c|c|c|c|c|c|} \hline
\multirow{2}{*}{\textbf{Язык}}  & \multirow{2}{*}{\textbf{Код}} & \multirow{2}{*}{\textbf{Дист}} & \multirow{2}{*}{\textbf{N}}  &  \multicolumn{2}{c|}{\textbf{Метрики}} \\ %\hline
\cline{5-6}
& & & & Точность & Макро-F1 \\ \hline \hline
русский & ru & 0 & 1,501,878 & 80.8 & 79.8\\
китайский (Тайвань) & zh-TW & 92.2 & 1,025,366 & 79.6 & 79.1\\
китайский & zh & 92.2 & 1,025,366 & 78.0 & 77.7\\
английский & en & 60.3 & 5,731,625 & 75.2 & 75.6\\
японский & ja & 93.3 & 1,124,097 & 72.4 & 70.5\\
словенский & sl & 4.2 & 162,453 & 70.3 & 69.0\\
шведский & sv & 59.5 & 3,763,579 & 70.2 & 69.6\\
малайский & ms & n/c & 320,631 & 68.9 & 67.7\\
итальянский & it & 45.8 & 1,466,064 & 68.8 & 68.0\\
индонезийский & id & 91.2 & 440,952 & 68.7 & 67.5\\
нидерландский & nl & 64.6 & 1,944,129 & 68.7 & 68.5\\
португальский & pt & 61.6 & 1,007,323 & 68.6 & 68.7\\
испанский & es & 51.7 & 1,480,965 & 68.2 & 68.0\\
датский & da & 66.2 & 240,436 & 67.8 & 66.7\\
французский & fr & 61.0 & 2,046,793 & 65.5 & 65.5\\
персидский & fa & 72.4 & 643,750 & 65.2 & 64.2\\
турецкий & tr & 86.2 & 316,969 & 64.5 & 62.4\\
вьетнамский & vi & 95.0 & 1,190,187 & 64.3 & 65.1\\
норвежский букмол & nb & 67.2 & 495,395 & 64.3 & 64.0\\
польский & pl & 5.1 & 1,303,297 & 64.2 & 62.2\\
азербайджанский & az & 87.7 & 138,538 & 63.9 & 63.1\\
каталанский & ca & 60.3 & 591,783 & 61.4 & 60.4\\
венгерский & hu & 87.2 & 437,984 & 61.3 & 60.0\\
иврит & he & 88.9 & 231,868 & 60.9 & 59.5\\
хинди & hi & 69.8 & 127,044 & 60.7 & 58.7\\
корейский & ko & 89.5 & 429,369 & 60.4 & 59.6\\
\hline
\end{tabular}
}
\end{minipage}
\begin{minipage}{0.5\textwidth}
\scalebox{0.55}{
\begin{tabular}[baseline={(0,2.1)}]{|c|c|c|c|c|c|} \hline
\multirow{2}{*}{\textbf{Язык}}  & \multirow{2}{*}{\textbf{Код}} & \multirow{2}{*}{\textbf{Дист}} & \multirow{2}{*}{\textbf{N}}  &  \multicolumn{2}{c|}{\textbf{Метрики}} \\ %\hline
\cline{5-6}
& & & & Точность & Макро-F1 \\ \hline \hline
румынский & ro & 55.0 & 388,896 & 57.1 & 53.9\\
урду & ur & 66.7 & 140,939 & 56.4 & 55.9\\
арабский & ar & 86.5 & 619,692 & 56.2 & 55.7\\
каннада & kn & 90.8 & 23,844 & 56.1 & 53.0\\
филиппинский & tl & 91.9 & 80,992 & 55.0 & 51.3\\
телугу & te & 96.7 & 69,354 & 53.7 & 49.3\\
финский & fi & 88.9 & 445,606 & 53.3 & 51.3\\
бирманский & my & 86.0 & 39,823 & 52.5 & 49.7\\
африкаанс & af & 64.8 & 62,963 & 52.4 & 50.3\\
тамильский & ta & 94.7 & 118,119 & 52.4 & 50.1\\
немецкий & de & 64.5 & 2,227,483 & 52.2 & 51.6\\
албанский & sq & 69.4 & 74,871 & 51.5 & 47.2\\
латышский & lv & 49.1 & 88,189 & 49.6 & 48.4\\
малаялам & ml & 96.7 & 59,305 & 48.7 & 46.3\\
армянский & hy & 77.8 & 246,571 & 48.1 & 47.5\\
бенгальский & bn & 66.3 & 61,294 & 47.3 & 45.3\\
тайский & th & 89.5 & 127,010 & 46.5 & 44.9\\
греческий & el & 75.3 & 153,855 & 46.3 & 44.8\\
грузинский & ka & 96.0 & 124,694 & 39.2 & 38.1\\
яванский & jv & 95.4 & 54,964 & 38.7 & 37.1\\
монгольский & mn & 86.2 & 18,353 & 36.6 & 33.7\\
исландский & is & 68.9 & 45,873 & 32.6 & 29.9\\
суахили & sw & 95.1 & 45,806 & 31.0 & 28.0\\
валлийский & cy & 75.5 & 101,472 & 28.5 & 25.3\\
кхмерский & km & 97.1 & 6,741 & 16.1 & 8.6\\
амхарский & am & 86.6 & 14,375 & 12.1 & 5.0\\ \hline
\hline
\end{tabular}
}
\end{minipage}
\end{table*}
}


\newcommand{\mtldreamA}{
\begin{table}[htbp]
    \caption{Точность (взвешенный-F1) для многозадачной классификации для различных моделей. «1 в 1» означает оригинальные модели, «6 в 1» -- многозадачную модель с одним линейным слоем, обученную на аннотациях всех упомянутых в таблице классификаторов, «3 в 1 (cobot)» -- многозадачную модель с одним линейным слоем, обученную только на аннотациях классификаторов cobot topics, cobot dialogact topics и cobot dialogact intents, «3 в 1 (не cobot)» -- многозадачную модель с одним линейным слоем, обученную только на аннотациях остальных классификаторов(классификаторы эмоций, тональности и токсичности).}
    \label{mtldream:1}
    \centering
    \scalebox{0.65}{
    \begin{tabular}{|c|c|c|c|c|} 
    \hline
    \multirow{2}{*}{3адача} & \multicolumn{4}{c|}{Модели} \\
    \cline{2-5}
     & \textbf{1 в 1} & \textbf{6 в 1} & \textbf{3 в 1 (cobot)} & \textbf{3 в 1 (не cobot)}\\ 
    \hline
    cobot topics   & --- & 84~(83) & 82~(84) & --- \\
    \hline
    cobot dialogact topics  & --- & 76~(64) & 78~(66) & --- \\ 
    \hline
    cobot dialogact intents & --- & 69~(65) & 70~(67) & --- \\ 
    \hline
    Эмоции  & 92~(75) & 82~(60) & --- & 85~(67) \\
    \hline
    Тональность & 72~(68) & 60~(57) & --- & 66~(62) \\ 
    \hline
    Токсичность & 92~(60) & 92~(59) & --- & 93~(60)\\ 
    \hline
    \end{tabular}}
\end{table}}

\newcommand{\mtldreamB}{
\begin{table}[htbp]
\centering
\caption {Точность (взвешенный-F1) для моделей \underline{без диалоговой истории} для многозадачной модели с 1 линейным слоем и PAL-BERT на псевдоразмеченных данных из Alexa Prize Challenge 4, оценка на «чистых» тестовых данных для не-коботовских задач и на псевдоразмеченных для коботовских задач. «1 в 1» означает оригинальные модели.}
\label{mtldream:4}
\resizebox{\textwidth}{!}{%
\begin{tabular}{|c||c|c|c|c|} \hline
\multirow{2}{*}{3адача} & \multicolumn{4}{c|}{Модели} \\
\cline{2-5}
 & 7 в 1 & \begin{tabular}[c]{@{}l@{}}7 в 1\\ жесткие метки\end{tabular} & \begin{tabular}[c]{@{}l@{}}7 в 1\\ PAL-BERT \\ жесткие метки\end{tabular} & \begin{tabular}[c]{@{}l@{}}1 в 1\\ \end{tabular} \\
\hline
\hline
cobot topics & \textbf{81.9(80)} & 80.2(78.1) & 81.8(79.5) & 1(1) \\
\hline
cobot dialogact topics & 80.5(62.6) & 79.9(61.6) & \textbf{81.4(63.2)} & 1(1) \\
\hline
cobot dialogact intents & 75.2(63.5) & 74.5(62.6) & \textbf{76.7(63)} & 1(1) \\
\hline
Эмоции & 40.1(24.5) & 72(64.1) & \textbf{78.8(75.4)} & 92(75.1) \\
\hline
Тональность & 68.3(60.7) & 72.7(60.9) & \textbf{73.3(58.5)} & 72.1(68.1) \\
\hline
Токсичность & 93.2(194) & 93.1(18) & \textbf{93.5(18.6)} & 92.2(59.6) \\
\hline
Фактоидность & 80.5(80.6) & 81.6(81.4) & \textbf{82.9(83.1)} & 88.6(88.4) \\
\hline
\end{tabular}
}
\end{table}}

\newcommand{\mtldreamC}{
\begin{table}[htbp]
\centering
\caption {Точность (взвешенный-F1) для оценки моделей в третьей серии экспериментов. Для не-Коботовских задач при оценке используются оригинальные тестовые наборы данных, для коботовских -- тестовая часть разбиения данных. «1 в 1» означает оригинальные модели, «История» означает использование диалоговой истории для Коботовских задач.}
\label{mtldream:5}
\resizebox{\textwidth}{!}{%
\begin{tabular}{|c||c|c|c|c|c|c|c|} \hline
\multirow{1}{*}{} & \multicolumn{7}{c|}{Модель} \\ 
\cline{2-8}
         &7 в 1 & 7 в 1 & PAL-BERT & PAL-BERT & PAL-BERT & PAL-BERT & 1 в 1 \\
 История & нет & есть & есть & есть  & есть & есть & есть \\ \hline
 Псевдоразметка & \multirow{2}{*}{полная} & \multirow{2}{*}{полная} & \multirow{2}{*}{нет} & \multirow{2}{*}{\begin{tabular}[c]{@{}l@{}}только \\не-Коботовские \end{tabular}} & \multirow{2}{*}{полная} & \multirow{2}{*}{\begin{tabular}[c]{@{}l@{}}тональность и \\фактоидность\end{tabular}} & \multirow{2}{*}{нет} \\ 
 \cline{1-1}
Задача & & & & & & & \\
\hline
\hline
cobot topics & \textbf{70.1(66.6)} & 56.8(53.3) & 83.3(81) & 83.1(80.8) & \textbf{86.3(84.3)} & 82.8(81) & 1(1) \\
\hline
cobot dialogact topics & 75.6(51.9) & 85.2(66.7) & \textbf{87.1(70.4)} & 86.9(70.4) & \textbf{90.6(80.4)} & 86.8(69.8) & 1(1) \\
\hline
cobot dialogact intents & 51.5(40.6) & 72.8(51.6) & \textbf{76.8(56.3)} & 76.5(56.1) & \textbf{82.8(68.5)} & 75.3(55.4) & 1(1) \\
\hline
Эмоции & 90.5(88) & 91.7(88.3) & \textbf{92.7(90.6)} & 92.4(89.3) & 92.3(89.7) & 92.6(91) & 92(75.1) \\
\hline
Тональность & 72(63.3) & 71.3(65.7) & \textbf{70.6(64.8)} & 72.7(65.9) & 71.3(64.7) & \textbf{75.4(66.4)} & 72.1(68.1) \\
\hline
Токсичность & 93.8(19.9) & 93.2(21) & \textbf{92.8(25.3)} & 93.2(29.8) & 93.2(26.9) & \textbf{93.9(25.9)} & 92.2(59.6) \\
\hline
Фактоидность & 78.9(80.9) & 79.4(81.7) & \textbf{83.4(83.1)} & 84.6(84.4) & \textbf{86.9(86.6)} & 85.4(85.3) & 88.6(88.4) \\
\hline
\end{tabular}
}
\end{table}}

\newcommand{\mtldreamD}{
\begin{table}[htbp]
\centering
\caption {Точность/взвешенный-F1) для оценки моделей в экспериментах с трансформер-инвариантными моделями. Для не-Коботовских задач при оценке используются оригинальные тестовые наборы данных, для коботовских -- тестовая часть разбиения данных. Как distilbert обозначается модель \textit{distilbert-base-uncased}, как bert модель \textit{bert-base-uncased}. «С историей» означает использование диалоговой истории только в задаче MIDAS, «Без истории» означает, что деалоговая история не использовалась ни в одной задаче «Размер» означает размер обучающей выборки. Режим S означает, что обучались однозадачные модели, M означает, что обучалась многозадачная модель. }
\label{tab:tr-ag-dream}% label всегда желательно идти после caption
\resizebox{\textwidth}{!}{
\begin{tabular}{|c||c|c|c|c|c|c|} \hline

Задача & Размер &\begin{tabular}[c]{@{}l@{}}distilbert, S\\с историей\end{tabular} & \begin{tabular}[c]{@{}l@{}}distilbert, M\\с историей\end{tabular}  & \begin{tabular}[c]{@{}l@{}}distilbert, M\\без истории\end{tabular} & \begin{tabular}[c]{@{}l@{}}bert, S\\с историей\end{tabular} & \begin{tabular}[c]{@{}l@{}}bert, M\\с историей\end{tabular}\\ \hline \hline
       
Эмоции              & 39.5k & 70.47/70.30 & 68.18/67.86 & 67.59/67.32         & 71.48/71.16 & 67.27/67.23 \\ \hline
Токсичность            & 1.62M & 94.53/93.64 & 93.84/93.5  & 93.86/93.41         & 94.54/93.15 & 93.94/93.4 \\ \hline
Тональность            & 94k  & 74.75/74.63 & 72.55/72.21 & 72.22/71.9          & 75.95/75.88 & 75.65/75.62 \\ \hline
Интенты MIDAS          & 7.1k & 80.53/79.81 & 72.73/71.56~ & 73.69/73.26 & 82.3/82.03  & 77.01/76.38 \\ \hline
DeepPavlov Topics & 1.8M & 87.48/87.43 & 86.98/86.9  & 87.01/87.05         & 88.09/88.1  & 87.43/87.47 \\ \hline
cobot topics ~                  & 216k & 79.88/79.9  & 77.31/77.36 & 77.45/77.35         & 80.68/80.67 & 78.21/78.22 \\ \hline
\begin{tabular}[c]{@{}l@{}}cobot dialogact\\ topics \end{tabular}            & 127k & 76.81/76.71 & 76.92/76.79 & 76.8/76.7          & 77.02/76.97 & 76.86/76.74 \\ \hline
\begin{tabular}[c]{@{}l@{}}cobot dialogact \\intents \end{tabular}           & 318k & 77.07/77.7  & 76.83/76.76 & 76.65/76.57         & 77.28/77.72 & 76.96/76.89 \\ \hline
Средее для 9 задач                   & 4.22M & 80.36/80.20    & 78.48/78.22 & 78.36/78.15         & 81.31/81.12  & 79.3/79.11 \\ \hline
\begin{tabular}[c]{@{}l@{}} Видеопамяти \\ использовано, Мб    \end{tabular}            &    & 2418*9=21762 & 2420     & 2420             & 3499*9=31491 & 3501    \\ \hline
\end{tabular}
}
\end{table}}
