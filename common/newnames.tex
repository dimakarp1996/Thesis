
% Новые переменные, которые могут использоваться во всём проекте
% ГОСТ 7.0.11-2011
% 9.2 Оформление текста автореферата диссертации
% 9.2.1 Общая характеристика работы включает в себя следующие основные структурные
% элементы:
% актуальность темы исследования;
\newcommand{\actualityTXT}{Актуальность темы.}
% степень ее разработанности;
\newcommand{\progressTXT}{Степень разработанности темы.}
% цели и задачи;
\newcommand{\aimTXT}{Целью}
\newcommand{\tasksTXT}{задачи}
% научную новизну;
\newcommand{\noveltyTXT}{Научная новизна:}
% теоретическую и практическую значимость работы;
%\newcommand{\influenceTXT}{Теоретическая и практическая значимость}
\newcommand {\appropriationTXT}{Соответствие диссертации паспорту научной специальности}
% или чаще используют просто
\newcommand{\influenceTXT}{Практическая значимость}
% методологию и методы исследования;
\newcommand{\methodsTXT}{Методология и методы исследования.}
% положения, выносимые на защиту;
\newcommand{\defpositionsTXT}{Основные положения, выносимые на~защиту:}
% степень достоверности и апробацию результатов.
\newcommand{\reliabilityTXT}{Достоверность}
\newcommand{\probationTXT}{Апробация работы.}

\newcommand{\contributionTXT}{Личный вклад.}
\newcommand{\publicationsTXT}{Публикации.}


%%% Заголовки библиографии:

% для автореферата:
\newcommand{\bibtitleauthor}{Публикации автора по теме диссертации}

% для стиля библиографии `\insertbiblioauthorgrouped`
\newcommand{\bibtitleauthorvak}{В изданиях из списка ВАК РФ}
\newcommand{\bibtitleauthorscopus}{В изданиях, входящих в международную базу цитирования Scopus}
\newcommand{\bibtitleauthorwos}{В изданиях, входящих в международную базу цитирования Web of Science}
\newcommand{\bibtitleauthorother}{В прочих изданиях}
\newcommand{\bibtitleauthorconf}{В сборниках трудов конференций}
\newcommand{\bibtitleauthorpatent}{Зарегистрированные патенты}
\newcommand{\bibtitleauthorprogram}{Зарегистрированные программы для ЭВМ}

% для стиля библиографии `\insertbiblioauthorimportant`:
\newcommand{\bibtitleauthorimportant}{Наиболее значимые \protect\MakeLowercase\bibtitleauthor}

% для списка литературы в диссертации и списка чужих работ в автореферате:
\newcommand{\bibtitlefull}{Список литературы} % (ГОСТ Р 7.0.11-2011, 4)



\newcommand{\dream_T1}{Обучить и опубликовать и опубликованы оригинальные нейросетевые модели для использования в рамках диалоговой платформы DREAM и библиотеки DeepPavlov, в том числе трансформер-агностичные многозадачные модели с предложенной автором диссертационной работы архитектурой.}

\newcommand{\dream_TASK1}{Создать и выложить в открытый доступ диалоговую платформу DREAM, на которой в дальнейшем могло бы изучаться прикладное применение многозадачных нейросетевых моделей. Проверить качество этой диалоговой платформы международным конкурсом "Alexa Prize Socialbot Grand Challenge".}
\newcommand{\dream_NOVELTY1}{Была создана и выложена в открытый доступ диалоговая платформа DREAM, на которй в дальнейшем может изучаться прикладное применение многозадачных нейросевых моделей. Качество этой диалоговой платформы было проверено международным конкурсом "Alexa Prize Socialbot Grand Challenge".}
\newcommand{\dream_DEFPOS1}{}
\newcommand{\pseudolabel_TASK1}{}
\newcommand{\pseudolabel_NOVELTY1}{}
\newcommand{\pseudolabel_DEFPOS1}{}
\newcommand{\tr-ag_TASK1}{Провести эксперименты для сравнения различных вариантов выбора архитектуры многозадачных нейросетевых моделей, а также выбора сэмплирования и выбора псевдоразметки данных для определенных типов таких моделей.}
\newcommand{\tr-ag_NOVELTY1}{}
\newcommand{\tr-ag_DEFPOS1}{}
\newcommand{\tr-ag_TASK2}{Провести эмпирический анализ закономерностей переноса знаний в трансформер-агностичных многозадачных нейросетевых моделях между английским и русским языком, а также между различными диалоговыми задачами. В частности, исследовать зависимость этого переноса от размера обучающей выборки.}
\newcommand{\tr-ag_NOVELTY2}{}
\newcommand{\tr-ag_DEFPOS2}{}
\newcommand{\rutopics_TASK1}{Предложить новый русскоязычный открытый набор тематических данных для решения задачи русскоязычной тематической классификации и фундаментальных задач исследования переноса знаний на разговорных данных.}
\newcommand{\rutopics_NOVELTY1}{Был предложен новый русскоязычный открытый набор тематических данных \texttt{YAQTopics} для решения задачи русскоязычной тематической классификации и фундаментальных задач исследования переноса знаний.}
\newcommand{\rutopics_DEFPOS1}{Предложенный автором новый русскоязычный открытый набор тематических данных \texttt{YAQTopics} подходит для решения задачи русскоязычной тематической классификации и фундаментальных задач исследования переноса знаний.}
\newcommand{\rutopics_TASK2}{Проверить зависимость межъязыкового переноса знаний на разговорных данных в многоязычных нейросетевых моделях от размера предобучающей выборки и генеалогической близости языков.}
\newcommand{\rutopics_NOVELTY2}{Была проверена зависимость межъязыкового переноса знаний на разговорных данных в многоязычных нейросетевых моделях от размера предобучающей выборки и генеалогической близости языков.}
\newcommand{\rutopics_DEFPOS2}{Для многоязычных нейросетевых моделей качество переноса знаний на разные языки на тематических данных сильно коррелирует с размером предобучающей выборки для каждого языка, но при этом не коррелирует с генеалогической близостью этого языка к русскому.}
\newcommand{\mtldream_TASK1}{}
\newcommand{\mtldream_NOVELTY1}{}
\newcommand{\mtldream_DEFPOS1}{}




Для достижения поставленной цели необходимо было решить следующие {\tasks}:
\begin{enumerate}

  \item 
  \item Исследовать различные варианты выбора архитектуры многозадачных моделей на основе архитектуры Трансформер, а также выбора сэмплирования и выбора псевдоразметки данных для определенных типов таких моделей.
  \item Исследовать перенос знаний в многозадачных моделях на основе архитектуры Трансформер между английским и русским языком, а также между различными диалоговыми задачами. В частности, исследовать зависимость этого переноса от размера обучающей выборки.
  
  \item Предложить новый русскоязычный открытый набор тематических данных для прикладных задач диалоговой платформы DREAM и фундаментальных задач исследования переноса знаний.
  \item Проверить зависимость межъязыкового переноса знаний на разговорных данных в многоязычных нейросетевых моделях от размера предобучающей выборки и генеалогической близости языков.
  \item Разработать и опубликовать многозадачные трансформер-агностичные модели для диалоговой платформы DREAM и библиотеки DeepPavlov.
  
  \item Создать и выложить в открытый доступ диалоговую платформу DREAM, на которой в дальнейшем могло бы изучаться прикладное применение многозадачных нейросетевых моделей.
  
  \item Интегрировать многозадачные трансформер-агностичные модели в диалоговую платформу DREAM. Помимо этого, решить другие прикладные задачи, связанные с этой платформой( такие, как разработка сценарных навыков).
  \newline
  \newline
\end{enumerate}

{\defpositions}
\begin{enumerate}
\item Предложенные многозадачные трансформер-агностичные модели модели демонстрируют более высокую степень экономии памяти, чем однозадачные модели ( прирост числа параметров порядка 0.1\%). Эти модели были применены в диалоговой платформе DREAM и библиотеке DeepPavlov.
\item Предложенные многозадачные трансформер-агностичные модели демонстрируют уровень качества, приближающийся к качеству однозадачных моделей. Просадка среднего качества на исследованных наборах данных не превышает один процент.
\item Предложенный тематический набор русскоязычных данных \texttt{YAQTopics} пригоден для решения задачи тематической классификации разговорных данных.
\item Для задач с относительно маленьким объемом данных по сравнению с другими задачами и при этом достаточно большой степенью похожести на какие-то задачи, имеющиеся бОльший объем данных, многозадачные модели показывают прирост метрик. Этот прирост усиливается для каждой задачи, если обучающая выборка достаточно мала.
\item Если номенклатура классов для английского и русского языка соответствует друг другу, то добавление англоязычных данных к русскоязычным для той же задачи помогает поднять качество многоязычной нейросетевой модели. Чем меньше русскоязычных данных, тем прирост качества выше, как для таких многозадачных моделях, так и для однозадачных.
\item В условиях предыдущего пункта перенос знаний для многозадачных моделей происходит эффективнее, если данные добавляются в рамках одной задачи, а не в рамках разных.
\item Для многоязычных нейросетевых моделей качество переноса знаний на разные языки на тематических данных сильно коррелирует с размером предобучающей выборки для каждого языка, но при этом не коррелирует с генеалогической близостью этого языка к русскому.
\item Трансформер-агностичность многозадачной нейросетевой модели повышает ее гибкость и удобство для стройки.
\item Предложенные схемы псевдоразметки данных повышают качество многозадачных моделей на тестовой выборке, но создает риск переобучения в случае возникновения дисбаланса классов.

\item Диалоговая платформа DREAM показывает результаты, близкие к лучшим аналогичным системам на момент создания. При этом она находится в открытом доступе, что дает возможность 
\end{enumerate}

{\novelty}
\begin{enumerate}
  \шеуь 
  \item Впервые было проведено комплексное исследование влияния выбора методов сэмплирования, архитектуры и псевдоразметки данных для многозадачных моделей на основе архитектуры Трансформер.
  \item Был исследован перенос знаний в многозадачных моделях на основе архитектуры Трансформер между английским и русским языком, а также между различными диалоговыми задачами. В частности, была исследована зависимость этого переноса от размера обучающей выборки.
  \item Был предложен новый русскоязычный открытый набор тематических данных \texttt{YAQTopics} для прикладных задач диалоговой платформы DREAM и фундаментальных задач исследования переноса знаний.
  \item Была проверена зависимость межъязыкового переноса знаний на разговорных данных в многоязычных нейросетевых моделях от размера предобучающей выборки и генеалогической близости языков.
  \item Была создана и выложена в открытый доступ диалоговая платформа DREAM, на которой в дальнейшем изучалось прикладное применение многозадачных нейросетевых моделей.
  \item Обучены и опубликованы оригинальные нейросетевые модели для использования в рамках диалоговой платформы DREAM и библиотеки DeepPavlov, в том числе трансформер-агностичные многозадачные модели с предложенной автором диссертационной работы архитектурой.
  \ite
  