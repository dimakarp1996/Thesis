%% Согласно ГОСТ Р 7.0.11-2011:
%% 5.3.3 В заключении диссертации излагают итоги выполненного исследования, рекомендации, перспективы дальнейшей разработки темы.
%% 9.2.3 В заключении автореферата диссертации излагают итоги данного исследования, рекомендации и перспективы дальнейшей разработки темы.
%ПРОСТО СКОПИРОВАЛ ПОЛОЖЕНИЯ ВЫНОСИМЫЕ ДЛЯ ЗАЩИТЫ
\begin{enumerate}
  \item {Псевдоразметка данных при помощи однозадачных моделей улучшает метрики многозадачных моделей. При этом объединение классов оправдывает себя только для задач, достаточно сильно похожих друг на друга.}
  \item {Для достаточно малых данных рассмотренные автором многозадачные энкодер-агностичные модели, основанные на архитектуре Трансформер, начинают превосходить по своей средней точности однозадачные, в особенности - за счет задач с наименьшим объемом данных. При этом для таких многоязычных моделей наблюдается также перенос знаний с английского языка на русский в рамках одной задачи, и чем меньше русскоязычных данных, тем сильнее выражен перенос. Эта закономерность справедлива и для однозадачных моделей.}
  \item {Для многоязычных нейросетевых моделей качество переноса знаний на разные языки на тематических данных сильно коррелирует с размером предобучающей выборки для каждого языка, но при этом не коррелирует с генеалогической близостью этого языка к языку дообучения.}
  %\item {Платформа Dream пригодна для изучения прикладного применения многозадачных нейросетевых моделей.}
  %\item {Рассмотренные многозадачные нейросетевые архитектуры пригодны для практического применения в диалоговых платформах и в рамках open-source библиотек. При этом исследованные и внедренные автором энкодер-агностичные нейросетевые модели выигрывают у модульных архитектур за счет энкодер-агностичности, а у моделей с одним линейным слоем - за счёт большей гибкости, отсутствия необходимости в псевдоразметке и как следствие - меньшей склонности к переобучению.}
\end{enumerate}
