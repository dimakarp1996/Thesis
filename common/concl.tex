%% Согласно ГОСТ Р 7.0.11-2011:
%% 5.3.3 В заключении диссертации излагают итоги выполненного исследования, рекомендации, перспективы дальнейшей разработки темы.
%% 9.2.3 В заключении автореферата диссертации излагают итоги данного исследования, рекомендации и перспективы дальнейшей разработки темы.
%ПРОСТО СКОПИРОВАЛ ПОЛОЖЕНИЯ ВЫНОСИМЫЕ ДЛЯ ЗАЩИТЫ
Выявлен ряд закономерностей:
\begin{enumerate}
  \item {Псевдоразметка данных с использованием однозадачных моделей улучшает метрики многозадачных моделей. При этом объединение классов оправдывает себя только для задач, достаточно сильно похожих друг на друга.}
  \item {Для достаточно малых данных многозадачные энкодер-агностичные модели превосходят по своей средней точности однозадачные, в особенности -- за счет задач с наименьшим объемом данных. При этом для таких многоязычных моделей наблюдается также перенос знаний с английского языка на русский в рамках одной задачи, и чем меньше русскоязычных данных, тем сильнее выражен перенос. Эта закономерность справедлива и для однозадачных моделей.}
  \item {Для многоязычных нейросетевых моделей качество переноса знаний на разные языки на тематических данных сильно коррелирует с размером предобучающей выборки для каждого языка, но при этом после поправки на размер предобучающей выборки статистически значимая корреляция с генеалогической близостью этого языка к языку дообучения не обнаруживается.}
  %\item {Платформа Dream пригодна для изучения прикладного применения многозадачных нейросетевых моделей.}
  %\item {Рассмотренные многозадачные нейросетевые архитектуры пригодны для практического применения в диалоговых платформах и в рамках open-source библиотек. При этом исследованные и внедренные автором энкодер-агностичные нейросетевые модели выигрывают у модульных архитектур за счет энкодер-агностичности, а у моделей с одним линейным слоем - за счёт большей гибкости, отсутствия необходимости в псевдоразметке и как следствие - меньшей склонности к переобучению.}
\end{enumerate}

Результаты работы также обладают достаточной практической значимостью, так как:
\begin{enumerate}
  \item Был создан ряд компонент диалоговой платформы мирового уровня, впервые в России вышедшей в полуфинал престижных мировых конкурсов Alexa Prize 3 и Alexa Prize 4. В число этих компонент входят многозадачные нейросетевые модели: многозадачная нейросетевая модель с одним линейным слоем, многозадачная нейросетевая модель на основе архитектуры PAL-BERT и многозадачная энкодер-агностичная нейросетевая модель. Диалоговая платформа имеет полностью открытый код, что дает возможность легкого переиспользования любой части проделанной над ней работы. При этом многозадачная энкодер-агностичная модель даёт на девяти задачах данной диалоговой платформы экономию видеопамяти $\sim$90\% и экономию оперативной памяти $\sim$79\% по сравнению с аналогичными однозадачными моделями, даже не учитывая эффект от возможности быстрой замены базовой модели.
  \item Программный код для реализации многозадачной энкодер-агностичной нейросетевой модели встроен в библиотеку DeepPavlov, имевшую более 500000 скачиваний на момент встраивания кода. Данные модели позволяют решить большое число задач без дополнительных вычислительных затрат, не считая затрат на использование задаче-специфичных линейных слоёв (всего $\sim$0.1\% дополнительных параметров для решения сразу пяти задач вместо одной).
\end{enumerate}

Таким образом, цель данной работы - определение закономерностей, влияющих на перенос знаний между языками и задачами в многозадачных нейросетевых моделях на различных архитектурах и на особенности прикладного применения этих моделей в диалоговых платформах - выполнена. 