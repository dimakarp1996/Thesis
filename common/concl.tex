%% Согласно ГОСТ Р 7.0.11-2011:
%% 5.3.3 В заключении диссертации излагают итоги выполненного исследования, рекомендации, перспективы дальнейшей разработки темы.
%% 9.2.3 В заключении автореферата диссертации излагают итоги данного исследования, рекомендации и перспективы дальнейшей разработки темы.
\begin{enumerate}
  \item На основе анализа \ldots
  \item Численные исследования показали, что \ldots
  \item Математическое моделирование показало \ldots
  \item Исходный код всех упомянутых в предыдущих пунктах моделей и программ опубликован в открытом доступе, как набор данных YAQTopics, как часть библиотеки DeepPavlov и диалоговой платформы DREAM или же как дополнительные материалы к статьям. 
\end{enumerate}
\iffalse
сследование зависимости качества решения задачи классификации
текстов от доменной специфичности, в частности языкового стиля, по
казало, что использование векторных представлений соответствующего
целевой задаче домена позволяет улучшить результаты для задач клас
сификации текстов (до 3.6 пунктов F-1 метрики на 5 разных задачах).
Предложенный подход был применен к построению классификаторов
для диалоговой системы DREAM.
2. Предложенный метод интеграции нейросетевых моделей предсказания
здравого смысла повышает количество высказываний бота, содер
жащих или дополняющих контекст диалога до здравого смысла.
Предложенные разговорные навыки, использующие данный метод,
были интегрированы в диалоговую систему DREAM испытаны и при
менены на реальных пользователях.
3. В соответствие с предложенной схемой разметки уровней здравого
смысла в диалогах, был размечен закрытый набор реальных диа
логовых данных, проведено исследование корреляции предложенной
разметки здравого смысла и нескольких автоматических метрик.
4. Исследование корреляции представленной разметки здравого смысла
и автоматических метрик показало, что тональность и токсичность ре
акции пользователя скоррелированы с явными проявления здравого
смысла и полным отсутствием здравого смысла.
5. Предложенные алгоритмы управления диалогом интегрированы в
компоненты Skill Selector и Response Selector диалоговой систе
мы DREAM в конкурсах «Alexa Prize Challenge 3» и «Alexa Prize
Challenge 4».
6. Предложенный алгоритм Response Selector на основе тегов позволил
улучшить выбор финального ответа. Доля реплик, соответствующих
контексту, возросла на 15.5% по сравнению с базовым эвристическим
алгоритмом и более, чем на 20% по сравнению с версией, не использу
ющей предложенные условия.
116

\fi