\usepackage[cp1251]{inputenc}
 % Новые переменные, которые могут использоваться во всём проекте 
 % ГОСТ 7.0.11-2011 
 % 9.2 Оформление текста автореферата диссертации 
 % 9.2.1 Общая характеристика работы включает в себя следующие основные структурные 
 % элементы: 
 % актуальность темы исследования; 
 \newcommand{\actualityTXT}{Актуальность темы.
Актуальность темы обоснована стремительным развитием нейросетевых моделей. В последнее время нейросетевые модели на основе трансформеров, типа BERT, стали чаще применяться в различных областях, в том числе в диалоговых системах. Это связано с тем, что они показывают более высокие результаты, чем иные методы машинного обучения. В то же самое время, такие модели требуют вычислительных ресурсов, которые могут быть дорогостоящими. } 
 % степень ее разработанности; 
 \newcommand{\progressTXT}{Степень разработанности темы. В связи с актуальностью данной темы развитие получает идея многозадачного обучения - использование одной и той же модели для решения нескольких задач машинного обучения. Тем не менее, различные методы многозадачного обучения и их применение в диалоговых системах еще изучены не до конца. } 
 % цели и задачи; 
 \newcommand{\aimTXT}{Целью данной работы является исследование различных методов подготовки и псевдоразметки данных для многозадачного машинного обучения,а также  исследование применения различных нейросетевых архитектур для многозадачного машинного обучения при обработке естественного языка в диалоговых системах.} 
\newcommand{\tasksTXT}{ задачи:
 
1. Обучить и опубликовать в открытом доступе однозадачные модели классификации, адаптированные для разговорных дан­
ных конкурса Alexa Prize.
2. Предложить и разработать сценарные разговорные навыки для
диалоговой системы.
3. Разработать и опубликовать многозадачные модели, объединяющие опубликованные ранее однозадачные модели.
4. Исследовать влияние аугментации данных, изменения архитектуры и методов сэмплирования на многозадачные модели, используя русскоязычные и англоязычные данные. На основании этих исследований - выбрать оптимальные комбинации архитектуры, методов сэмплирования и аугментации данных для дальнейшего применения.
5. Исследовать применение многозадачных моделей, обученные на основании предыдущего пункта, на реальных пользователях.} 
 % научную новизну; 
 \newcommand{\noveltyTXT}{Научная новизна:
1. Впервые было проведено комплексное исследование влияния выбора методов псевдоразметки данных на обучение нейросетевых многозадачных моделей различных архитектур.
2. Обучены и опубликованы оригинальные однозадачные и многозадачные нейросетевые модели, готовые к использованию в качестве компонентов диалоговой системы DREAM.
3. Предложены и опубликованы оригинальные разговорные навыки для диалоговой системы DREAM. } 
 % appropriatiobln_passport; 
\newcommand{\passportTXT}{Соответствие диссертации паспорту научной специальности. Пункты 1 и 2 Научной новизны соотвествуют пункту 5 "Комплексные исследования научных и технических проблем с применением современной технологии математического моделирования и
вычислительного эксперимента." специальности 05.13.18 «Математическое моделирование, численные методы и комплексы программ». Пункт 3  Научной новизны соответствует пункту 8 «Разработка систем компьютерного и имитаци­онного моделирования» специальности 05.13.18 «Математическое моделирование,
численные методы и комплексы программ».} 
 % или чаще используют просто 
 \newcommand{\influenceTXT}{Практическая значимость. Практическая значимость заключается в следующем. \newlineВпервые в России были разработаны диалоговые системы мирового уровня, вышедшие в полуфинал престижных мировых конкурсов Alexa Prize 3 и Alexa Prize 4 (в конкурсах было 10 и 9 участников соответственно, из более чем 300 кандидатов). \newline В этих системах, помимо основанных на правилах алгоритмах, применялись нейросетевые модели, в том числе и многозадачные описанные в диссертации модели. \newline Помимо этого, программный код данных диалоговых систем был выложен в открытый доступ, что сделало результаты работы более доступными для мирового сообщества. \newline Программный код для реализации многозадачной трансформер-инвариантной нейросетевой модели используется в библиотеке DeepPavlov и имел более.... скачиваний на март 2023 года.
Алгоритмы, использованные в данной работе, применены также в программе для ЭВМ [А6], на которую получено свидетельство о государственной регистрации.} 
 % методологию и методы исследования; 
 \newcommand{\methodsTXT}{Методология и методы исследования. В данной работе были применены: \begin{enumerate} \item метод численного эксперимента для исследования задач классификации текстов; \item основы теории вероятностей; \item методы машинного обучения и теории глубокого обучения; \item методы разработки на языках Python, Bash, включая разработку программного кода для библиотек с открытым исходным кодом
DeepPavlov и DeepPavlov Agent \end{enumerate}} 
 % положения, выносимые на защиту; 
\newcommand {\reliabilityTXT}{Достоверность полученных результатов обеспечивается экспериментами на наборах диалоговых данных, а также применением в соревнованиях "Alexa Prize Challenge 3 и "Alexa Prize Challenge 4". Результаты находятся в качественном соответствии с результатами, полученными другими авторами.} 
 \newcommand{\probationTXT}{Апробация работы. Результаты работы были представлены автором на конференции Диалог-2021 [А4], и опубликованы в журналах Proceedings of En&T 2018[А1], Computational Linguistics and Information Technologies[А4], Proceedings of Alexa Prize 3[А2],Proceedings of Alexa Prize 4[А3], Труды МФТИ [А5]. Помимо этого, разработки, описанные в данной диссертации, были внедрены в диалоговые системы \texttt{DREAM} и \texttt{DREAM} 2, используемые в конкурсе Alexa Prize 3, Alexa Prize 4 и после него. Также на разработки, основанные на использованных в предыдущих работах алгоритмах, было получено свидетельство о регистрации ПО [A6].}
  
 \newcommand{\contributionTXT}{Личный вклад.В работе [А1] автор отвечал за ряд важных компонент диалоговой системы, включающих в себя парафразер. Исследование, разработка и сравнительный анализ методов псевдоразметки данных, описанных в [A4], были выполнены автором самостоятельно. В работах [A2],  [A5] автор отвечал за ряд важных компонент диалоговой системы - навыки обсуждения книг, эмоций, коронавируса, классификатор эмоций, и также за не ставший частью системы генеративный навык. В работе [A3] автор, помимо вышеупомянутых навыков, отвечал также за встройку многозадачного классификатора, основанного на модели PAL-BERT. В дальнейшем автор отвечал за встройку многозадачных моделей в диалоговую систему DREAM. Алгоритмы для обработки естественного языка, использованные в [A2], [A3], [A5], использовались также в [A6].} 
 \newcommand{\publicationsTXT}{Публикации. Основные результаты по теме диссертации изложены в 5 печатных изданиях, 1 из которых издано в журналах, индексируе мых RSCI, 1 — в периодических научных журналах, индексируемых Web of Science и Scopus, 1 — в тезисах докладов. Получено свидетельство о депонировании на  1 программу для ЭВМ. } 
  

  
 %%% Заголовки библиографии: 
  
 % для автореферата: 
 \newcommand{\bibtitleauthor}{Публикации автора по теме диссертации:
 смотреть как вставляются
} 
  
 % для стиля библиографии `\insertbiblioauthorgrouped` 
 \newcommand{\bibtitleauthorvak}{В изданиях из списка ВАК РФ} 
 \newcommand{\bibtitleauthorscopus}{В изданиях, входящих в международную базу цитирования Scopus} 
 \newcommand{\bibtitleauthorwos}{В изданиях, входящих в международную базу цитирования Web of Science} 
 \newcommand{\bibtitleauthorother}{В прочих изданиях} 
 \newcommand{\bibtitleauthorconf}{В сборниках трудов конференций} 
  
 % для стиля библиографии `\insertbiblioauthorimportant`: 
 \newcommand{\bibtitleauthorimportant}{Наиболее значимые \protect\MakeLowercase\bibtitleauthor} 
  
 % для списка литературы в диссертации и списка чужих работ в автореферате: 
\newcommand{\bibtitlefull}{Список литературы} % (ГОСТ Р 7.0.11-2011, 4)



