\section*{Общая характеристика работы}

\newcommand{\actuality}{\underline{\textbf{\actualityTXT}}}
\newcommand{\progress}{\underline{\textbf{\progressTXT}}}
\newcommand{\aim}{\underline{{\textbf\aimTXT}}}
\newcommand{\tasks}{\underline{\textbf{\tasksTXT}}}
\newcommand{\novelty}{\underline{\textbf{\noveltyTXT}}}
\newcommand{\appropriation}{\underline{\textbf{\appropriationTXT}}}
\newcommand{\influence}{\underline{\textbf{\influenceTXT}}}
\newcommand{\methods}{\underline{\textbf{\methodsTXT}}}
\newcommand{\defpositions}{\underline{\textbf{\defpositionsTXT}}}
\newcommand{\reliability}{\underline{\textbf{\reliabilityTXT}}}
\newcommand{\probation}{\underline{\textbf{\probationTXT}}}
\newcommand{\contribution}{\underline{\textbf{\contributionTXT}}}
\newcommand{\publications}{\underline{\textbf{\publicationsTXT}}}


{\actuality} 
Актуальность темы обоснована стремительным развитием нейросетевых моделей. В последнее время нейросетевые модели на основе трансформеров, типа BERT, стали чаще применяться в различных областях, в том числе в диалоговых системах. Это связано с тем, что они показывают более высокие результаты, чем иные методы машинного обучения. В то же самое время, такие модели требуют вычислительных ресурсов, которые могут быть дорогостоящими. В связи с этим развитие получает идея многозадачного обучения - использование одной и той же модели для решения нескольких задач машинного обучения. Тем не менее, перенос данных в многозадачных нейросетевых моделях между задачами и языками, особенно для задач, применимых в диалоговых системах, всё еще не изучены до конца. Наборов данных в открытом доступе для задач диалоговых систем, таких, как тематическая классификация, также недостаточно.


{\aim} данной работы является определение закономерностей, влияющих на перенос знаний между языками и задачами в многозадачных нейросетевых моделях на различных архитектурах, а также на особенности прикладного применения этих моделей в диалоговой платформе.

Для достижения поставленной цели необходимо было решить следующие {\tasks}:

\begin{enumerate}
  \item {Создать и выложить в открытый доступ диалоговую платформу, на которой в дальнейшем могло бы изучаться прикладное применение многозадачных нейросетевых моделей. Проверить качество этой диалоговой платформы оценками пользователей.}
  \item {Проверить применимость технологий, использованных в диалоговой платформе DREAM, в иных прикладных задачах.}
  \item {Провести эксперименты для сравнения различных схем псевдоразметки данных для многозадачных нейросетевых моделей с одним линейным слоем.}
  \item {Провести эксперименты для сравнения различных вариантов выбора архитектуры многозадачных нейросетевых моделей, а также выбора сэмплирования для определенных типов таких моделей.}
  \item {Провести эксперименты для анализа закономерностей переноса знаний в трансформер-агностичных многозадачных нейросетевых моделях между различными диалоговыми задачами. В частности, провести оценку зависимости этого переноса от размера обучающей выборки.}
  \item {Провести эксперименты для анализа закономерностей переноса знаний в многоязычных трансформер-агностичных многозадачных нейросетевых моделях между различными языками - с английского языка на русский. В частности, провести оценку зависимости этого переноса от размера обучающей выборки. Рассмотреть также применимость этих выводов для однозадачных моделей.}
  \item {Проверить пригодность русскоязычного открытого набора тематических данных для решения задачи русскоязычной тематической классификации и фундаментальных задач исследования переноса знаний на разговорных данных.}
  \item {Проверить зависимость межъязыкового переноса знаний на разговорных данных в многоязычных нейросетевых моделях от размера предобучающей выборки и генеалогической близости языков.}
  \item {Интегрировать рассмотренные в диссертации многозадачные нейросетевые архитектуры в диалоговую платформу, оценить применимость данных архитектур и провести их сравнительный анализ на основе опыта практического применения. На основании этого анализа произвести интеграцию также в open-source библиотеку.}
  \newline
  \newline
\end{enumerate}


{\novelty}
\begin{enumerate}
  \item {Создана и выложена в открытый доступ диалоговая платформа DREAM, на которй в дальнейшем может изучаться прикладное применение многозадачных нейросетевых моделей. Качество этой диалоговой платформы было проверено оценками пользователей в рамках конкурса "Alexa Prize Socialbot Grand Challenge", по результатам которых платформа DREAM вышла в полуфинал этого конкурса.}
  \item {Проверена применимость технологий, использованных в диалоговой платформе DREAM, на прикладной задаче по созданию сервиса для работы с текстами texter-ocr-cv-microservice.}
  \item {Проведены эксперименты для сравнения различных схем псевдоразметки данных для многозадачных нейросетевых моделей с одним линейным слоем на примере задач из набора данных GLUE.}
  \item {Проведены эксперименты для сравнения различных вариантов выбора архитектуры многозадачных трансформер-агностичных нейросетевых моделей, сравнения их с аналогичными однозадачными моделями для разных тел, а также выбора сэмплирования для определенных типов таких моделей.}
  \item {Проведены эксперименты для анализа закономерностей переноса знаний в трансформер-агностичных многозадачных нейросетевых моделях между различными диалоговыми задачами. В частности, была произведена оценка зависимости этого переноса от размера обучающей выборки.}
  \item {Проведены эксперименты для анализа закономерностей переноса знаний в многоязычных трансформер-агностичных многозадачных нейросетевых моделях между различными языками - с английского языка на русский. В частности, была проведена оценка зависимости этого переноса от размера обучающей выборки. Была рассмотрена также применимость этих выводов для однозадачных моделей.}
  \item {Проверена пригодность хотя бы 1 русскоязычного открытого набора тематических данных \texttt{YAQTopics} для решения задачи русскоязычной тематической классификации и фундаментальных задач исследования переноса знаний.}
  \item {Проверена зависимость межъязыкового переноса знаний на разговорных данных в многоязычных нейросетевых моделях от размера предобучающей выборки и генеалогической близости языков.}
  \item {Рассмотренные в диссертации многозадачные нейросетевые архитектуры были интегрированы в диалоговую платформу DREAM, была оценена применимость и был проведен их сравнительный анализ на основе опыта применения. На основании этого анализа была произведена также интеграция в open-source библиотеку DeepPavlov} 
\end{enumerate}

{\appropriation}
Пункты 3,4, 5, 6, 7 и 8 Научной новизны соответствуют Пункту 8 "Комплексные исследования научных и технических проблем с применением современной технологии математического моделирования и
вычислительного эксперимента." специальности 1.2.2 «Математическое моделирование, численные методы и комплексы программ». специальности 1.2.2 «Математическое моделирование, численные методы и комплексы программ», так как предложенный набор данных может использовать для валидации различных моделей машинного обучения. Пункты 1,2 и 9 Научной новизны соответствует пункту 6 "Разработка систем компьютерного и имитационного моделирования, алгоритмов и методов имитационного моделирования на основе анализа математических моделей" специальности 1.2.2 «Математическое моделирование, численные методы и комплексы программ».

{\defpositions}
\begin{enumerate}
  \item {Диалоговая платформа \texttt{DREAM} пригодна для изучения прикладного применения многозадачных нейросетевых моделей. Оценки пользователей в рамках международного конкурса "Alexa Prize Socialbot Grand Challenge", обеспечившие двукратный выход в полуфинал этой платформы, показывают высокое качество диалоговой платформы на момент её создания.}
  \item {На примере прикладной задачи по созданию сервиса для работы с текстами texter-ocr-cv-microservice показана применимость технологий, использованных в диалоговой платформе DREAM, за пределами этой диалоговой платформы.}
  \item {Псевдоразметка данных при помощи однозадачных моделей улучшает метрики многозадачных моделей. При этом объединение классов оправдывает себя только для задач, достаточно сильно похожих друг на друга.}
  \item {Было показано на различных наборах данных, что многозадачные трансформер-агностичные нейросетевые модели показывают себя не хуже ряда других, более сложных архитектур, а предложенный метод сэмплирования - не хуже ряда других методов сэмплирования. При этом многозадачные трансформер-агностичные модели по данным проведенным экспериментах дают среднюю просадку не более 1 процента по сравнению с однозадачными моделями. А если какие-то задачи достаточно похожи друг на друга, как например, в бенчмарке GLUE, многозадачные модели за счет таких задач в среднем даже превосходят однозадачные модели.}
  \item {Было показано, что для достаточно малых данных многозадачные трансформер-агностичных модели начинают превосходить по своей средней точности однозадачные, в особенности - за счет задач с наименьшим объемом данных.}
  \item {Было показано, что если в основе многозадачной трансформер-агностичной модели лежит многоязычный BERT, то добавление английских данных к русским при соответствующей номенклатуре классов позволяет улучшить метрики на 1-5\%. Чем меньше изначально русскоязычных данных, тем улучшение сильнее. Этот же вывод справедлив и для однозадачных моделей.}
  \item {Русскоязычный открытый набор тематических данных \texttt{YAQTopics} пригоден для решения задачи русскоязычной тематической классификации и фундаментальных задач исследования переноса знаний.}
  \item {Для многоязычных нейросетевых моделей качество переноса знаний на разные языки на тематических данных сильно коррелирует с размером предобучающей выборки для каждого языка, но при этом не коррелирует с генеалогической близостью этого языка к русскому.}
  \item {Рассмотренные многозадачные нейросетевые архитектуры пригодны для практического применения в диалоговых платформах и в рамках open-source библиотек. При этом предложенные автором трансформер-агностичные нейросетевые модели выигрывают у моделей типа PAL-BERT за счет трансформер-агностичности, а у моделей с одним линейным слоем - за счёт большей гибкости, отсутствия необходимости в псевдоразметке и как следствие - меньшей склонности к переобучению.}
\end{enumerate}


\iffalse
Направления исследований 1.2.2:
1. Разработка новых математических методов моделирования объектов и
явлений (физико-математические науки).
2. Разработка, обоснование и тестирование эффективных вычислительных
методов с применением современных компьютерных технологий.
3. Реализация эффективных численных методов и алгоритмов в виде
комплексов проблемно-ориентированных программ для проведения
вычислительного эксперимента.
4. Разработка новых математических методов и алгоритмов интерпретации
натурного эксперимента на основе его математической модели.
5. Разработка новых математических методов и алгоритмов валидации
математических моделей объектов на основе данных натурного эксперимента
или на основе анализа математических моделей.
6. Разработка систем компьютерного и имитационного моделирования,
алгоритмов и методов имитационного моделирования на основе анализа
математических моделей (технические науки).
7. Качественные или аналитические методы исследования математических
моделей (технические науки).
8. Комплексные исследования научных и технических проблем с
применением современной технологии математического моделирования и
вычислительного эксперимента.
9. Постановка и проведение численных экспериментов, статистический
анализ их результатов, в том числе с применением современных
компьютерных технологий (технические науки).
\fi


{\influence}
Практическая значимость заключается в следующем. Впервые в России была разработана диалоговая платформа мирового уровня, вышедшая в полуфинал престижных мировых конкурсов Alexa Prize 3 и Alexa Prize 4 (в конкурсах было 10 и 9 участников соответственно, из более чем 300 кандидатов). Эта диалоговая платформы имеет полностью открытый код, что дает возможность легкого переиспользования любой части проделанной над ней работы. 

В этой платформе, в числе всего прочего, применялись многозадачные нейросетевые модели, описанные автором в данном работе - многозадачная нейросетевая модель с одним линейным слоем, многозадачная нейросетевая модель на основе архитектуры PAL-BERT и многозадачная трансформер-агностичная нейросетевая модель. Был проведен ряд других прикладных работ для улучшения данной платформы, включающий в себя разработку сценарных навыков.

Помимо этого, программный код для реализации многозадачной трансформер-агностичной нейросетевой модели встроен в библиотеку DeepPavlov, имеющую более 500 000 скачиваний на март 2023 года.
Алгоритмы, использованные в данной работе, применены также в программе для ЭВМ [А6], на которую получено свидетельство о государственной регистрации.


{\methods}
В данной работе были
применены:
\begin{enumerate}
\item Метод численного эксперимента для исследования задач обработки естественного языка;
\item Основы теории вероятностей;
\item Методы машинного обучения и теории глубокого обучения;
\item Методы разработки на языках Python, Bash.
\end{enumerate}




{\reliability} полученных результатов обеспечивается экспериментами на наборах диалоговых данных и наборе данных GLUE, описанными в~\cite{pseudolabel}(индексируется в Scopus), ~\cite{Болотин_Карпов_Рашков_Шкурак_2019}, ~\cite{rumtl},~\cite{enmtl},~\cite{rutopics},~\cite{dp_2023}, применением в соревнованиях "Alexa Prize Challenge 3" и "Alexa Prize Challenge 4", описанным в ~\cite{dream1}, ~\cite{dream2}, ~\cite{dream1_trudy}(входит в список ВАК), а также использованием результатов работы в диалоговой платформе \texttt{DREAM} и библиотеке DeepPavlov. Результаты находятся в соответствии с результатами, полученными другими авторами.


{\probation}
Апробация работы. Результаты работы были представлены автором на конференции Диалог-2021~\cite{pseudolabel}, и опубликованы в журналах Proceedings of En\&T 2018~\cite{Болотин_Карпов_Рашков_Шкурак_2019}, Computational Linguistics and Information Technologies~\cite{pseudolabel,rumtl},Proceedings of Alexa Prize 3~\cite{dream1},Proceedings of Alexa Prize 4~\cite{dream2}, Труды МФТИ~\cite{dream1_trudy}, Proceedings of AINL 2023~\cite{rutopics}, Proceedings of InterSpeech 2023~\cite{enmtl}, Proceedings of ACL Systems Workshop~\cite{dp_topics}. Помимо этого, разработки, описанные в данной диссертации, были внедрены в находящуюся в открытом доступе диалоговую платформу \texttt{DREAM}, активно используемую в конкурсах Alexa Prize 3, Alexa Prize 4 и после них. Также на разработки, основанные на результатах работы, было получено свидетельство о регистрации ПО~\cite{Дуплякин_Дмитрий_Ондар_Ушаков_2021}.


{\contribution} В работе~\cite{Болотин_Карпов_Рашков_Шкурак_2019} автор отвечал за ряд важных компонент диалоговой системы, включающих в себя парафразер. Исследование, разработка и сравнительный анализ методов псевдоразметки данных, описанных в работе~\cite{pseudolabel}, были выполнены автором самостоятельно. В работах~\cite{dream1,dream1_trudy,dream2} автор отвечал за ряд важных компонент диалоговой системы - навыки обсуждения книг, эмоций, коронавируса, классификаторы эмоций,интентов,момента остановки диалога, TF-IDF Retrieval, Grounding Skill, Gossip Skill, генеративный навык и многозадачную нейросетевую модель. Технические решения для работы с естественным языком, использованные в \cite{dream1,dream1_trudy,dream2}, использовались также в~\cite{Дуплякин_Дмитрий_Ондар_Ушаков_2021}. В работах~\cite{rumtl,enmtl,rutopics} все исследования также были выполнены автором самостоятельно. В работе~\cite{dp_2023} автор отвечал за эксперименты с многозадачными трансформер-агностичными моделями для библиотеки DeepPavlov и их описание.

\ifnumequal{\value{bibliosel}}{0}
{%%% Встроенная реализация с загрузкой файла через движок bibtex8. (При желании, внутри можно использовать обычные ссылки, наподобие `\cite{vakbib1,vakbib2}`).
    {\publications} 

}%
{%%% Реализация пакетом biblatex через движок biber
    \begin{refsection}[bl-author]
        % Это refsection=1.
        % Процитированные здесь работы:
        %  * подсчитываются, для автоматического составления фразы "Основные результаты ..."
        %  * попадают в авторскую библиографию, при usefootcite==0 и стиле `\insertbiblioauthor` или `\insertbiblioauthorgrouped`
        %  * нумеруются там в зависимости от порядка команд `\printbibliography` в этом разделе.
        %  * при использовании `\insertbiblioauthorgrouped`, порядок команд `\printbibliography` в нём должен быть тем же (см. biblio/biblatex.tex)
        %
        % Невидимый библиографический список для подсчёта количества публикаций:
        \printbibliography[heading=nobibheading, section=1, env=countauthorvak,          keyword=biblioauthorvak]%
        \printbibliography[heading=nobibheading, section=1, env=countauthorwos,          keyword=biblioauthorwos]%
        \printbibliography[heading=nobibheading, section=1, env=countauthorscopus,       keyword=biblioauthorscopus]%
        \printbibliography[heading=nobibheading, section=1, env=countauthorconf,         keyword=biblioauthorconf]%
        \printbibliography[heading=nobibheading, section=1, env=countauthorother,        keyword=biblioauthorother]%
        \printbibliography[heading=nobibheading, section=1, env=countauthor,             keyword=biblioauthor]%
        \printbibliography[heading=nobibheading, section=1, env=countauthorvakscopuswos, filter=vakscopuswos]%
        \printbibliography[heading=nobibheading, section=1, env=countauthorscopuswos,    filter=scopuswos]%
        %
        \nocite{*}%
        %
        {\publications} Основные результаты по теме диссертации изложены в~\arabic{citeauthor}~печатных изданиях,
        \arabic{citeauthorvak} из которых изданы в журналах, рекомендованных ВАК\sloppy%
        \ifnum \value{citeauthorscopuswos}>0%
            , \arabic{citeauthorscopuswos} "--- в~периодических научных журналах, индексируемых Web of~Science и Scopus\sloppy%
        \fi%
        \ifnum \value{citeauthorconf}>0%
            , \arabic{citeauthorconf} "--- в~тезисах докладов.
        \else%
            .
        \fi
    \end{refsection}%
    \begin{refsection}[bl-author]
        % Это refsection=2.
        % Процитированные здесь работы:
        %  * попадают в авторскую библиографию, при usefootcite==0 и стиле `\insertbiblioauthorimportant`.
        %  * ни на что не влияют в противном случае
        \nocite{Болотин_Карпов_Рашков_Шкурак_2019}%vak
        \nocite{dream1}%vak
        \nocite{dream2}%vak
        \nocite{pseudolabel}
        \nocite{dream1_trudy}%vak
        \nocite{Дуплякин_Дмитрий_Ондар_Ушаков_2021}%vak
        \nocite{rumtl}
        \nocite{rutopics}
        \nocite{enmtl}
        \nocite{dp_2023}
    \end{refsection}%
        %
        % Всё, что вне этих двух refsection, это refsection=0,
        %  * для диссертации - это нормальные ссылки, попадающие в обычную библиографию
        %  * для автореферата:
        %     * при usefootcite==0, ссылка корректно сработает только для источника из `external.bib`. Для своих работ --- напечатает "[0]" (и даже Warning не вылезет).
        %     * при usefootcite==1, ссылка сработает нормально. В авторской библиографии будут только процитированные в refsection=0 работы.
        %
        % Невидимый библиографический список для подсчёта количества внешних публикаций
        % Используется, чтобы убрать приставку "А" у работ автора, если в автореферате нет
        % цитирований внешних источников.
        % Замедляет компиляцию
    \ifsynopsis
    \ifnumequal{\value{draft}}{0}{
      \printbibliography[heading=nobibheading, section=0, env=countexternal,          keyword=biblioexternal]%
    }{}
    \fi
}
\iffalse
При использовании пакета \verb!biblatex! будут подсчитаны все работы, добавленные
в файл \verb!biblio/author.bib!. Для правильного подсчёта работ в~различных
системах цитирования требуется использовать поля:
\begin{itemize}
        \item \texttt{authorvak} если публикация индексирована ВАК,
        \item \texttt{authorscopus} если публикация индексирована Scopus,
        \item \texttt{authorwos} если публикация индексирована Web of Science,
        \item \texttt{authorconf} для докладов конференций,
        \item \texttt{authorother} для других публикаций.
\end{itemize}
Для подсчёта используются счётчики:
\begin{itemize}
        \item \texttt{citeauthorvak} для работ, индексируемых ВАК,
        \item \texttt{citeauthorscopus} для работ, индексируемых Scopus,
        \item \texttt{citeauthorwos} для работ, индексируемых Web of Science,
        \item \texttt{citeauthorvakscopuswos} для работ, индексируемых одной из трёх баз,
        \item \texttt{citeauthorscopuswos} для работ, индексируемых Scopus или Web of~Science,
        \item \texttt{citeauthorconf} для докладов на конференциях,
        \item \texttt{citeauthorother} для остальных работ,
        \item \texttt{citeauthor} для суммарного количества работ.
\end{itemize}
% Счётчик \texttt{citeexternal} используется для подсчёта процитированных публикаций.

Для добавления в список публикаций автора работ, которые не были процитированы в
автореферате требуется их~перечислить с использованием команды \verb!\nocite! в
\verb!Synopsis/content.tex!.
\fi
 % Характеристика работы по структуре во введении и в автореферате не отличается (ГОСТ Р 7.0.11, пункты 5.3.1 и 9.2.1), потому её загружаем из одного и того же внешнего файла, предварительно задав форму выделения некоторым параметрам

%Диссертационная работа была выполнена при поддержке грантов \dots

%\underline{\textbf{Объем и структура работы.}} Диссертация состоит из~введения,
%четырех глав, заключения и~приложения. Полный объем диссертации
%\textbf{ХХХ}~страниц текста с~\textbf{ХХ}~рисунками и~5~таблицами. Список
%литературы содержит \textbf{ХХX}~наименование.
\section*{Содержание работы}
Во \underline{\textbf{введении}} обосновывается актуальность
исследований, проводимых в~рамках данной диссертационной работы,
приводится обзор научной литературы по изучаемой проблеме,
формулируется цель, ставятся задачи работы, излагается научная новизна
и практическая значимость представляемой работы. 

\underline{\textbf{Первая глава}} является обзорной. В этой главе дается определение нейросетевых методов машинного обучения для задач обработки естественного языка. 
Дается представление о различных базовых понятиях, таких, как метод обратного распространения ошибки, полносвязная архитектура и токенизация. 

Нейросетевые методы машинного обучения - это методы, основанные на использовании искусственных нейронных сетей. В данной работе рассматривается класс искусственных нейронных сетей, представляющих собой совокупность слоев с функциями активации, таких, что подаваемые на вход данные проходят через различные слои по очереди, где каждый слой представляет собой многомерную функцию многих переменных. Итоговый выход нейронной сети подается в функцию потерь, после чего функция потерь оптимизируется методом обратного распространения ошибки. 

Метод обратного распространения ошибки - один из методов «обучения с учителем», то есть подход, при котором модель учится решать задачу, чтобы соответствовать набору примеров входных/выходных данных. Для определения того, насколько ответ, данный нейронной сетью, соответствует требуемому, вводится функция потерь. Далее выполняется поиск точки минимума функции потерь в пространстве параметров искусственной нейронной сети для данного набора примеров входных данных. 

Все самые главные достижения в области нейронных сетей в 21 веке были связаны именно с применением нейросетевых подходов. 

Одним из классических видов нейронных сетей являются полносвязные нейронные сети - сети, состоящие из полносвязных слоев. Будем называть полносвязным слоем с M нейронами взвешенную сумму значений входного вектора x размерности N, к каждому элементу которой затем применяется функция активации $\sigma(y)$:

\begin{equation}
\begin{split} 
\color{black}z=\sigma(y)\\
\color{black}y=W_{0}+W_{1}X
\end{split}
\label{nn:0}
\end{equation}
где $W_{1}$ - матрица весов (weights) полносвязного слоя размерности $N*M$, $W_{0}$ - матрица биасов (bias) полносвязного слоя размерности M, $\sigma$ - некая нелинейная функция активации.

где $i$ - индекс $K$-мерного вектора $z$.
Для регуляризации в таких слоях (как и в других, более сложных) применяется также дропаут, предложенный в~\cite{dropout}. При использовании данного метода некий процент элементов выходного вектора (как правило, 10-20\%) приравнивается к нулю. Такая техника мешает «переобучению» нейронной сети, улучшая тем самым ее обобщающую способность.

Токенизация текста - это его разбиение на элементарные единицы(токены) перед предобработкой. Один токен соответствует одному слову либо же одной более мелкой его единице(букве или слогу) в зависимости от метода токенизации. Для нижеописанного метода Word2Vec токеном является 1 слово, для нижеописанной архитектуры BERT - слово либо слог.

Помимо этого, в первой главе разбираются нейросетевые архитектуры, использовавшиеся в данной диссертационной работе. В частности, разбирается архитектура Трансформер и нейросетевая модель {BERT}, основанная на данной архитектуре.

Последний раздел главы посвящен многозадачным нейросетевым моделям. В нём даётся определение многозадачного обучения и приводится классификация многозадачных нейросетевых архитектур.

Многозадачное обучение - это метод разделения параметров между моделями, обучающимися выполнять несколько задач. Все имеющиеся многозадачные нейросетевые архитектуры можно разделить на четыре типа:
\begin{itemize}
\item[*] Параллельные архитектуры. Для данного типа архитектур одни и те же «общие» слои используются для примеров из каждой задачи, при этом выход «общих» слоев обрабатывается независимо своим специфическим слоем для каждой задачи. 
\item[*] Иерархические архитектуры. Для данного типа архитектур задачи обрабатываются зависимо друг от друга: так, результат классификации примера для одной из задач может использоваться при решении другой из задач как дополнительный входной параметр.
\item[*] Модульные архитектуры. Нейронная сеть в данных архитектурах делится на общие модули и задаче-специфичные модули, где общие модули имеют одни и те же веса для всех задач, а задаче-специфичные модули - свои веса для каждой из задач.
\item[*] Генеративно-состязательные архитектуры. Для данного типа архитектур генератор и дискриминатор обучаются совместно таким образом, что дискриминатор пытается предсказать, из какой задачи пример, по его выдаваемому генератором представлению. А генератор, соответственно, пытается сгенерировать такое представление, чтобы дискриминатор мог предсказать задачу как можно хуже. 
\end{itemize}
В завершении первого раздела приводится подробный обзор двух архитектур, на которых основывались последующие разделы данной диссертационной работы. А именно, модель {MT-DNN}, имеющая параллельную архитектуру, и модель {PAL-BERT}, имеющая модульную архитектуру. 

\underline{\textbf{Вторая глава}} посвящена обзору диалоговой платформы DREAM. В главе подробно рассмотрена структура этой диалоговой платформы, ее эволюция в течение конкурсов «Alexa Prize Socialbot Grand Challenge 3» и «Alexa Prize Socialbot Grand Challenge 4», в полуфинал которых вышла платформа.  Проводимые автором работы над этой платформой частично продолжали проводившуюся ранее автором работу над диалоговой моделью «с персоной»~\cite{Болотин_Карпов_Рашков_Шкурак_2019}.

%Именно потребности этой платформы стимулировали создание многозадачных нейросетевых моделей, описанных в данной диссертационной работе, в этой платформе они и получали свое прикладное применение. 
В главе показано, что диалоговая платформа DREAM пригодна для изучения прикладного применения многозадачных нейросетевых моделей.

\underline{\textbf{Третья глава}} посвящена многозадачным нейросетевым моделям с одним линейным слоем - простейшему типу многозадачных моделей. Такие модели имеют архитектуру, аналогичную оригинальной модели BERT, при этом в финальном линейном слое разные нейроны отвечают за разные задачи.

В данной главе проводились эксперименты на следующих задачах из набора данных GLUE:
\begin{enumerate}
    \item[*] QQP - задача классификации пар предложений из сайта Quora.com на 2 класса - и не дубликат.
    \item[*] MNLI - задача классификации пар предложений из различных тем на три класса - логическое следование, логическое противоречие и нейтральный(ни следование, ни противоречие). Валидационный и тестовый наборы данных в MNLI существуют в 2 вариантах - из тех же тем, что и тренировочный(MNLI-m) и из остальных тем(MNLI-mm).
    \item[*] SST-2 - задача классификации предложений на два класса - положительная тональность и отрицательная тональность.
    \item[*] RTE - задача классификации пар предложений на два класса - логическое следствие и нет логического следствия.
\end{enumerate}

Для этих задач сравниваются различные способы псевдоразметки данных при обучении многозадачной модели с одним линейным слоем:
\begin{enumerate}
\item[*] Базовый способ - стандартное обучение однозадачных моделей.
\item[*] Независимые метки - метки для каждой задачи считаются независимыми (т.е всего девять классов), каждый пример имеет метку 1 для того класса своей задачи, к которому он принадлежит и 0 для всех остальных классов всех остальных задач. 
\item[*] Мягкие независимые метки - аналогично предыдущему подходу, но вероятности всех остальных классов каждой из остальных задач считаются равными друг другу так, чтобы их сумма равнялась 1. 
\item[*] Дополненные независимые метки - аналогичен предыдущему подходу, но вероятности для всех классов каждой из остальных задач получаются при помощи предсказаний однозадачной модели, обученной исключительно на этой задаче.
\item[*] Мягкое вероятностное предположение - объединение классов с сокращением их числа до пяти: положительный, отрицательный, логическое следствие, логическое противоречие, нейтральный. В остальном та же логика, что и для режима Мягкие независимые метки.
\item[*] Мягкие предсказанные метки - аналогичен предыдущему, но все недостающие вероятности для каждой из задач не считаются равновероятными, а определяются дополнительной разметкой от модели для каждой задачи. 
\item[*] Жесткие предсказанные метки - аналогичен предыдущему, но для меток, полученных из предсказаний оригинальной модели, максимальная вероятность для каждой задачи округляется до 1, а все остальные вероятности до 0.
\item[*] Мягкие независимые метки, замороженная голова - аналогично режиму Мягкие независимые метки, но линейный слой заморожен и обучается только тело модели.
\item[*] Независимые метки, замороженная голова - аналогично режиму Независимые метки, но линейный слой заморожен и обучается только тело модели.
\end{enumerate}

\begin{table}[htbp]
\centering
\caption {Лучшая точность на тестовых данных (при лучшей скорости обучения из выбираемых, среднее по 3 запускам). }
\label{tab:ps2}% label всегда желательно идти после caption
\resizebox{\textwidth}{!}
{%
\begin{tabular}{|c||c|c|c|c|c|c|}
\hline
Название эксперимента &Среднее& \({RTE}\)& \({QQP}\)&\({MNLI-m}\)&\({MNLI-mm}\)&\({SST}\)\\
\hline
\hline
Базовый(из оригинальной статьи)& 78.8\% & 66.4\%  & 71.2\% & 84.6\% & 83.4\% & 93.5\% \\
\hline
Базовый(воспроизведенный)& 77.6\% & 62.7\% & 71.0\% & 83.1\% & 82.7\% & 93.5\% \\
\hline
Независимые метки &79.0\% & 71.5\% & 70.9\% & 82.7\% & 81.7\% & 91.3\% \\
\hline
\begin{tabular}[c]{@{}l@{}}Независимые метки,\\замороженная голова\end{tabular}   & 78.2\% & 66.9\%& \textbf{71.8\%} & 82.6\% & 81.8\% & 91.9\% \\
\hline
Мягкие независимые метки  &78.9\% & 69.3\% & 71.3\% & 82.8\% & 82.1\% & 92.6\% \\
\hline
\begin{tabular}[c]{@{}l@{}}Мягкие независимые метки,\\замороженная голова\end{tabular}  &79.1\% & 70.0\% & 71.5\% & 83.0\% &\textbf{ 82.3\%} & 92.4\% \\
\hline
Дополненные независимые метки &77.6\% & 64.2\%&\textbf{ 71.8\%} & 81.2\% & 80.7\% & \textbf{93.2\%} \\
\hline
Мягкое вероятностное предположение  &\textbf{79.7\%} & \textbf{72.7\%} & 70.7\% &\textbf{ 83.4\%} &\textbf{82.3\%} & 92.5\% \\
\hline
Мягкие предсказанные метки  & 78.8\% & 70.3\% & 70.7\% & 81.7\% & 81.7\% & 92.5\% \\
\hline
Жесткие предсказанные метки & 79.1\% & 71.3\% &71.1\% & 81.7\% & 81.4\% & 92.6\% \\
\hline

\end{tabular}
}
\end{table}

На задачах, сильнее всего похожих друг на друга (MNLI и RTE) лучше всего показывают себя способы, подразумевающие объединение меток, что говорит о том, что в определенных случаях оно может быть оправдано. 

С другой стороны, объединение меток не даёт улучшений для более разнородных задач, таких, как QQP и SST, что показывает ограничения этого метода. Лучше всего для таких задач показывает себя метод Дополненные независимые метки, подразумевающий параллельную псевдоразметку данных. 

Главный вывод из этой главы - псевдоразметка данных при помощи однозадачных моделей улучшает метрики многозадачных моделей. При этом объединение классов оправдывает себя только для задач, достаточно сильно похожих друг на друга. 

Результаты данной работы представлены в статье~\cite{pseudolabel}.

\underline{\textbf{Четвёртая глава}} посвящена трансформер-агностичным многозадачным моделям. В частности, подробно описывается архитектура трансформер-агностичной многозадачной модели. 
Модель позволяет более гибко подстраиваться под каждую задачу. В отличие от модели, описанной в предыдущей главе, она не требует параллельной разметки. В главе подробно раскрыты преимущества данной модели над многозадачной моделью с одним линейным слоем.

Отдельный раздел содержит описание экспериментов, которые не сработали - модификация [CLS]-выхода модели BERT, использование задаче-специфичных тренируемых токенов и методов дистилляции при помощи приближения весов, а также эксперименты с разными методами сэмплирования. 

Исследование пригодности трансформер-агностичных многозадачных нейросетевых моделей для решения диалоговых задач проводилось для пяти задач - классификация эмоций, тональности, токсичности, интентов и тематическая классификация. Для этих задач использовалилсь параллельные русскоязычные и англоязычные наборы данных - CEDR и go\_emotions для классификации эмоций, RuReviews и DynaSent для классификации тональности, RuToxic и Wiki Talk для классификации токсичности и MASSIVE для классификации интентов и тематической классификации.

В главе показано на данных наборах данных, что как для русского, так и для английского языка предложенные многозадачные модели либо незначительно хуже однозадачных, либо не хуже. Выводы также проверены на наборе данных GLUE. Показано также, что при уменьшении обучающей выборки многозадачные модели с какого-то достаточно маленького размера данных, порядка 200-2000 примеров на задачу, начинают превосходить однозадачные модели. В особенности за счет не самых больших наборов данных из выборки. Помимо этого, показано, что добавление англоязычных данных к русскоязычным эффективнее делать для многоязычных моделей, объединяя данные для каждой задачи, а не считая каждые такие данные отдельный задачей, при условии соответствия номенклатуры классов. И что само это добавление улучшает качество многоязычной модели на русскоязычных данных - от 1 до 5 процентов, чем изначально русскоязычных данных меньше, тем улучшение сильнее.

Результаты данной главы представлены в работах автора~\cite{rumtl,enmtl}.

\underline{\textbf{Пятая глава}} расширяет работу, проделанную в \textbf{четвертой главе}. В данной главе проводятся исследования русскоязычного тематического набора данных - YAQTopics.  Данный набор состоит из 76 тематических классов и имеет более 500 тысяч примеров - пар «вопрос-ответ» из сервиса «Яндекс.Кью»~\cite{yandex_q}. Этот набор данных существенно превосходит другие существовавшие до него русскоязычные наборы данных для разговорной тематической классификации - как по числу примеров, так и по числу тематических классов. В главе использовалась однометочная часть этого набора. 

\begin{table}[t]
\centering
\scalebox{0.7}{
\begin{tabular}{|c||c|c|c|c|c|} 
\textbf{тип данных}  & \multicolumn{2}{c|}{\textbf{однометочные}} & \multicolumn{2}{c|}{\textbf{многометочные}} & \multirow{2}{*}{\textbf{равноразмерные}}\\
\cline{1-5}
\textbf{класс}  & \multicolumn{1}{c|}{все} & \multicolumn{1}{c|}{отвеченные} & \multicolumn{1}{c|}{все} & \multicolumn{1}{c|}{отвеченные} & \\\hline \hline
\textit{Размер набора данных} & 361650 & 266597 & 170930 & 137341 & 264786\\ \hline
\textit{Размер 6классового поднабора данных} & 18864 & 15912 & 27191 & 20569 & 15830 \\ \hline
\textit{Музыка} & 9514 & 5809 & 4456 & 3287 & 5797 \\ \hline
\textit{Еда, напитки и кулинария} & 5750 & 4758 & 14096 & 11084 & 4723 \\ \hline
\textit{Медиа и коммуникации} & 4505 & 2637 & 5577 & 3948 & 2619 \\ \hline
\textit{Транспорт} & 2435 & 1625 & 1933 & 1387 & 1613 \\ \hline
\textit{Новости} & 945 & 602 & 912 & 720 & 600\\ \hline
\textit{Погода} & 890 & 481 & 217 & 143 & 478 \\ \hline
    
\end{tabular}
}
\caption{Размеры набора данных {YAQTopics} по классу и части}
%\centering
\label{tab:rutopics:sizes}
\end{table}

Для оценки качества данного набора данных использовался набор данных MASSIVE. Валидация проводилась на совпадающих 6 классах из валидационной части русскоязычного набора данных MASSIVE, а тестирование - на объединении тренировочных и тестовых частей этого набора. Все эксперименты проводились на 4 моделях - русскоязычный BERT от DeepPavlov, дистиллированный BERT от DeepPavlov, русскоязычный BERT от Сбербанка и многоязычный BERT от авторов оригинальной модели.

Показано при оценке на наборе данных MASSIVE~\cite{massive}, что данный набор данных хорошо подходит для тематической классификации(точность на вопросах из 6 классов {YAQTopics} 85 процентов для русскоязычных моделей на отвеченных однометочных данных). 

При этом самой информативной частью данного набора, повзоляющей определить тему, являются вопросы, так как конкатенация к ним ответов не давала стойких улучшений, использование же ответов вместо вопросов лишь ухудшало показатели. Данный вывод справедлив для всех базовых моделей. Использование суммаризованных ответов вместо ответов не изменяет этот вывод.

Было также показано на 5-кратной кроссвалидации, что все рассмотренные выше модели показывают точность выше 70 процентов даже на всём 76-классовом наборе однометочных данных из YAQTopics.

Также на примере обучения многоязычной модели \textit{bert-base-multilingual-cased} на данном наборе данных (см. Таблицу~\ref{tab:rutopics:crosslingual} можно сделать вывод, что при переносе знаний с русскоязычного набора данных YAQTopics на другие языки (51 язык) качество модели для каждого языка хорошо коррелирует с приближенным размером предобучающей выборки для этого языка ( корреляция Спирмена 0.773 с пи-значением 2.997e-11). При этом корреляция качества модели для каждого языка с генеалогической близостью этого языка к русскому не является статистически значимой.
\begin{table*}
\caption{Точность (f1) модели \textit{bert-base-multilingual-cased} на объединенном тестовом наборе данных {MASSIVE} для всех языков. Модель обучалась на версии \textbf{Q} набора данных {YAQTopics}. \textbf{Код} означает код языка(ISO 639-1), \textbf{N} означает число статей в Википедии на этом языке на 11 октября 2018 года, \textbf{Дистанция} означает лингвистическую дистанцию между этим языком и русским, посчитанную в соответствии с работой~\cite{lang_sim}. Усреднено по трем запускам.}
\label{tab:rutopics:crosslingual}
\centering
   \scalebox{0.5}{
\begin{tabular}{|c|c|c||c|c|c|} \hline
\multirow{2}{*}{\textbf{Язык}}  & \multirow{2}{*}{\textbf{Код}} & \multirow{2}{*}{\textbf{Дистанция}} & \multirow{2}{*}{\textbf{Число статей}}  &  \multicolumn{2}{c|}{\textbf{Метрики}} \\ %\hline
\cline{5-6}
& & & & Точность & Макро-F1 \\ \hline \hline
русский & ru & 0 & 1,501,878 & 80.8 & 79.8\\
китайский (Тайвань) & zh-TW & 92.2 & 1,025,366 & 79.6 & 79.1\\
китайский & zh & 92.2 & 1,025,366 & 78.0 & 77.7\\
английский & en & 60.3 & 5,731,625 & 75.2 & 75.6\\
японский & ja & 93.3 & 1,124,097 & 72.4 & 70.5\\
словенский & sl & 4.2 & 162,453 & 70.3 & 69.0\\
шведский & sv & 59.5 & 3,763,579 & 70.2 & 69.6\\
малайский & ms & n/c & 320,631 & 68.9 & 67.7\\
итальянский & it & 45.8 & 1,466,064 & 68.8 & 68.0\\
индонезийский & id & 91.2 & 440,952 & 68.7 & 67.5\\
нидерландский & nl & 64.6 & 1,944,129 & 68.7 & 68.5\\
португальский & pt & 61.6 & 1,007,323 & 68.6 & 68.7\\
испанский & es & 51.7 & 1,480,965 & 68.2 & 68.0\\
датский & da & 66.2 & 240,436 & 67.8 & 66.7\\
французский & fr & 61.0 & 2,046,793 & 65.5 & 65.5\\
персидский & fa & 72.4 & 643,750 & 65.2 & 64.2\\
турецкий & tr & 86.2 & 316,969 & 64.5 & 62.4\\
вьетнамский & vi & 95.0 & 1,190,187 & 64.3 & 65.1\\
норвежский букмол & nb & 67.2 & 495,395 & 64.3 & 64.0\\
польский & pl & 5.1 & 1,303,297 & 64.2 & 62.2\\
азербайджанский & az & 87.7 & 138,538 & 63.9 & 63.1\\
каталанский & ca & 60.3 & 591,783 & 61.4 & 60.4\\
венгерский & hu & 87.2 & 437,984 & 61.3 & 60.0\\
иврит & he & 88.9 & 231,868 & 60.9 & 59.5\\
хинди & hi & 69.8 & 127,044 & 60.7 & 58.7\\
корейский & ko & 89.5 & 429,369 & 60.4 & 59.6\\
румынский & ro & 55.0 & 388,896 & 57.1 & 53.9\\
урду & ur & 66.7 & 140,939 & 56.4 & 55.9\\
арабский & ar & 86.5 & 619,692 & 56.2 & 55.7\\
каннада & kn & 90.8 & 23,844 & 56.1 & 53.0\\
филиппинский & tl & 91.9 & 80,992 & 55.0 & 51.3\\
телугу & te & 96.7 & 69,354 & 53.7 & 49.3\\
финский & fi & 88.9 & 445,606 & 53.3 & 51.3\\
бирманский & my & 86.0 & 39,823 & 52.5 & 49.7\\
африкаанс & af & 64.8 & 62,963 & 52.4 & 50.3\\
тамильский & ta & 94.7 & 118,119 & 52.4 & 50.1\\
немецкий & de & 64.5 & 2,227,483 & 52.2 & 51.6\\
албанский & sq & 69.4 & 74,871 & 51.5 & 47.2\\
латышский & lv & 49.1 & 88,189 & 49.6 & 48.4\\
малаялам & ml & 96.7 & 59,305 & 48.7 & 46.3\\
армянский & hy & 77.8 & 246,571 & 48.1 & 47.5\\
бенгальский & bn & 66.3 & 61,294 & 47.3 & 45.3\\
тайский & th & 89.5 & 127,010 & 46.5 & 44.9\\
греческий & el & 75.3 & 153,855 & 46.3 & 44.8\\
грузинский & ka & 96.0 & 124,694 & 39.2 & 38.1\\
яванский & jv & 95.4 & 54,964 & 38.7 & 37.1\\
монгольский & mn & 86.2 & 18,353 & 36.6 & 33.7\\
исландский & is & 68.9 & 45,873 & 32.6 & 29.9\\
суахили & sw & 95.1 & 45,806 & 31.0 & 28.0\\
валлийский & cy & 75.5 & 101,472 & 28.5 & 25.3\\
кхмерский & km & 97.1 & 6,741 & 16.1 & 8.6\\
амхарский & am & 86.6 & 14,375 & 12.1 & 5.0\\
\end{tabular}
}
\end{table*}

Результаты данной работы представлены в статье~\cite{rutopics}.

\underline{\textbf{Шестая глава}} посвящена прикладному использованию многозадачных моделей, описанных в данной работе.

Первой версией многозадачных моделей были модели с одним линейным слоем. Эти модели использовались в диалоговой платформе DREAM для замены облачных классификаторов реплик и классификаторов качества диалога от Amazon. Модели обучались на предсказаниях соответствующих моделей от Amazon, которые были сделаны в течение первого из двух конкурсов Alexa Prize, в котором автор работы принимал участие. Данные конкурса были разбиты в соотношении 90/8/2 на тренировочную, валидационную и тестовую выборку. Тестирование для задач классификации тональности, эмоций и токсичности проводилось на их оригинальных наборах тестовых данных (упомянутых во второй главе), тестирование для задач Cobot Topics, Cobot DialogAct Topics и Cobot DialogAct Intents - на тестовой подвыборке

Заметим, что разметка для всех используемых примеров была параллельной, т.к все классификаторы от Amazon отрабатывали для каждой из реплик. Эти модели поддерживали задачи классификации токсичности, тональности и эмоций, а также замену классификаторов Cobot Topics, Cobot DialogAct Topics, Cobot DialogAct Intents. Т.е использованный подход был аналогичен подходу «Жесткие независимые метки» из третьей главы, но без объединения меток. 

Результаты этой модели представлены ниже.

\begin{table}[htbp]
    \caption{Точность(взвешенный макро-F1) для многозадачной классификации для различных моделей. Однозадачные модели означает оригинальные модели, 6 в 1 - многозадачную модель с одним линейным слоем, обученную на аннотациях всех упомянутых в таблице классификаторов, 3 в 1(кобот) - многозадачную модель с одним линейным слоем, обученную только на аннотациях классификаторов cobot topics, cobot dialogact topics и cobot dialogact intents, 3 в 1(не кобот) - многозадачную модель с одним линейным слоем, обученную только на аннотациях остальных классификаторов(классификаторы эмоций, тональности и токсичности).}
    \label{mtldream:1}
    \centering
    \scalebox{0.65}{
    \begin{tabular}{c|c|c|c|c} 
    \hline
    \textbf{Модель} & \textbf{Однозадачные модели} & \textbf{6 в 1} & \textbf{3 в 1(Кобот)} & \textbf{3 в 1(Не кобот)}\\ 
    \hline
    cobot topics   & --- & 0.84~(0.83) & 0.82~(0.84) & --- \\
    \hline
    cobot dialogact topics  & --- & 0.76~(0.64) & 0.78~(0.66) & --- \\ 
    \hline
    cobot dialogact intents & --- & 0.69~(0.65) & 0.70~(0.67) & --- \\ 
    \hline
    классификация эмоций  & 0.92~(0.75) & 0.82~(0.60) & --- & 0.85~(0.67) \\
    \hline
    классификация тональности & 0.72~(0.68) & 0.60~(0.57) & --- & 0.66~(0.62) \\ 
    \hline
    классификация токсичности & 0.92~(0.60) & 0.92~(0.59) & --- & 0.93~(0.60)\\ 
    \hline
    \end{tabular}}
\end{table}

Аналогичная модель была обучена также для замены модуля от Amazon, оценивавшего качество диалога по пяти метрикам - ответ интересный, ответ развлекает пользователя, ответ понятный, ответ ошибочный, ответ по теме. Обученный на массиве диалогов из Alexa Prize Challenge 4, линейный слой предсказывал вектор из 5 величин от 0 до 1, метрик для мониторинга считалась средним квадратичным отклонением.

Использование данной модели дало СКО(среднеквадратичное отклонение) показателей, равное 0.31, на тестовой выборке.

Второй версией, работавшей в диалоговой платформе DREAM в течение продолжительного времени, являлась модель на основе PAL-BERT - модели, описанной в первой главе. Эта модель показала более высокое качество, чем модель с одним линейным слоем, даже если им подавались на вход одни и те же псевдоразмеченные данные Alexa Prize, как можно видеть из следующей таблицы.
\begin{table}[htbp]
\centering
\caption {Точность (f1) с диалоговой историей для многозадачной модели с 1 линейным слоем и PAL-BERT на псевдоразмеченных данных из Alexa Prize Challenge 4, оценка на «чистых» тестовых данных для не-коботовских задач и на псевдоразмеченных для коботовских задач.}
\label{mtldream:4}
\resizebox{\textwidth}{!}{%
\begin{tabular}{|c||c|c|c|c|} \hline
Задача | Модель & \begin{tabular}[c]{@{}l@{}}7 в 1\\ без истории\end{tabular} & \begin{tabular}[c]{@{}l@{}}7 в 1\\ без истории \\ жесткие метки\end{tabular} & \begin{tabular}[c]{@{}l@{}}7 в 1\\ PAL-BERT \\ без истории \\ жесткие метки\end{tabular} & \begin{tabular}[c]{@{}l@{}}Оригинальные\\ модели\end{tabular} \\

\hline
\hline
cobot topics & \textbf{0.819(0.8)} & 0.802(0.781) & 0.818(0.795) & 1(1) \\
\hline
cobot dialogact topics & 0.805(0.626) & 0.799(0.616) & \textbf{0.814(0.632)} & 1(1) \\
\hline
cobot dialogact intents & 0.752(0.635) & 0.745(0.626) & \textbf{0.767(0.63)} & 1(1) \\
\hline
Классификация эмоций & 0.401(0.245) & 0.72(0.641) & \textbf{0.788(0.754)} & 0.92(0.751) \\
\hline
Классификация тональности & 0.683(0.607) & 0.727(0.609) & \textbf{0.733(0.585)} & 0.721(0.681) \\
\hline
Классификация токсичности & 0.932(0.194) & 0.931(0.18) & \textbf{0.935(0.186)} & 0.922(0.596) \\
\hline
Классификация фактоидности & 0.805(0.806) & 0.816(0.814) & \textbf{0.829(0.831)} & 0.886(0.884) \\
\hline
\end{tabular}
}
\end{table}

Преимущество PAL-BERT над моделью с одним линейным слоем сохранилось и после добавления в обе модели обучения на псевдоразмеченных данных, не относящихся к Alexa Prize (а именно, на псевдоразмеченных обучающих данных для не-Коботовских задач). Это позволило приблизить метрики PAL-BERT к метрикам оригинальных моделей.

Результаты этой модели в финальной серии экспериментов с PAL-BERT представлены ниже.


\begin{table}[htbp]
\centering
\caption {Точность (f1) для оценки моделей в третьей серии экспериментов. Для не-Коботовских задач при оценке используются оригинальные тестовые наборы данных, для коботовских - тестовая часть разбиения данных.}
\label{mtldream:5}
\resizebox{\textwidth}{!}{%
\begin{tabular}{|c||c|c|c|c|c|c|c|} \hline
Модель/Задача & \begin{tabular}[c]{@{}l@{}}7 в 1\\ без истории\\базовый\end{tabular} &\begin{tabular}[c]{@{}l@{}}7 в 1\\ с историей\\базовый\end{tabular} & \begin{tabular}[c]{@{}l@{}}7 в 1\\ PAL-BERT\\без псевдоразметки\end{tabular} & \begin{tabular}[c]{@{}l@{}}7 в 1\\ PAL-BERT \\ псевдоразметка \\ только \\ неКоботовских \\данных\end{tabular} & \begin{tabular}[c]{@{}l@{}}7 в 1\\ PAL-BERT \\ полная \\псевдоразметка\end{tabular} & \begin{tabular}[c]{@{}l@{}}7 в 1\\ PAL-BERT \\псевдоразметка \\только \\для тональности\\ и фактоидности\end{tabular} & \begin{tabular}[c]{@{}l@{}} Оригинальные \\ модели \end{tabular}\\
\hline
\hline
cobot topics & \textbf{0.701(0.666)} & 0.568(0.533) & 0.833(0.81) & 0.831(0.808) & \textbf{0.863(0.843)} & 0.828(0.81) & 1(1) \\
\hline
cobot dialogact topics & 0.756(0.519) & 0.852(0.667) & \textbf{0.871(0.704)} & 0.869(0.704) & \textbf{0.906(0.804)} & 0.868(0.698) & 1(1) \\
\hline
cobot dialogact intents & 0.515(0.406) & 0.728(0.516) & \textbf{0.768(0.563)} & 0.765(0.561) & \textbf{0.828(0.685)} & 0.753(0.554) & 1(1) \\
\hline
Классификация эмоций & 0.905(0.88) & 0.917(0.883) & \textbf{0.927(0.906)} & 0.924(0.893) & 0.923(0.897) & 0.926(0.91) & 0.92(0.751) \\
\hline
Классификация тональности & 0.72(0.633) & 0.713(0.657) & \textbf{0.706(0.648)} & 0.727(0.659) & 0.713(0.647) & \textbf{0.754(0.664)} & 0.721(0.681) \\
\hline
Классификация токсичности & 0.938(0.199) & 0.932(0.21) & \textbf{0.928(0.253)} & 0.932(0.298) & 0.932(0.269) & \textbf{0.939(0.259)} & 0.922(0.596) \\
\hline
Классификация фактоидности & 0.789(0.809) & 0.794(0.817) & \textbf{0.834(0.831)} & 0.846(0.844) & \textbf{0.869(0.866)} & 0.854(0.853) & 0.886(0.884) \\
\hline
\end{tabular}
}
\end{table}


В дальнейшем данная модель была заменена трансформер-агностичной моделью, описанной в четвёртой главе. Это было связано в первую очередь с изменениями технических требований системы DREAM, в число которых вошла возможность быстро подставлять разные базовые модели в многозадачную архитектуру. Данные требования привели к переходу на параллельную архитектуру - а именно, трансформер-агностичную модель.


В трансформер-агностичную модель была добавлена классификация семантических интентов MIDAS и тематическая классификация DeepPavlov Topics, также были актуализированы либо дополнительно предобработаны данные для других задач. 

Трансформер-агностичность модели позволила заменить обычную модель дистиллированной, дополнительно выиграв в памяти.

\begin{table}[htbp]
\centering
\caption {Точность/взвешенный макро-F1) для оценки моделей в экспериментах с трансформер-агностичными моделями. Для не-Коботовских задач при оценке используются оригинальные тестовые наборы данных, для коботовских - тестовая часть разбиения данных. Мы обозначаем как distilbert модель \textit{distilbert-base-uncased}, как bert модель \textit{bert-base-uncased}. "Однозадачный" означает, что отдельная модель обучалась для каждой из задач.}
\label{tab:tr-ag-dream}% label всегда желательно идти после caption
\resizebox{\textwidth}{!}{
\begin{tabular}{|c||c|c|c|c|c|c|} \hline
Задача/модель & Размер обучающей выборки &\begin{tabular}[c]{@{}l@{}}distilbert\\однозадачный\\с историей в MIDAS\end{tabular} & \begin{tabular}[c]{@{}l@{}}distilbert\\многозадачный\\с историей в MIDAS\end{tabular}  & \begin{tabular}[c]{@{}l@{}}distilbert\\многозадачный\\без истории в MIDAS\end{tabular} & \begin{tabular}[c]{@{}l@{}}bert\\однозадачный\\с историей в MIDAS\end{tabular} & \begin{tabular}[c]{@{}l@{}}bert\\многозадачный\\с историей в MIDAS\end{tabular}\\ \hline \hline
Классификация эмоций              & 39.5к & 70.47/70.30 & 68.18/67.86 & 67.59/67.32         & 71.48/71.16 & 67.27/67.23 \\ \hline
Классификация токсичности            & 1.62M & 94.53/93.64 & 93.84/93.5  & 93.86/93.41         & 94.54/93.15 & 93.94/93.4 \\ \hline
Классификация тональности            & 94k  & 74.75/74.63 & 72.55/72.21 & 72.22/71.9          & 75.95/75.88 & 75.65/75.62 \\ \hline
Классификация фактоидности           & 3.6k & 81.69/81.66 & 81.02/81.07 & 80.0/79.86          & 84.41/84.44 & 80.34/80.09 \\ \hline
Классификация интентов MIDAS          & 7.1k & 80.53/79.81 & 72.73/71.56~ & 73.69/73.26 & 82.3/82.03  & 77.01/76.38 \\ \hline
Тематическая классификация(DeepPavlov Topics) & 1.8M & 87.48/87.43 & 86.98/86.9  & 87.01/87.05         & 88.09/88.1  & 87.43/87.47 \\ \hline
Cobot topics~                  & 216k & 79.88/79.9  & 77.31/77.36 & 77.45/77.35         & 80.68/80.67 & 78.21/78.22 \\ \hline
Cobot dialogact topics~             & 127k & 76.81/76.71 & 76.92/76.79 & 76.8/76.7          & 77.02/76.97 & 76.86/76.74 \\ \hline
Cobot dialogact intents             & 318k & 77.07/77.7  & 76.83/76.76 & 76.65/76.57         & 77.28/77.72 & 76.96/76.89 \\ \hline
Средее для 9 задач                   & 4218k & 80.36/80.20    & 78.48/78.22 & 78.36/78.15         & 81.31/81.12  & 79.3/79.11 \\ \hline
Видеопамяти использовано, Мб               &    & 2418*9=21762 & 2420     & 2420             & 3499*9=31491 & 3501    \\ \hline
\end{tabular}
}
\end{table}

По сравнению с классификаторами в версии диалоговой платформы DREAM до внедрения многозадачной трансформер-агностичной модели (т.е основанной на модели PAL-BERT) многозадачная трансформер-агностичная модель дала экономию видеопамяти в 75 процентов, экономию оперативной памяти в 57 процентов и экономию времени на классификацию в 80-85 процентов.
 
Такая большая экономия времени на классификацию в основном связана с эффектом от трансформер-агностичности.\footnote{Хотя, конечно, роль сыграло и то, что для задачи классификации интентов MIDAS больше не требовался отдельный классификатор.} Если при использовании PAL-BERT для каждой задачи было необходимо получать предсказания многозадачной модели «с нуля», даже если они принимают одну и ту же фразу на вход, то при использовании многозадачной трансформер-агностичной модели появилась возможность один раз получить выход базового трансформера для этой фразы и дальше для всех других задачах, принимающих ее на вход, работать с этим выходом только линейными слоями, которые на порядки быстрее.

При этом, по сравнению с многозадачной моделью с одним линейным слоем, многозадачная трансформер-агностичная модель имеет ряд качественных преимуществ. Так, она не требует параллельной псевдоразметки под все задачи, что упрощает обучение модели. Проведение параллельной псевдоразметки может вносить нежелательные искажения в обучающую выборку, как показал опыт применения обученных на псевдоразмеченных данных моделей в диалоговой платформе DREAM. Помимо этого, многозадачная трансформер-агностичная модель может поддерживать не только многометочную, но и однометочную классификацию, а также ряд других задач, включающий в себя распознавание именованных сущностей или выбор между несколькими вариантами ответа.

Данные преимущества носили при выборе между этими двумя архитектурами многозадачных моделей решающий характер, несмотря на их сопоставимые показатели по расходу вычислительных ресурсов.

Трансформер-агностичная модель, помимо диалоговой платформы DREAM, была также внедрена в библиотеку DeepPavlov~\cite{dp_2023}(версия 1.1).

%Можно сослаться на свои работы в автореферате. Для этого в файле
%\verb!Synopsis/setup.tex! необходимо присвоить положительное значение
%счётчику \verb!\setcounter{usefootcite}{1}!. В таком случае ссылки на
%работы других авторов будут подстрочными.
%Изложенные в третьей главе результаты опубликованы в~\cite{vakbib1, vakbib2}.
%Использование подстрочных ссылок внутри таблиц может вызывать проблемы.

В \underline{\textbf{заключении}} приведены основные результаты работы, которые заключаются в следующем:
%% Согласно ГОСТ Р 7.0.11-2011:
%% 5.3.3 В заключении диссертации излагают итоги выполненного исследования, рекомендации, перспективы дальнейшей разработки темы.
%% 9.2.3 В заключении автореферата диссертации излагают итоги данного исследования, рекомендации и перспективы дальнейшей разработки темы.
%ПРОСТО СКОПИРОВАЛ ПОЛОЖЕНИЯ ВЫНОСИМЫЕ ДЛЯ ЗАЩИТЫ
\begin{enumerate}
  \item {Созданная при существенном вкладе автора платформа DREAM, находящаяся в открытом доступе, пригодна для изучения прикладного применения многозадачных нейросетевых моделей.}
  %\item {На примере прикладной задачи по созданию сервиса для работы с текстами texter-ocr-cv-microservice показана применимость технологий, использованных в диалоговой платформе DREAM, за пределами этой диалоговой платформы.}
  \item {Псевдоразметка данных при помощи однозадачных моделей улучшает метрики многозадачных моделей. При этом объединение классов оправдывает себя только для задач, достаточно сильно похожих друг на друга.}
  \item {Многозадачные трансформер-агностичные нейросетевые модели показывают себя не хуже ряда других, более сложных архитектур, а предложенный метод сэмплирования - не хуже ряда других методов сэмплирования. При этом многозадачные трансформер-агностичные модели по данным проведенных экспериментов дают среднюю просадку не более 1 процента по сравнению с однозадачными моделями. При достаточной степени похожести задач друг на друга модели за счет таких задач в среднем даже превосходят однозадачные модели.}
  \item {Для достаточно малых данных многозадачные трансформер-агностичных модели начинают превосходить по своей средней точности однозадачные, в особенности - за счет задач с наименьшим объемом данных.}
  \item {Если в основе многозадачной трансформер-агностичной модели лежит многоязычный BERT, то добавление английских данных к русским при соответствующей номенклатуре классов позволяет улучшить метрики на 1-5\%. Чем меньше изначально русскоязычных данных, тем улучшение сильнее. Этот же вывод справедлив и для однозадачных моделей.}
  \item {Русскоязычные модели, обученные на 6 классах из русскоязычного набора тематических данных YAQTopics, показывают точность около 85\% на наборе данных MASSIVE. Это оправдывает использование набора данных YAQTopics для решения задачи русскоязычной тематической классификации и фундаментальных задач исследования переноса знаний.}
  \item {Для многоязычных нейросетевых моделей качество переноса знаний на разные языки на тематических данных сильно коррелирует с размером предобучающей выборки для каждого языка, но при этом не коррелирует с генеалогической близостью этого языка к языку дообучения.}
  \item {Рассмотренные многозадачные нейросетевые архитектуры пригодны для практического применения в диалоговых платформах и в рамках open-source библиотек. При этом предложенные автором трансформер-агностичные нейросетевые модели выигрывают у моделей типа PAL-BERT за счет трансформер-агностичности, а у моделей с одним линейным слоем - за счёт большей гибкости, отсутствия необходимости в псевдоразметке и как следствие - меньшей склонности к переобучению.}
\end{enumerate}


\ifdefmacro{\microtypesetup}{\microtypesetup{protrusion=false}}{} % не рекомендуется применять пакет микротипографики к автоматически генерируемому списку литературы
\urlstyle{rm}                               % ссылки URL обычным шрифтом
\ifnumequal{\value{bibliosel}}{0}{% Встроенная реализация с загрузкой файла через движок bibtex8
  \renewcommand{\bibname}{\large \bibtitleauthor}
  \nocite{*}
  \insertbiblioauthor           % Подключаем Bib-базы
  %\insertbiblioexternal   % !!! bibtex не умеет работать с несколькими библиографиями !!!
}{% Реализация пакетом biblatex через движок biber
  % Цитирования.
  %  * Порядок перечисления определяет порядок в библиографии (только внутри подраздела, если `\insertbiblioauthorgrouped`).
  %  * Если не соблюдать порядок «как для \printbibliography», нумерация в `\insertbiblioauthor` будет кривой.
  %  * Если цитировать каждый источник отдельной командой --- найти некоторые ошибки будет проще.
  %
  %% authorvak
  \nocite{dream1_trudy}%
  %
  %% authorwos
  %
  %% authorscopus
  \nocite{pseudolabel}%
  \nocite{rutopics}
  \nocite{rumtl}
  \nocite{enmtl}
  %
  %% authorconf
  %
  %% authorother
  \nocite{dream1}%
  \nocite{dream2}%
  \nocite{Дуплякин_Дмитрий_Ондар_Ушаков_2021}

  \ifnumgreater{\value{usefootcite}}{0}{
    \begin{refcontext}[labelprefix={}]
      \ifnum \value{bibgrouped}>0
        \insertbiblioauthorgrouped    % Вывод всех работ автора, сгруппированных по источникам
      \else
        \insertbiblioauthor      % Вывод всех работ автора
      \fi
    \end{refcontext}
  }{
  \ifnum \value{citeexternal}>0
    \begin{refcontext}[labelprefix=A]
      \ifnum \value{bibgrouped}>0
        \insertbiblioauthorgrouped    % Вывод всех работ автора, сгруппированных по источникам
      \else
        \insertbiblioauthor      % Вывод всех работ автора
      \fi
    \end{refcontext}
  \else
    \ifnum \value{bibgrouped}>0
      \insertbiblioauthorgrouped    % Вывод всех работ автора, сгруппированных по источникам
    \else
      \insertbiblioauthor      % Вывод всех работ автора
    \fi
  \fi
  %  \insertbiblioauthorimportant  % Вывод наиболее значимых работ автора (определяется в файле characteristic во второй section)
 % \begin{refcontext}[labelprefix={}]    \insertbiblioexternal            % Вывод списка литературы, на которую ссылались в тексте автореферата
 % \end{refcontext}
  }
}
\ifdefmacro{\microtypesetup}{\microtypesetup{protrusion=true}}{}
\urlstyle{tt}                               % возвращаем установки шрифта ссылок URL
