\section*{Общая характеристика работы}

\newcommand{\actuality}{\underline{\textbf{\actualityTXT}}}
\newcommand{\progress}{\underline{\textbf{\progressTXT}}}
\newcommand{\aim}{\underline{{\textbf\aimTXT}}}
\newcommand{\tasks}{\underline{\textbf{\tasksTXT}}}
\newcommand{\novelty}{\underline{\textbf{\noveltyTXT}}}
\newcommand{\influence}{\underline{\textbf{\influenceTXT}}}
\newcommand{\methods}{\underline{\textbf{\methodsTXT}}}
\newcommand{\defpositions}{\underline{\textbf{\defpositionsTXT}}}
\newcommand{\reliability}{\underline{\textbf{\reliabilityTXT}}}
\newcommand{\probation}{\underline{\textbf{\probationTXT}}}
\newcommand{\contribution}{\underline{\textbf{\contributionTXT}}}
\newcommand{\publications}{\underline{\textbf{\publicationsTXT}}}


{\actuality} 
Актуальность темы обоснована стремительным развитием нейросетевых моделей. В последнее время нейросетевые модели на основе трансформеров, типа BERT, стали чаще применяться в различных областях, в том числе в диалоговых системах. Это связано с тем, что они показывают более высокие результаты, чем иные методы машинного обучения. В то же самое время, такие модели требуют вычислительных ресурсов, которые могут быть дорогостоящими. В связи с этим развитие получает идея многозадачного обучения - использование одной и той же модели для решения нескольких задач машинного обучения. Тем не менее, перенос данных в многозадачных нейросетевых моделях между задачами и языками, особенно для задач, применимых в диалоговых системах, всё еще не изучены до конца. Наборов данных в открытом доступе для задач диалоговых систем, таких, как тематическая классификация, также недостаточно.


{\aim} данной работы является определение закономерностей, влияющих на перенос знаний между языками и задачами в многозадачных нейросетевых моделях на различных архитектурах, а также на особенности прикладного применения этих моделей в диалоговой платформе.

Для достижения поставленной цели необходимо было решить следующие {\tasks}:

\begin{enumerate}
  \item {Создать и выложить в открытый доступ диалоговую платформу, на которой в дальнейшем могло бы изучаться прикладное применение многозадачных нейросетевых моделей. Проверить качество этой диалоговой платформы оценками пользователей.}
  \item {Проверить применимость технологий, использованных в диалоговой платформе DREAM, в иных прикладных задачах.}
  \item {Провести эксперименты для сравнения различных схем псевдоразметки данных для многозадачных нейросетевых моделей с одним линейным слоем.}
  \item {Провести эксперименты для сравнения различных вариантов выбора архитектуры многозадачных нейросетевых моделей, а также выбора сэмплирования для определенных типов таких моделей.}
  \item {Провести эксперименты для анализа закономерностей переноса знаний в трансформер-агностичных многозадачных нейросетевых моделях между различными диалоговыми задачами. В частности, провести оценку зависимости этого переноса от размера обучающей выборки.}
  \item {Провести эксперименты для анализа закономерностей переноса знаний в многоязычных трансформер-агностичных многозадачных нейросетевых моделях между различными языками - с английского языка на русский. В частности, провести оценку зависимости этого переноса от размера обучающей выборки. Рассмотреть также применимость этих выводов для однозадачных моделей.}
  \item {Проверить пригодность русскоязычного открытого набора тематических данных для решения задачи русскоязычной тематической классификации и фундаментальных задач исследования переноса знаний на разговорных данных.}
  \item {Проверить зависимость межъязыкового переноса знаний на разговорных данных в многоязычных нейросетевых моделях от размера предобучающей выборки и генеалогической близости языков.}
  \item {Интегрировать рассмотренные в диссертации многозадачные нейросетевые архитектуры в диалоговую платформу, оценить применимость данных архитектур и провести их сравнительный анализ на основе опыта практического применения. На основании этого анализа произвести интеграцию также в open-source библиотеку.}
  \newline
  \newline
\end{enumerate}


{\novelty}
\begin{enumerate}
  \item {Создана и выложена в открытый доступ диалоговая платформа DREAM, на которй в дальнейшем может изучаться прикладное применение многозадачных нейросетевых моделей. Качество этой диалоговой платформы было проверено оценками пользователей в рамках конкурса "Alexa Prize Socialbot Grand Challenge", по результатам которых платформа DREAM вышла в полуфинал этого конкурса.}
  \item {Проверена применимость технологий, использованных в диалоговой платформе DREAM, на прикладной задаче по созданию сервиса для работы с текстами texter-ocr-cv-microservice.}
  \item {Проведены эксперименты для сравнения различных схем псевдоразметки данных для многозадачных нейросетевых моделей с одним линейным слоем на примере задач из набора данных GLUE.}
  \item {Проведены эксперименты для сравнения различных вариантов выбора архитектуры многозадачных трансформер-агностичных нейросетевых моделей, сравнения их с аналогичными однозадачными моделями для разных тел, а также выбора сэмплирования для определенных типов таких моделей.}
  \item {Проведены эксперименты для анализа закономерностей переноса знаний в трансформер-агностичных многозадачных нейросетевых моделях между различными диалоговыми задачами. В частности, была произведена оценка зависимости этого переноса от размера обучающей выборки.}
  \item {Проведены эксперименты для анализа закономерностей переноса знаний в многоязычных трансформер-агностичных многозадачных нейросетевых моделях между различными языками - с английского языка на русский. В частности, была проведена оценка зависимости этого переноса от размера обучающей выборки. Была рассмотрена также применимость этих выводов для однозадачных моделей.}
  \item {Проверена пригодность хотя бы 1 русскоязычного открытого набора тематических данных \texttt{YAQTopics} для решения задачи русскоязычной тематической классификации и фундаментальных задач исследования переноса знаний.}
  \item {Проверена зависимость межъязыкового переноса знаний на разговорных данных в многоязычных нейросетевых моделях от размера предобучающей выборки и генеалогической близости языков.}
  \item {Рассмотренные в диссертации многозадачные нейросетевые архитектуры были интегрированы в диалоговую платформу DREAM, была оценена применимость и был проведен их сравнительный анализ на основе опыта применения. На основании этого анализа была произведена также интеграция в open-source библиотеку DeepPavlov} 
\end{enumerate}

{\appropriation}
Пункты 3,4, 5, 6, 7 и 8 Научной новизны соответствуют Пункту 8 "Комплексные исследования научных и технических проблем с применением современной технологии математического моделирования и
вычислительного эксперимента." специальности 1.2.2 «Математическое моделирование, численные методы и комплексы программ». специальности 1.2.2 «Математическое моделирование, численные методы и комплексы программ», так как предложенный набор данных может использовать для валидации различных моделей машинного обучения. Пункты 1,2 и 9 Научной новизны соответствует пункту 6 "Разработка систем компьютерного и имитационного моделирования, алгоритмов и методов имитационного моделирования на основе анализа математических моделей" специальности 1.2.2 «Математическое моделирование, численные методы и комплексы программ».

{\defpositions}
\begin{enumerate}
  \item {Диалоговая платформа \texttt{DREAM} пригодна для изучения прикладного применения многозадачных нейросетевых моделей. Оценки пользователей в рамках международного конкурса "Alexa Prize Socialbot Grand Challenge", обеспечившие двукратный выход в полуфинал этой платформы, показывают высокое качество диалоговой платформы на момент её создания.}
  \item {На примере прикладной задачи по созданию сервиса для работы с текстами texter-ocr-cv-microservice показана применимость технологий, использованных в диалоговой платформе DREAM, за пределами этой диалоговой платформы.}
  \item {Псевдоразметка данных при помощи однозадачных моделей улучшает метрики многозадачных моделей. При этом объединение классов оправдывает себя только для задач, достаточно сильно похожих друг на друга.}
  \item {Было показано на различных наборах данных, что многозадачные трансформер-агностичные нейросетевые модели показывают себя не хуже ряда других, более сложных архитектур, а предложенный метод сэмплирования - не хуже ряда других методов сэмплирования. При этом многозадачные трансформер-агностичные модели по данным проведенным экспериментах дают среднюю просадку не более 1 процента по сравнению с однозадачными моделями. А если какие-то задачи достаточно похожи друг на друга, как например, в бенчмарке GLUE, многозадачные модели за счет таких задач в среднем даже превосходят однозадачные модели.}
  \item {Было показано, что для достаточно малых данных многозадачные трансформер-агностичных модели начинают превосходить по своей средней точности однозадачные, в особенности - за счет задач с наименьшим объемом данных.}
  \item {Было показано, что если в основе многозадачной трансформер-агностичной модели лежит многоязычный BERT, то добавление английских данных к русским при соответствующей номенклатуре классов позволяет улучшить метрики на 1-5\%. Чем меньше изначально русскоязычных данных, тем улучшение сильнее. Этот же вывод справедлив и для однозадачных моделей.}
  \item {Русскоязычный открытый набор тематических данных \texttt{YAQTopics} пригоден для решения задачи русскоязычной тематической классификации и фундаментальных задач исследования переноса знаний.}
  \item {Для многоязычных нейросетевых моделей качество переноса знаний на разные языки на тематических данных сильно коррелирует с размером предобучающей выборки для каждого языка, но при этом не коррелирует с генеалогической близостью этого языка к русскому.}
  \item {Рассмотренные многозадачные нейросетевые архитектуры пригодны для практического применения в диалоговых платформах и в рамках open-source библиотек. При этом предложенные автором трансформер-агностичные нейросетевые модели выигрывают у моделей типа PAL-BERT за счет трансформер-агностичности, а у моделей с одним линейным слоем - за счёт большей гибкости, отсутствия необходимости в псевдоразметке и как следствие - меньшей склонности к переобучению.}
\end{enumerate}


\iffalse
Направления исследований 1.2.2:
1. Разработка новых математических методов моделирования объектов и
явлений (физико-математические науки).
2. Разработка, обоснование и тестирование эффективных вычислительных
методов с применением современных компьютерных технологий.
3. Реализация эффективных численных методов и алгоритмов в виде
комплексов проблемно-ориентированных программ для проведения
вычислительного эксперимента.
4. Разработка новых математических методов и алгоритмов интерпретации
натурного эксперимента на основе его математической модели.
5. Разработка новых математических методов и алгоритмов валидации
математических моделей объектов на основе данных натурного эксперимента
или на основе анализа математических моделей.
6. Разработка систем компьютерного и имитационного моделирования,
алгоритмов и методов имитационного моделирования на основе анализа
математических моделей (технические науки).
7. Качественные или аналитические методы исследования математических
моделей (технические науки).
8. Комплексные исследования научных и технических проблем с
применением современной технологии математического моделирования и
вычислительного эксперимента.
9. Постановка и проведение численных экспериментов, статистический
анализ их результатов, в том числе с применением современных
компьютерных технологий (технические науки).
\fi


{\influence}
Практическая значимость заключается в следующем. Впервые в России была разработана диалоговая платформа мирового уровня, вышедшая в полуфинал престижных мировых конкурсов Alexa Prize 3 и Alexa Prize 4 (в конкурсах было 10 и 9 участников соответственно, из более чем 300 кандидатов). Эта диалоговая платформы имеет полностью открытый код, что дает возможность легкого переиспользования любой части проделанной над ней работы. 

В этой платформе, в числе всего прочего, применялись многозадачные нейросетевые модели, описанные автором в данном работе - многозадачная нейросетевая модель с одним линейным слоем, многозадачная нейросетевая модель на основе архитектуры PAL-BERT и многозадачная трансформер-агностичная нейросетевая модель. Был проведен ряд других прикладных работ для улучшения данной платформы, включающий в себя разработку сценарных навыков.

Помимо этого, программный код для реализации многозадачной трансформер-агностичной нейросетевой модели встроен в библиотеку DeepPavlov, имеющую более 500 000 скачиваний на март 2023 года.
Алгоритмы, использованные в данной работе, применены также в программе для ЭВМ [А6], на которую получено свидетельство о государственной регистрации.


{\methods}
В данной работе были
применены:
\begin{enumerate}
\item Метод численного эксперимента для исследования задач обработки естественного языка;
\item Основы теории вероятностей;
\item Методы машинного обучения и теории глубокого обучения;
\item Методы разработки на языках Python, Bash.
\end{enumerate}




{\reliability} полученных результатов обеспечивается экспериментами на наборах диалоговых данных и наборе данных GLUE, описанными в~\cite{pseudolabel}(индексируется в Scopus), ~\cite{Болотин_Карпов_Рашков_Шкурак_2019}, ~\cite{rumtl},~\cite{enmtl},~\cite{rutopics},~\cite{dp_2023}, применением в соревнованиях "Alexa Prize Challenge 3" и "Alexa Prize Challenge 4", описанным в ~\cite{dream1}, ~\cite{dream2}, ~\cite{dream1_trudy}(входит в список ВАК), а также использованием результатов работы в диалоговой платформе \texttt{DREAM} и библиотеке DeepPavlov. Результаты находятся в соответствии с результатами, полученными другими авторами.


{\probation}
Апробация работы. Результаты работы были представлены автором на конференции Диалог-2021~\cite{pseudolabel}, и опубликованы в журналах Proceedings of En\&T 2018~\cite{Болотин_Карпов_Рашков_Шкурак_2019}, Computational Linguistics and Information Technologies~\cite{pseudolabel,rumtl},Proceedings of Alexa Prize 3~\cite{dream1},Proceedings of Alexa Prize 4~\cite{dream2}, Труды МФТИ~\cite{dream1_trudy}, Proceedings of AINL 2023~\cite{rutopics}, Proceedings of InterSpeech 2023~\cite{enmtl}, Proceedings of ACL Systems Workshop~\cite{dp_topics}. Помимо этого, разработки, описанные в данной диссертации, были внедрены в находящуюся в открытом доступе диалоговую платформу \texttt{DREAM}, активно используемую в конкурсах Alexa Prize 3, Alexa Prize 4 и после них. Также на разработки, основанные на результатах работы, было получено свидетельство о регистрации ПО~\cite{Дуплякин_Дмитрий_Ондар_Ушаков_2021}.


{\contribution} В работе~\cite{Болотин_Карпов_Рашков_Шкурак_2019} автор отвечал за ряд важных компонент диалоговой системы, включающих в себя парафразер. Исследование, разработка и сравнительный анализ методов псевдоразметки данных, описанных в работе~\cite{pseudolabel}, были выполнены автором самостоятельно. В работах~\cite{dream1,dream1_trudy,dream2} автор отвечал за ряд важных компонент диалоговой системы - навыки обсуждения книг, эмоций, коронавируса, классификаторы эмоций,интентов,момента остановки диалога, TF-IDF Retrieval, Grounding Skill, Gossip Skill, генеративный навык и многозадачную нейросетевую модель. Технические решения для работы с естественным языком, использованные в \cite{dream1,dream1_trudy,dream2}, использовались также в~\cite{Дуплякин_Дмитрий_Ондар_Ушаков_2021}. В работах~\cite{rumtl,enmtl,rutopics} все исследования также были выполнены автором самостоятельно. В работе~\cite{dp_2023} автор отвечал за эксперименты с многозадачными трансформер-агностичными моделями для библиотеки DeepPavlov и их описание.

\ifnumequal{\value{bibliosel}}{0}
{%%% Встроенная реализация с загрузкой файла через движок bibtex8. (При желании, внутри можно использовать обычные ссылки, наподобие `\cite{vakbib1,vakbib2}`).
    {\publications} 

}%
{%%% Реализация пакетом biblatex через движок biber
    \begin{refsection}[bl-author]
        % Это refsection=1.
        % Процитированные здесь работы:
        %  * подсчитываются, для автоматического составления фразы "Основные результаты ..."
        %  * попадают в авторскую библиографию, при usefootcite==0 и стиле `\insertbiblioauthor` или `\insertbiblioauthorgrouped`
        %  * нумеруются там в зависимости от порядка команд `\printbibliography` в этом разделе.
        %  * при использовании `\insertbiblioauthorgrouped`, порядок команд `\printbibliography` в нём должен быть тем же (см. biblio/biblatex.tex)
        %
        % Невидимый библиографический список для подсчёта количества публикаций:
        \printbibliography[heading=nobibheading, section=1, env=countauthorvak,          keyword=biblioauthorvak]%
        \printbibliography[heading=nobibheading, section=1, env=countauthorwos,          keyword=biblioauthorwos]%
        \printbibliography[heading=nobibheading, section=1, env=countauthorscopus,       keyword=biblioauthorscopus]%
        \printbibliography[heading=nobibheading, section=1, env=countauthorconf,         keyword=biblioauthorconf]%
        \printbibliography[heading=nobibheading, section=1, env=countauthorother,        keyword=biblioauthorother]%
        \printbibliography[heading=nobibheading, section=1, env=countauthor,             keyword=biblioauthor]%
        \printbibliography[heading=nobibheading, section=1, env=countauthorvakscopuswos, filter=vakscopuswos]%
        \printbibliography[heading=nobibheading, section=1, env=countauthorscopuswos,    filter=scopuswos]%
        %
        \nocite{*}%
        %
        {\publications} Основные результаты по теме диссертации изложены в~\arabic{citeauthor}~печатных изданиях,
        \arabic{citeauthorvak} из которых изданы в журналах, рекомендованных ВАК\sloppy%
        \ifnum \value{citeauthorscopuswos}>0%
            , \arabic{citeauthorscopuswos} "--- в~периодических научных журналах, индексируемых Web of~Science и Scopus\sloppy%
        \fi%
        \ifnum \value{citeauthorconf}>0%
            , \arabic{citeauthorconf} "--- в~тезисах докладов.
        \else%
            .
        \fi
    \end{refsection}%
    \begin{refsection}[bl-author]
        % Это refsection=2.
        % Процитированные здесь работы:
        %  * попадают в авторскую библиографию, при usefootcite==0 и стиле `\insertbiblioauthorimportant`.
        %  * ни на что не влияют в противном случае
        \nocite{Болотин_Карпов_Рашков_Шкурак_2019}%vak
        \nocite{dream1}%vak
        \nocite{dream2}%vak
        \nocite{pseudolabel}
        \nocite{dream1_trudy}%vak
        \nocite{Дуплякин_Дмитрий_Ондар_Ушаков_2021}%vak
        \nocite{rumtl}
        \nocite{rutopics}
        \nocite{enmtl}
        \nocite{dp_2023}
    \end{refsection}%
        %
        % Всё, что вне этих двух refsection, это refsection=0,
        %  * для диссертации - это нормальные ссылки, попадающие в обычную библиографию
        %  * для автореферата:
        %     * при usefootcite==0, ссылка корректно сработает только для источника из `external.bib`. Для своих работ --- напечатает "[0]" (и даже Warning не вылезет).
        %     * при usefootcite==1, ссылка сработает нормально. В авторской библиографии будут только процитированные в refsection=0 работы.
        %
        % Невидимый библиографический список для подсчёта количества внешних публикаций
        % Используется, чтобы убрать приставку "А" у работ автора, если в автореферате нет
        % цитирований внешних источников.
        % Замедляет компиляцию
    \ifsynopsis
    \ifnumequal{\value{draft}}{0}{
      \printbibliography[heading=nobibheading, section=0, env=countexternal,          keyword=biblioexternal]%
    }{}
    \fi
}
\iffalse
При использовании пакета \verb!biblatex! будут подсчитаны все работы, добавленные
в файл \verb!biblio/author.bib!. Для правильного подсчёта работ в~различных
системах цитирования требуется использовать поля:
\begin{itemize}
        \item \texttt{authorvak} если публикация индексирована ВАК,
        \item \texttt{authorscopus} если публикация индексирована Scopus,
        \item \texttt{authorwos} если публикация индексирована Web of Science,
        \item \texttt{authorconf} для докладов конференций,
        \item \texttt{authorother} для других публикаций.
\end{itemize}
Для подсчёта используются счётчики:
\begin{itemize}
        \item \texttt{citeauthorvak} для работ, индексируемых ВАК,
        \item \texttt{citeauthorscopus} для работ, индексируемых Scopus,
        \item \texttt{citeauthorwos} для работ, индексируемых Web of Science,
        \item \texttt{citeauthorvakscopuswos} для работ, индексируемых одной из трёх баз,
        \item \texttt{citeauthorscopuswos} для работ, индексируемых Scopus или Web of~Science,
        \item \texttt{citeauthorconf} для докладов на конференциях,
        \item \texttt{citeauthorother} для остальных работ,
        \item \texttt{citeauthor} для суммарного количества работ.
\end{itemize}
% Счётчик \texttt{citeexternal} используется для подсчёта процитированных публикаций.

Для добавления в список публикаций автора работ, которые не были процитированы в
автореферате требуется их~перечислить с использованием команды \verb!\nocite! в
\verb!Synopsis/content.tex!.
\fi
 % Характеристика работы по структуре во введении и в автореферате не отличается (ГОСТ Р 7.0.11, пункты 5.3.1 и 9.2.1), потому её загружаем из одного и того же внешнего файла, предварительно задав форму выделения некоторым параметрам

%Диссертационная работа была выполнена при поддержке грантов \dots

%\underline{\textbf{Объем и структура работы.}} Диссертация состоит из~введения,
%четырех глав, заключения и~приложения. Полный объем диссертации
%\textbf{ХХХ}~страниц текста с~\textbf{ХХ}~рисунками и~5~таблицами. Список
%литературы содержит \textbf{ХХX}~наименование.

\section*{Содержание работы}
Во \underline{\textbf{введении}} обосновывается актуальность
исследований, проводимых в~рамках данной диссертационной работы,
приводится обзор научной литературы по изучаемой проблеме,
формулируется цель, ставятся задачи работы, излагается научная новизна
и практическая значимость представляемой работы. 

Во \underline{\textbf{второй главе}} разбираются нейросетевые архитектуры, имеющие отношение к данной диссертационной работе. Это архитектура Трансформер~\cite{vaswani_2017}, и это нейросетевая модель \textbf{BERT}~\cite{devlin_2018}, основанная на данной архитектуре.

\underline{\textbf{Третья глава}} посвящена многозадачным нейросетевым моделям. В первом разделе третьей главы дается определение многозадачного обучения, и дается деление всех имеющихся многозадачных архитектур на четыре типа - параллельные, модульные, иерархические и генеративно-состязательные. В следующих двух разделах приводится подробный обзор двух архитектур, на которых основывались последующие разделы данной диссертационной работы. А именно, модель \textbf{MT-DNN}~\cite{mtdnn}, имеющая параллельную архитектуру, и модель \textbf{PAL-BERT}~\cite{stickland_2019}, имеющая модульную архитектуру. 

\underline{\textbf{Четвертая глава}} посвящена обзору диалоговой платформы DREAM. Именно потребности этой платформы стимулировали создание многозадачных нейросетевых моделей, описанных в данной диссертационной работе, в этой платформе они и получали свое прикладное применение. В главе подробно рассмотрена структура этой диалоговой платформы, ее эволюция в течение конкурсов "Alexa Prize Socialbot Grand Challenge 3" и "Alexa Prize Socialbot Grand Challenge 4". Отдельное внимание уделено личному вкладу автора данной диссертационной работы в каждую из версий диалоговой платформы DREAM. 

Вкладом в первую версию являются аннотаторы Emotion Classification и Dialog Termination, навыки открытого домена TF-IDF Retrieval и генеративный навык, а также навыки закрытого домена Book Skill, Coronavirus Skill и (в своей основной части) Emotion Skill, которые были разработаны автором данной диссертационной работы самостоятельно. Первая версия диалоговой платформы DREAM описана в работах~\cite{dream1,dream1_trudy}.

К вкладу автора во вторую версию относится в первую очередь обучение и интеграция многозадачной нейросетевой модели с одним линейным слоем. Эти работы подробно описаны в \textbf{восьмой главе}.  Также к вкладу автора относятся обучение модели классификации интентов на основе семантических классов из набора данных MIDAS~\cite{midas}, разработка сценарного навыка Grounding Skill, значительное участие в разработке навыка закрытого домена Gossip, значительное улучшение сценарных навыков Book Skill(на основе Wikidata), Emotion Skill и аннотатора Intent Catcher. Помимо этого, автор принимал активное участие в отладке других навыков и аннотаторов на основе ежедневного анализа диалогов системы DREAM и ее личного тестирования. 

Вторая версия диалоговой системы DREAM описана в работе~\cite{dream2}. Технические решения, применяемые при работе над второй версии диалоговой системы DREAM, использовались также в сторонней работе автора, на которую получено свидетельство о депонировании ~\cite{Дуплякин_Дмитрий_Ондар_Ушаков_2021}. 


\underline{\textbf{Пятая глава}} посвящена многозадачным нейросетевым моделям с одним линейным слоем - простейшему типу многозадачных моделей. Для четырех задач из бенчмарка GLUE~\cite{wang_2018}(MNLI, QQP, SST, RTE) сравниваются различные способы псевдоразметки данных при обучении многозадачной модели с одним линейным слоем. Делается вывод, что если задачи достаточно сильно отличаются друг от друга, то самый лучший способ обучения таких моделей - получение вероятностей для каждого примера, не имеющего метки для той или иной задачи, по предсказаниям модели, учившейся только для этой задачи. Именно этот метод подробнее описан в \textbf{восьмой главе}.

\underline{\textbf{Шестая глава}} посвящена трансформер-агностичным многозадачным моделям. В частности, в этой главе приводится архитектура трансформер-агностичной многозадачной модели. 

Была 

Глава 6. Трансформер-агностичные модели . . . . . . . . . . . . . 69
6.0.1 Архитектура трансформер-агностичной многозадачной
модели . . . . . . . . . . . . . . . . . . . . . . . . . . . . . 69
6.0.2 Какие эксперименты не сработали . . . . . . . . . . . . . . 71
6.0.3 Преимущество трансформер-агностичной многозадачной
модели над многозадачной моделью с одним линейным
слоем . . . . . . . . . . . . . . . . . . . . . . . . . . . . . . 73

6.1 Наборы данных . . . . . . . . . . . . . . . . . . . . . . . . . . . . . 74
6.1.1 Классификация эмоций . . . . . . . . . . . . . . . . . . . . 75
6.1.2 Классификация тональности . . . . . . . . . . . . . . . . . 76
6.1.3 Классификация токсичности . . . . . . . . . . . . . . . . . 77
6.1.4 Классификация интентов и тематическая классификация 78
6.2 Настройки экспериментов . . . . . . . . . . . . . . . . . . . . . . . 78
6.3 Многозадачные и однозадачные модели - эксперименты на
полном наборе данных . . . . . . . . . . . . . . . . . . . . . . . . . 79
6.3.1 Эффект уменьшения размера обучающей выборки
(англоязычные данные) . . . . . . . . . . . . . . . . . . . . 81
6.3.2 Многоязычные многозадачные модели - эффект
кросс-языкового обучения . . . . . . . . . . . . . . . . . . . 83
6.3.3 Насколько помогает добавление англоязычных данных? . 85
6.4 Выводы и анализ результатов . . . . . . . . . . . . . . . . . . . . 86
\underline{\textbf{Седьмая глава}} расширяет работу, проделанную в \textbf{шестой главе}. В данной главе предлагается новый тематический набор данных - \texttt{YAQTopics}.  Данный набор состоит из 76 тематических классов и имеет более 500 
Глава 7. Исследование переноса знаний в многоязычных
моделях на новом тематическом наборе данных . . . . . 7 МОЖЕТ ПОМЕНЯТЬСЯ
7.1 Введение . . . . . . . . . . . . . . . . . . . . . . . . . . . . . . . . 7
7.2 Набор данных YAQTopics . . . . . . . . . . . . . . . . . . . . . . . 8
7.3 Выбор представления набора данных YAQTopics . . . . . . . . . . 11
7.3.1 Обучение модели для сравнения . . . . . . . . . . . . . . . 11
7.4 Перенос знаний между языками . . . . . . . . . . . . . . . . . . . 15
7.5 Обсуждение и анализ результатов .

\fi

\underline{\textbf{Восьмая глава}}
Глава 8. Использование в диалоговой платформе DREAM
многозадачных моделей . . . . . . . . . . . . . . . . . . . . . 91
8.1 Использование многозадачных моделей с одним линейным слоем 91
8.1.1 Использование многозадачных моделей с одним
линейным слоем для объединения и замены
классификаторов реплик . . . . . . . . . . . . . . . . . . . 91
8.1.2 Использование многозадачных моделей с одним
линейным слоем для замены модели для оценки диалога . 94
8.2 Использование модели PAL-BERT в диалоговой платформе
DREAM . . . . . . . . . . . . . . . . . . . . . . . . . . . . . . . . . 95
8.3 Выводы . . . . . . . . . . . . . . . . . . . . . . . . . . . . . . . . . 100
8.4 Использование многозадачной трансформер-агностичной
модели в диалоговой платформе DREAM . . . . . . . . . . . . . . 101
8.4.1 Экономия памяти GPU, CPU и быстродействия . . . . . . 1

Можно сослаться на свои работы в автореферате. Для этого в файле
\verb!Synopsis/setup.tex! необходимо присвоить положительное значение
счётчику \verb!\setcounter{usefootcite}{1}!. В таком случае ссылки на
работы других авторов будут подстрочными.
Изложенные в третьей главе результаты опубликованы в~\cite{vakbib1, vakbib2}.
Использование подстрочных ссылок внутри таблиц может вызывать проблемы.

В \underline{\textbf{четвертой главе}} приведено описание

В \underline{\textbf{заключении}} приведены основные результаты работы, которые заключаются в следующем:
%% Согласно ГОСТ Р 7.0.11-2011:
%% 5.3.3 В заключении диссертации излагают итоги выполненного исследования, рекомендации, перспективы дальнейшей разработки темы.
%% 9.2.3 В заключении автореферата диссертации излагают итоги данного исследования, рекомендации и перспективы дальнейшей разработки темы.
%ПРОСТО СКОПИРОВАЛ ПОЛОЖЕНИЯ ВЫНОСИМЫЕ ДЛЯ ЗАЩИТЫ
\begin{enumerate}
  \item {Созданная при существенном вкладе автора платформа DREAM, находящаяся в открытом доступе, пригодна для изучения прикладного применения многозадачных нейросетевых моделей.}
  %\item {На примере прикладной задачи по созданию сервиса для работы с текстами texter-ocr-cv-microservice показана применимость технологий, использованных в диалоговой платформе DREAM, за пределами этой диалоговой платформы.}
  \item {Псевдоразметка данных при помощи однозадачных моделей улучшает метрики многозадачных моделей. При этом объединение классов оправдывает себя только для задач, достаточно сильно похожих друг на друга.}
  \item {Многозадачные трансформер-агностичные нейросетевые модели показывают себя не хуже ряда других, более сложных архитектур, а предложенный метод сэмплирования - не хуже ряда других методов сэмплирования. При этом многозадачные трансформер-агностичные модели по данным проведенных экспериментов дают среднюю просадку не более 1 процента по сравнению с однозадачными моделями. При достаточной степени похожести задач друг на друга модели за счет таких задач в среднем даже превосходят однозадачные модели.}
  \item {Для достаточно малых данных многозадачные трансформер-агностичных модели начинают превосходить по своей средней точности однозадачные, в особенности - за счет задач с наименьшим объемом данных.}
  \item {Если в основе многозадачной трансформер-агностичной модели лежит многоязычный BERT, то добавление английских данных к русским при соответствующей номенклатуре классов позволяет улучшить метрики на 1-5\%. Чем меньше изначально русскоязычных данных, тем улучшение сильнее. Этот же вывод справедлив и для однозадачных моделей.}
  \item {Русскоязычные модели, обученные на 6 классах из русскоязычного набора тематических данных YAQTopics, показывают точность около 85\% на наборе данных MASSIVE. Это оправдывает использование набора данных YAQTopics для решения задачи русскоязычной тематической классификации и фундаментальных задач исследования переноса знаний.}
  \item {Для многоязычных нейросетевых моделей качество переноса знаний на разные языки на тематических данных сильно коррелирует с размером предобучающей выборки для каждого языка, но при этом не коррелирует с генеалогической близостью этого языка к языку дообучения.}
  \item {Рассмотренные многозадачные нейросетевые архитектуры пригодны для практического применения в диалоговых платформах и в рамках open-source библиотек. При этом предложенные автором трансформер-агностичные нейросетевые модели выигрывают у моделей типа PAL-BERT за счет трансформер-агностичности, а у моделей с одним линейным слоем - за счёт большей гибкости, отсутствия необходимости в псевдоразметке и как следствие - меньшей склонности к переобучению.}
\end{enumerate}


При использовании пакета \verb!biblatex! список публикаций автора по теме
диссертации формируется в разделе <<\publications>>\ файла
\verb!common/characteristic.tex!  при помощи команды \verb!\nocite!

\ifdefmacro{\microtypesetup}{\microtypesetup{protrusion=false}}{} % не рекомендуется применять пакет микротипографики к автоматически генерируемому списку литературы
\urlstyle{rm}                               % ссылки URL обычным шрифтом
\ifnumequal{\value{bibliosel}}{0}{% Встроенная реализация с загрузкой файла через движок bibtex8
  \renewcommand{\bibname}{\large \bibtitleauthor}
  \nocite{*}
  \insertbiblioauthor           % Подключаем Bib-базы
  %\insertbiblioexternal   % !!! bibtex не умеет работать с несколькими библиографиями !!!
}{% Реализация пакетом biblatex через движок biber
  % Цитирования.
  %  * Порядок перечисления определяет порядок в библиографии (только внутри подраздела, если `\insertbiblioauthorgrouped`).
  %  * Если не соблюдать порядок "как для \printbibliography", нумерация в `\insertbiblioauthor` будет кривой.
  %  * Если цитировать каждый источник отдельной командой --- найти некоторые ошибки будет проще.
  %
  %% authorvak
  \nocite{vakbib1}%
  \nocite{vakbib2}%
  %
  %% authorwos
  \nocite{wosbib1}%
  %
  %% authorscopus
  \nocite{scbib1}%
  %
  %% authorconf
  \nocite{confbib1}%
  \nocite{confbib2}%
  %
  %% authorother
  \nocite{bib1}%
  \nocite{bib2}%

  \ifnumgreater{\value{usefootcite}}{0}{
    \begin{refcontext}[labelprefix={}]
      \ifnum \value{bibgrouped}>0
        \insertbiblioauthorgrouped    % Вывод всех работ автора, сгруппированных по источникам
      \else
        \insertbiblioauthor      % Вывод всех работ автора
      \fi
    \end{refcontext}
  }{
  \ifnum \value{citeexternal}>0
    \begin{refcontext}[labelprefix=A]
      \ifnum \value{bibgrouped}>0
        \insertbiblioauthorgrouped    % Вывод всех работ автора, сгруппированных по источникам
      \else
        \insertbiblioauthor      % Вывод всех работ автора
      \fi
    \end{refcontext}
  \else
    \ifnum \value{bibgrouped}>0
      \insertbiblioauthorgrouped    % Вывод всех работ автора, сгруппированных по источникам
    \else
      \insertbiblioauthor      % Вывод всех работ автора
    \fi
  \fi
  %  \insertbiblioauthorimportant  % Вывод наиболее значимых работ автора (определяется в файле characteristic во второй section)
  \begin{refcontext}[labelprefix={}]    \insertbiblioexternal            % Вывод списка литературы, на которую ссылались в тексте автореферата
  \end{refcontext}
  }
}
\ifdefmacro{\microtypesetup}{\microtypesetup{protrusion=true}}{}
\urlstyle{tt}                               % возвращаем установки шрифта ссылок URL
