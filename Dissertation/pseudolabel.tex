\chapter{Использование псевдоразметки данных в многозадачных моделях для решения задач GLUE}\label{ch:pseudolabel}
% https://www.overleaf.com/7535984655svjbkdpsdgpc И ЦИТАТЫ И ВИД ТАБЛИЦ

Простейшей реализацией энкодер-агностичной модели для решения большого количества задач является использование модели с одним «трансформером» в качестве тела (например, \textit{bert-base-uncased}) и одним линейным слоем на все задачи. Данный вид модели максимально экономичен по потребляемой памяти, как и по технической реализации. Тем не менее, при обучении его на нескольких наборах данных возникает проблема, так как у каждого набора данных есть метки только для классов «своей» задачи. У некоторых наборов данных классы могут быть схожими, у других -- кардинально различаться. Одним из способов решения этой проблемы является псевдоразметка данных. Другим, менее универсальным способом, является объединение меток. 
В работе~\cite{pseudolabel} автором диссертации была поставлена задача -- \textbf{разработать новые методы псевдоразметки данных, получить оценку их влияния на качество многозадачных моделей и выделить тип задач, для которых объединение меток при многозадачной классификации оправдано.}}

Эксперименты, проделанные в рамках решения данной задачи, подробно описаны ниже.




\section{Описание экспериментов}\label{ch:pseudolabel/sect1}
В каждом из экспериментов архитектура модели оставалась той же самой, изменялись лишь метки классов, подаваемые на вход модели. Модель с одним линейным слоем, основанная на архитектуре \textit{bert-base-uncased}, тренировалась предсказывать вероятность от 0 до 1 для каждого класса, в многометочном режиме. Т.е каждый нейрон в финальном слое мог принимать значения от 0 до 1, при этом в финальном линейном слое разные нейроны отвечали за разные задачи.
Модель оценивалась на следующих классификационных задачах -- Multi-Genre Natural Language Inference(далее -- MNLI), Quora Question Pairs(далее -- QQP), Stanford Sentiment Treebank -- 2(далее -- SST-2) и Recognizing Textual Entailment(далее -- RTE) из набора данных GLUE~\cite{wang_2018}. 

Суть данных задач заключается в следующем:
\begin{enumerate}
    \item QQP -- задача классификации пар предложений из сайта Quora.com на 2 класса -- и не дубликат.
    \item MNLI -- задача классификации пар предложений из различных тем на три класса -- логическое следование, логическое противоречие и нейтральный(ни следование, ни противоречие). Валидационный и тестовый наборы данных в MNLI существуют в 2 вариантах -- из тех же тем, что и тренировочный(MNLI-m) и из остальных тем(MNLI-mm).
    \item SST-2 -- задача классификации предложений на два класса -- положительная тональность и отрицательная тональность.
    \item RTE --  задача классификации пар предложений на два класса -- логическое следствие и нет логического следствия.
\end{enumerate}

Первые 3 задачи были выбраны, так как их наборы данных были достаточно велики (более 50000 примеров для каждой из задач). Четвертая задача была выбрана, чтобы получить возможность исследовать объединение классов при псевдоразметке. Также для каждой из этих задач были воспроизведены результаты из вышеупомянутой статьи(заметим, что оригинальная модель не была многометочной). Примеры из всех задач были перемешаны и выбирались случайным образом. В связи с вычислительными ограничениями, все эксперименты проводились при модели-теле BERT-Base-Uncased~\cite{devlin_2018}, аналогичной оригинальной статье, с ней же шло и сравнение всех остальных экспериментов.

\section{Условные обозначения}\label{subch:pseudolabel/sect2}
   Для формул ниже будут использоваться следующие условные обозначения:
   \begin{itemize}
   \item $\color {black} +$ и $\color {black}-$ означают положительный(\textit{positive}) и отрицательный(\textit{negative}) классы в наборе данных SST-2 
   
   \item $\color {black} d$ и $\color {black}!d$означают класс «дубликат» (\textit{duplicate}) и «не дубликат» \textit{not duplicate}  в наборе данных QQP;
   
   \item $\color {black} e, c, n$: метки «логическое следствие»(\textit{entailment}), «логическое противоречие»(\textit{contradiction}) и «нейтральный»(\textit{neutral}) из набора данных MNLI;
   
   \item $\color {black}\varepsilon, !\varepsilon $: метки «логическое следствие»(\textit{entailment}) и «логическое противоречие»(\textit{not entailment}) из набора данных RTE;

   
   \item  $\color{black}\mathit{MNLIpred}, \mathit{RTEpred}, \mathit{QQPpred}, \mathit{SSTpred}$ -- предсказания модели, обученной на соответствующей задаче (MNLI, RTE, QQP, SST-2) для метки из нижнего индекса формулы. 
   
   \item  $\color{black} I$ означает округление предсказанной оригинальной моделью вероятностного вектора: самый большой элемент считается равным 1, остальные приравниваются к нулю.
   
   \item $\mathit{MNLIpred}^{!e}_{n}$,$\mathit{MNLIpred}^{!e}_{c}$ -- вероятности, предсказанные для классов «нейтральный» и «логическое противоречие» модели MNLI при условии, что вероятность класса «логическое следствие» задается равной нулю. После этого вероятности этих классов вычисляются так, как если бы классов было не 3, а всего 2.
   
     
   
   \item  $\color{black} prob^{label}_{task}$ вектор с вероятностями от 0 до 1 (1 класс -- 1 вероятность), которые нам необходимо присвоить примеру из набора данных $\color{black}{task}$,
   имеющему в этом наборе данных метку $\color{black}{label}$;
   
   \item  $\color{black} P_{label}$ вероятность P метки $\color{black} label$, где P от 0 до 1.
   
   \end{itemize}   

   Это значит, что, например:
   \begin{itemize}
   
   \item $\color{black} \mathit{MNLIpred}_e$ -- вероятность метки \textit{entailment}, предсказанной моделью, обученной на MNLI;
   
   \item $\color{black} I(\mathit{MNLIpred})_e$ равняется 1, если \textit{entailment} самый вероятный предсказанный класс для MNLI, и 0 в других случаях. 
   \end{itemize}
   
      \begin{equation}
   \color{black} \mathit{MNLIpred}^{!e}_{n} = \mathit{MNLIpred}_{n}/(\mathit{MNLIpred}_{n} + \mathit{MNLIpred}_{c}) \label{eq:ps0:0}
   \end{equation}
        \begin{equation}
   \color{black} \mathit{MNLIpred}^{!e}_{c} = \mathit{MNLIpred}_{c}/(\mathit{MNLIpred}_{n} + \mathit{MNLIpred}_{c}) \label{eq:ps0:1}
   \end{equation}

   Вероятностные вектора обозначаются квадратными скобками.

\section{Способы обучения многозадачной модели}\label{subch:pseudolabel/sect3}

Автором были рассмотрены разные способы обучения многозадачных моделей. Эти способы представлены ниже.

   \subsection{Независимые метки}\label{subch:pseudolabel/sect3/sub1}

В данном подходе, модель обучается на данных из всех задач -- RTE, MNLI, QQP и SST-2. При этом массивы меток для каждой из задач независимы: для каждого примера вероятность всех классов, кроме изначально заданного в одном из этих наборов данных, считается равной нулю.  Всего при данном подходе 9 классов -- 3 для задачи MNLI и по 2 для каждой из остальных 3 задач.
Вероятности, подаваемые на вход модели, для данного метода можно представить следующими формулами:

 \begin{equation}
		\color{black} prob^{\varepsilon}_{RTE}  = [1_{\varepsilon}, 0_{!\varepsilon},0_{e},0_{c},0_{n},0_{d}, 0_{!d},0_{+},0_{-}]  \label{eq:ps1:0}
\end{equation}
\begin{equation}
		\color{black} prob^{!\varepsilon}_{RTE} = [0_{\varepsilon}, 1_{!\varepsilon},0_{e},0_{c},0_{n},0_{d}, 0_{!d},0_{+},0_{-}] \label{eq:ps1:1}
\end{equation}
\begin{equation}
\color{black} prob^{e}_{MNLI} = [0_{\varepsilon}, 0_{!\varepsilon},1_{e},0_{c},0_{n},0_{d}, 0_{!d},0_{+},0_{-}] \label{eq:ps1:2}
\end{equation}
\begin{equation}
\color{black} prob^{c}_{MNLI} = [0_{\varepsilon}, 0_{!\varepsilon},0_{e},1_{c},0_{n},0_{d}, 0_{!d},0_{+},0_{-}] \label{eq:ps1:3}
\end{equation}
\begin{equation}
\color{black} prob^{n}_{MNLI} = [0_{\varepsilon}, 0_{!\varepsilon},0_{e},0_{c},1_{n},0_{d}, 0_{!d},0_{+},0_{-}] \label{eq:ps1:4}
\end{equation}
\begin{equation}
\color{black} prob^{d}_{QQP} = [0_{\varepsilon}, 0_{!\varepsilon},0_{e},0_{c},0_{n},1_{d}, 0_{!d},0_{+},0_{-}] \label{eq:ps1:5}
\end{equation}
\begin{equation}
\color{black} prob^{!d}_{QQP} = [0_{\varepsilon}, 0_{!\varepsilon},0_{e},0_{c},0_{n},0_{d}, 1_{!d},0_{+},0_{-}] \label{eq:ps1:6}
\end{equation}
\begin{equation}
\color{black} prob^{+}_{SST} = [0_{\varepsilon}, 0_{!\varepsilon},0_{e},0_{c},0_{n},0_{d}, 0_{!d},1_{+},0_{-}] \label{eq:ps1:7}
\end{equation}
\begin{equation}
\color{black} prob^{-}_{SST} = [0_{\varepsilon}, 0_{!\varepsilon},0_{e},0_{c},0_{n},0_{d}, 0_{!d},0_{+},1_{-}] \label{eq:ps1:8}
\end{equation}

\subsection{Мягкие независимые метки}\label{subch:pseudolabel/sect3/sub2}



Данный подход аналогичен подходу \textbf{Независимые метки} с одним отличием -- для каждого примера, обнуляется не вероятность абсолютно всех классов кроме того, что изначально задан, а только вероятность других классов той же самой задачи. Вероятности других классов других задач считаются одинаковыми, с суммой вероятностей равной 1 для каждой из задач.
Вероятности, подаваемые на вход модели, для данного метода можно представить следующими формулами:
   
\begin{equation}
\color{black} prob^{\varepsilon}_{RTE}  = [1_{\varepsilon}, 0_{!\varepsilon},1/3_{e},1/3_{c},1/3_{n},1/2_{d}, 1/2_{!d},1/2_{+},1/2_{-}]
\label{eq:ps2:0}
\end{equation}
\begin{equation}
\color{black} prob^{!\varepsilon}_{RTE} = [0_{\varepsilon}, 1_{!\varepsilon},1/3_{e},1/3_{c},1/3_{n},1/2_{d}, 1/2_{!d},1/2_{+},1/2_{-}]
\label{eq:ps2:1}
\end{equation}
\begin{equation}
\color{black} prob^{e}_{MNLI} = [1/2_{\varepsilon}, 1/2_{!\varepsilon},1_{e},0_{c},0_{n},1/2_{d}, 1/2_{!d},1/2_{+},1/2_{-}]
\label{eq:ps2:2}
\end{equation}
\begin{equation}
\color{black} prob^{c}_{MNLI} = [1/2_{\varepsilon}, 1/2_{!\varepsilon},0_{e},1_{c},0_{n},1/2_{d}, 1/2_{!d},1/2_{+},1/2_{-}]
\label{eq:ps2:3}
\end{equation}
\begin{equation}
\color{black} prob^{n}_{MNLI} = [1/2_{\varepsilon}, 1/2_{!\varepsilon},0_{e},0_{c},1_{n},1/2_{d}, 1/2_{!d},1/2_{+},1/2_{-}]
\label{eq:ps2:4}
\end{equation}
\begin{equation}
\color{black} prob^{d}_{QQP} = [1/2_{\varepsilon}, 1/2_{!\varepsilon},1/3_{e},1/3_{c},1/3_{n},1_{d}, 0_{!d},1/2_{+},1/2_{-}]
\label{eq:ps2:5}
\end{equation}
\begin{equation}
\color{black} prob^{!d}_{QQP} = [1/2_{\varepsilon}, 1/2_{!\varepsilon},1/3_{e},1/3_{c},1/3_{n},0_{d}, 1_{!d},1/2_{+},1/2_{-}]
\label{eq:ps2:6}
\end{equation}
\begin{equation}
\color{black} prob^{+}_{SST} = [1/2_{\varepsilon}, 1/2_{!\varepsilon},1/3_{e},1/3_{c},1/3_{n},1/2_{d}, 1/2_{!d},1_{+},0_{-}]
\label{eq:ps2:7}
\end{equation}
\begin{equation}
\color{black} prob^{-}_{SST} = [1/2_{\varepsilon}, 1/2_{!\varepsilon},1/3_{e},1/3_{c},1/3_{n},1/2_{d}, 1/2_{!d},0_{+},1_{-}]
\label{eq:ps2:8}
\end{equation}

\subsection{Дополненные независимые метки}\label{subch:pseudolabel/sect3/sub3}

Данный подход аналогичен подходам \textbf{Независимые метки} и \textbf{Мягкие независимые метки}, кроме одного отличия. Для каждого примера, вероятности всех классов из «не своего» набора данных не считаются одинаковыми, а определяются в соответствии с предсказаниями модели, обученной на этом «не своем» наборе данных в рамках воспроизведения оригинальной статьи.
Вероятности, подаваемые на вход модели, для данного метода можно описать следующими формулами:

\begin{align}
\color{black} prob^{\varepsilon}_{RTE}  = [1_{\varepsilon}, 0_{!\varepsilon},{\mathit{MNLIpred}}_{e},\mathit{MNLIpred}_{c},\mathit{MNLIpred}_{n},\mathit{QQPpred}_{d}, \nonumber\\
\color{black} \mathit{QQPpred}_{!d},\mathit{SSTpred}_{+},\mathit{SSTpred}_{-}]
\label{eq:ps3:0}
\end{align}
\begin{align}
\color{black} prob^{!\varepsilon}_{RTE} = [0_{\varepsilon}, 1_{!\varepsilon},{\mathit{MNLIpred}}_{e},\mathit{MNLIpred}_{c},\mathit{MNLIpred}_{n},\mathit{QQPpred}_{d},\nonumber\\ \color{black} \mathit{QQPpred}_{!d},\mathit{SSTpred}_{+},\mathit{SSTpred}_{-}]
\label{eq:ps3:1}
\end{align}
\begin{align}
\color{black} prob^{e}_{MNLI} = [\mathit{RTEpred}_{\varepsilon}, \mathit{RTEpred}_{!\varepsilon},1_{e},0_{c},0_{n},\mathit{QQPpred}_{d},\nonumber\\ \color{black} \mathit{QQPpred}_{!d},\mathit{SSTpred}_{+},\mathit{SSTpred}_{-}]
\label{eq:ps3:2}
\end{align}
\begin{align}
\color{black} prob^{c}_{MNLI} = [\mathit{RTEpred}_{\varepsilon}, \mathit{RTEpred}_{!\varepsilon},0_{e},1_{c},0_{n},\mathit{QQPpred}_{d},\nonumber\\ \color{black} \mathit{QQPpred}_{!d},\mathit{SSTpred}_{+},\mathit{SSTpred}_{-}]
\label{eq:ps3:3}
\end{align}
\begin{align}
\color{black} prob^{n}_{MNLI} = [\mathit{RTEpred}_{\varepsilon}, \mathit{RTEpred}_{!\varepsilon},0_{e},0_{c},1_{n},\mathit{QQPpred}_{d},\nonumber\\ \color{black} \mathit{QQPpred}_{!d},\mathit{SSTpred}_{+},\mathit{SSTpred}_{-}]
\label{eq:ps3:4}
\end{align}
\begin{align}
\color{black} prob^{d}_{QQP} = [\mathit{RTEpred}_{\varepsilon}, \mathit{RTEpred}_{!\varepsilon},\mathit{MNLIpred}_{e},\nonumber\\ \color{black} \mathit{MNLIpred}_{c},\mathit{MNLIpred}_{n},1_{d},0_{!d},\mathit{SSTpred}_{+},\mathit{SSTpred}_{-}]
\label{eq:ps3:5}
\end{align}
\begin{align}
\color{black} prob^{!d}_{QQP} = [\mathit{RTEpred}_{\varepsilon}, \mathit{RTEpred}_{!\varepsilon},\mathit{MNLIpred}_{e},\mathit{MNLIpred}_{c},\nonumber\\ \color{black} \mathit{MNLIpred}_{n},0_{d},1_{!d},\mathit{SSTpred}_{+},\mathit{SSTpred}_{-}]
\label{eq:ps3:6}
\end{align}
\begin{align}
\color{black} prob^{+}_{SST} = [\mathit{RTEpred}_{\varepsilon}, \mathit{RTEpred}_{!\varepsilon},\mathit{MNLIpred}_{e},\mathit{MNLIpred}_{c},\nonumber\\ \color{black} \mathit{MNLIpred}_{n},\mathit{QQPpred}_{d},\mathit{QQPpred}_{!d},1_{+},0_{-}]
\label{eq:ps3:7}
\end{align}
\begin{align}
\color{black} prob^{-}_{SST} = [\mathit{RTEpred}_{\varepsilon}, \mathit{RTEpred}_{!\varepsilon},\mathit{MNLIpred}_{e},\mathit{MNLIpred}_{c},\nonumber\\ \color{black} \mathit{MNLIpred}_{n},\mathit{QQPpred}_{d},\mathit{QQPpred}_{!d},0_{+},1_{-}]
\label{eq:ps3:8}
\end{align}

\subsection{Мягкое вероятностное предположение}\label{subch:pseudolabel/sect3/sub4}

В данном эксперименте, как и в экспериментах \textbf{Независимые метки}, \textbf{Мягкие независимые метки} и \textbf{Дополненные независимые метки}, модель была обучена на данных из всех четырех задач -- RTE, MNLI, QQP и SST-2. Но при этом метки для этих задач считались зависимыми. А именно, количество классов для модели было сокращено до 5: положительный(positive), негативный(negative), логическое следствие(entailment), логическое противоречие(contradiction), нейтральный(neutral). Набор классов из этих наборов данных был переведен в вероятности этих 5 классов в соответствие со следующими правилами:
\begin{itemize}
\item Метки из всех наборов данных, кроме SST-2, считаются на 50\% положительными и на 50\% отрицательными.
\item Метки из набора данных SST-2 считаются имеющими классы «логическое следствие», «логическое противоречие» и «нейтральный» с вероятностью $\frac{1}{3}$ каждый. Им назначаются вероятности классов«положительный» и«отрицательный» в соответствии с оригинальным набором данных.
\item Меткам из набора данных MNLI назначаются вероятности классов «логическое следствие», «логическое противоречие» и «нейтральный» в соответствии с оригинальным набором данных.
\item Меткам из набора данных QQP, если у них класс «дубликат», назначается вероятность класса «логическое следствие», равная 1 и вероятности классов «логическое противоречие» и «нейтральный», равные 0. Иначе вероятность класса «логическое следствие» считается равной 0, а вероятности классов «логическое противоречие» и «нейтральный» считаются равными по 0.5 каждая.
\item Меткам из набора данных RTE, если у них класс «логическое следствие», тоже назначается вероятность класса «логическое следствие», равная 1 и вероятности классов «логическое противоречие» и «нейтральный», равная 0. Иначе вероятность класса «логическое следствие» считается равной 0, а вероятности классов «логическое противоречие» и «нейтральный» считаются равными по 0.5 каждая.
\end{itemize}

Вероятности, подаваемые на вход модели, для данного метода можно представить следующими формулами:

\begin{equation}
\color{black} prob^{\varepsilon}_{RTE}  = [{1}_{e}, {0}_{c},{0}_{n},{1/2}_{+},{1/2}_{-}]
\label{eq:ps4:0}
\end{equation}
\begin{equation}
\color{black} prob^{!\varepsilon}_{RTE}  = [{0}_{e}, {1/2}_{c},{1/2}_{n},{1/2}_{+},{1/2}_{-}]
\label{eq:ps4:1}
\end{equation}
\begin{equation}
\color{black} prob^{e}_{MNLI}  = [{1}_{e}, {0}_{c},{0}_{n},{1/2}_{+},{1/2}_{-}]
\label{eq:ps4:2}
\end{equation}
\begin{equation}
\color{black} prob^{c}_{MNLI}  = [{0}_{e}, {1}_{c},{0}_{n},{1/2}_{+},{1/2}_{-}]
\label{eq:ps4:3}
\end{equation}
\begin{equation}
\color{black} prob^{n}_{MNLI}  = [{0}_{e}, {0}_{c},{1}_{n},{1/2}_{+},{1/2}_{-}]
\label{eq:ps4:4}
\end{equation}
\begin{equation}
\color{black} prob^{d}_{QQP}  = [{1}_{e}, {0}_{c},{0}_{n},{1/2}_{+},{1/2}_{-}]
\label{eq:ps4:5}
\end{equation}
\begin{equation}
\color{black} prob^{!d}_{QQP}  = [{0}_{e}, {1/2}_{c},{1/2}_{n},{1/2}_{+},{1/2}_{-}]
\label{eq:ps4:6}
\end{equation}
\begin{equation}
\color{black} prob^{+}_{SST} = [{1/3}_{e}, {1/3}_{c},{1/3}_{n},{1}_{+},{0}_{-}]
\label{eq:ps4:7}
\end{equation}
\begin{equation}
\color{black} prob^{-}_{SST} = [{1/3}_{e}, {1/3}_{c},{1/3}_{n},{0}_{+},{1}_{-}]
\label{eq:ps4:8}
\end{equation}

\subsection{Мягкие предсказанные метки}\label{subch:pseudolabel/sect3/sub5}

Данный подход аналогичен подходу \textbf{Мягкое вероятностное предположение} с одним отличием. Все недостающие вероятности для каждой из задач не считаются равновероятными, а определяются дополнительной разметкой от модели для каждой задачи, а именно:
\begin{itemize}
\item Если пример не из набора данных SST-2, вероятность положительного и отрицательного классов определяется по предсказаниям модели, обученной на наборе данных SST-2. Иначе, как и в предыдущем подходе, эти вероятности берутся из оригинального набора данных.
\item Меткам из набора данных MNLI, как и в предыдущем подходе, назначаются вероятности классов «логическое следствие», «логическое противоречие» и «нейтральный» в соответствии с оригинальным набором данных.
\item Меткам из набора данных QQP, если у них класс «дубликат», как и в предыдущем подходе, назначается вероятность класса «логическое следствие», равная 1 и вероятности классов «логическое противоречие» и «нейтральный», равные 0. Иначе вероятность класса «логическое следствие» считается равной 0, а вероятности классов «логическое противоречие» и «нейтральный» определяются по предсказаниям модели, обученной на наборе данных MNLI, нормализованных на сумму предсказанных вероятностей этих 2 классов.
\item Меткам из набора данных RTE, если у них класс «логическое следствие», как и в предыдущем подходе, назначается вероятность класса «логическое следствие», равная 1 и вероятности классов «логическое противоречие» и «нейтральный», равная 0.  Иначе вероятность класса «логическое следствие» считается равной 0, а вероятности классов «логическое противоречие» и «нейтральный» определяются по предсказаниям модели, обученной на наборе данных MNLI, нормализованных на сумму предсказанных вероятностей этих 2 классов.
\end{itemize}


Вероятности, подаваемые на вход модели, для данного метода можно описать следующими формулами:



\begin{equation}
\color{black} prob^{\varepsilon}_{RTE}  = [{1}_{e}, {0}_{c},{0}_{n},{\mathit{SSTpred}}_{+},{\mathit{SSTpred}}_{-}]
\label{eq:ps5:0}
\end{equation}
\begin{equation}
\color{black} prob^{!\varepsilon}_{RTE}  = [{0}_{e}, \mathit{MNLIpred}^{!e}_{c},\mathit{MNLIpred}^{!e}_{n},{\mathit{SSTpred}}_{+},{\mathit{SSTpred}}_{-}]
\label{eq:ps5:1}
\end{equation}
\begin{equation}
\color{black} prob^{e}_{MNLI}  = [{1}_{e}, {0}_{c},{0}_{n},{\mathit{SSTpred}}_{+},{\mathit{SSTpred}}_{-}]
\label{eq:ps5:2}
\end{equation}
\begin{equation}
\color{black} prob^{c}_{MNLI}  = [{0}_{e}, {1}_{c},{0}_{n},{\mathit{SSTpred}}_{+},{\mathit{SSTpred}}_{-}]
\label{eq:ps5:3}
\end{equation}
\begin{equation}
\color{black} prob^{n}_{MNLI}  = [{0}_{e}, {0}_{c},{1}_{n},{\mathit{SSTpred}}_{+},{\mathit{SSTpred}}_{-}]
\label{eq:ps5:4}
\end{equation}
\begin{equation}
\color{black} prob^{d}_{QQP}  = [{1}_{e}, {0}_{c},{0}_{n},{\mathit{SSTpred}}_{+},{\mathit{SSTpred}}_{-}]
\label{eq:ps5:5}
\end{equation}
\begin{equation}
\color{black} prob^{!d}_{QQP}  = [{0}_{e}, \mathit{MNLIpred}^{!e}_{c},\mathit{MNLIpred}^{!e}_{n},{\mathit{SSTpred}}_{+},{\mathit{SSTpred}}_{-}]
\label{eq:ps5:6}
\end{equation}
\begin{equation}
\color{black} prob^{+}_{SST} = [\mathit{MNLIpred}_{e}, \mathit{MNLIpred}_{c},\mathit{MNLIpred}_{n},{1}_{+},{0}_{-}]
\label{eq:ps5:7}
\end{equation}
\begin{equation}
\color{black} prob^{-}_{SST} = [\mathit{MNLIpred}_{e}, \mathit{MNLIpred}_{c},\mathit{MNLIpred}_{n},{0}_{+},{1}_{-}]
\label{eq:ps5:8}
\end{equation}

\subsection{Жесткие предсказанные метки}\label{subch:pseudolabel/sect3/sub6}

Данный подход аналогичен подходу \textbf{Мягкие предсказанные метки} с одним изменением. Для меток, полученных из предсказаний оригинальной модели, максимальная вероятность для каждой задачи округляется до 1, а все остальные вероятности до 0.
Вероятности, подаваемые на вход модели, для данного метода можно описать следующими формулами:
\begin{equation}
\color{black} prob^{\varepsilon}_{RTE}  = [{1}_{e}, {0}_{c},{0}_{n},
\color{black}I({\mathit{SSTpred}}_{+}),{I(\mathit{SSTpred}}_{-})]
\label{eq:ps6:0}
\end{equation}
\begin{equation}
\color{black} prob^{!\varepsilon}_{RTE}  = [{0}_{e}, \mathit{MNLIpred}^{!e}_{c},\mathit{MNLIpred}^{!e}_{n},\nonumber\\\color{black}I({\mathit{SSTpred}}_{+}),{Id(\mathit{SSTpred}}_{-})]
\label{eq:ps6:1}
\end{equation}
\begin{equation}
\color{black} prob^{e}_{MNLI}  = [{1}_{e}, {0}_{c},{0}_{n},I({\mathit{SSTpred}}_{+}),{I\mathit{SSTpred}}_{-})]
\label{eq:ps6:2}
\end{equation}
\begin{equation}
\color{black} prob^{c}_{MNLI}  = [{0}_{e}, {1}_{c},{0}_{n},I({\mathit{SSTpred}}_{+}),{I(\mathit{SSTpred}}_{-})]
\label{eq:ps6:3}
\end{equation}
\begin{equation}
\color{black} prob^{n}_{MNLI}  = [{0}_{e}, {0}_{c},{1}_{n},I({\mathit{SSTpred}}_{+}),{I(\mathit{SSTpred}}_{-})]
\label{eq:ps6:4}
\end{equation}
\begin{equation}
\color{black} prob^{d}_{QQP}  = [{1}_{e}, {0}_{c},{0}_{n},I(\mathit{SSTpred}_{+}),I(\mathit{SSTpred}_{-})]
\label{eq:ps6:5}
\end{equation}
\begin{equation}
\color{black} prob^{!d}_{QQP}  = [{0}_{e}, I(\mathit{MNLIpred}^{!e}_{c}),I(\mathit{MNLIpred}^{!e}_{n}),\nonumber\\
\color{black} I(\mathit{SSTpred}_{+}),I(\mathit{SSTpred}_{-})]
\label{eq:ps6:6}
\end{equation}
\begin{equation}
\
\color{black} prob^{+}_{SST} = [I(\mathit{MNLIpred}_{e}), I(\mathit{MNLIpred}_{c}),\nonumber\\
\color{black} I(\mathit{MNLIpred}_{n}),{1}_{+},{0}_{-}]
\label{eq:ps6:7}
\end{equation}
\begin{equation}
\color{black} prob^{-}_{SST} = [I(\mathit{MNLIpred}_{e}), I\mathit{MNLIpred}_{c}),\nonumber\\
\color{black} I(\mathit{MNLIpred}_{n}),{0}_{+},{1}_{-}]
\label{eq:ps6:8}
\end{equation}


\subsection{Независимые метки, замороженная голова}\label{subch:pseudolabel/sect3/sub7}

Данный подход аналогичен подходу \textbf{Независимые метки}, с тем исключением, что линейный классификационный слой не обучается, а обучается только тело модели. Все формулы для данного подхода аналогичны формулам для подхода \textbf{Независимые метки}.

\subsection{Мягкие независимые метки, замороженная голова}\label{subch:pseudolabel/sect3/sub8}

Данный подход аналогичен подходу \textbf{Мягкие независимые метки}, с тем исключением, что линейный классификационный слой не обучается, а обучается только тело модели. Все формулы для данного подхода аналогичны формулам для подхода \textbf{Мягкие независимые метки}.
\section{Настройки и результаты экспериментов}\label{ch:pseudolabel/sect4}

Для каждого из вышеописанных подходов, включая воспроизведение результатов оригинальной статьи, было сделано 4 попытки воспроизведения. Как и в оригинальной статье, эти 4 попытки отличались только скоростями обучения. По образцу этой статьи скорость обучения была 2e-5, 3e-5, 4e-5 и 5e-5 для первой, второй третьей и четвертой попытки соответственно. В качестве финальной была выбрана скорость обучения, для которой точность на валидационном наборе данных была максимальной. Длительность обучения была ограничена 3 эпохами. Полные валидационные данные для всех экспериментов можно увидеть в оригинальной статье.
Результаты на валидационных и тестовых данных приведены в Таблицах~\ref{tab:ps1} и~\ref{tab:ps2} соответственно. Стоит особо отметить, что данные результаты были достигнуты с числом параметров, на 10-13\% меньшим, чем в~\cite{stickland_2019}, и без изменений в базовой архитектуре нейросетевой модели.

\begin{table}[htbp]
\centering
\caption {Лучшая точность на валидационных данных (при лучшей скорости обучения из выбираемых, среднее по 3 запускам)}
\label{tab:ps1}% label всегда желательно идти после caption
\resizebox{\textwidth}{!}{%
\begin{tabular}{|c||c|c|c|c|c|}
\hline
\multirow{2}{*}{Эксперимент} & \multicolumn{5}{c|}{Задача}  \\
\cline{2-6}
 & Среднее& RTE & QQP & MNLI-m & SST-2 \\
\hline
\hline
Базовый (воспроизведённый) & 81.3 & 64.6 & 90.8 & 77.3 & 92.7   \\
\hline
Независимые метки    & 82.8  & \textbf{78.3} & 90.6 & 75.8 & 92.0  \\
\hline
Независимые метки, замороженная голова  & 82.5 & 76.1 & 90.5 & 75.7 & 91.4  \\
\hline
Мягкие независимые метки     & 82.2 & 69.7 & 89.5 & 75.9 & \textbf{92.6} \\
\hline
Мягкие независимые метки, замороженная голова & 82.6 & 74.4 & 90.4 & \textbf{76.7} & 91.2  \\
\hline
Дополненные независимые метки   & 81.4 & 68.1 & 90.5 & 75.6 & 92.4 \\
\hline
Мягкое вероятностное предположение  & \textbf{84.2} & 78.2 & \textbf{90.7} & 76.2 & 91.9   \\
\hline
Мягкие предсказанные метки  & 83.2 & 76.3 & 90.5 & 76.0 & 92.2  \\
\hline
Жесткие предсказанные метки & 82.9 & 77.4 & 90.6 & 75.3 & 90.7 \\
\hline
\end{tabular}
}
\end{table}

\begin{table}[htbp]
\centering
\caption {Лучшая точность на тестовых данных (при лучшей скорости обучения из выбираемых, среднее по 3 запускам)}
\label{tab:ps2}% label всегда желательно идти после caption
\resizebox{\textwidth}{!}
{%
\begin{tabular}{|c||c|c|c|c|c|c|}
\hline
\multirow{2}{*}{Эксперимент} & \multicolumn{6}{c|}{Задача}  \\
\cline{2-7}
& Среднее& RTE& QQP& MNLI-m &MNLI-mm& SST-2\\
\hline
Базовый (оригинальная статья) & 78.8 & 66.4  & 71.2 & 84.6 & 83.4 & 93.5 \\
\hline
Базовый (воспроизведённый) & 77.6 & 62.7 & 71.0 & 83.1 & 82.7 & 93.5 \\
\hline
Независимые метки &79.0 & 71.5 & 70.9 & 82.7 & 81.7 & 91.3 \\
\hline
Мягкие независимые метки  &78.9 & 69.3 & 71.3 & 82.8 & 82.1 & 92.6 \\
\hline
Дополненные независимые метки &77.6 & 64.2&\textbf{ 71.8} & 81.2 & 80.7 & \textbf{93.2} \\
\hline
Мягкое вероятностное предположение  &\textbf{79.7} & \textbf{72.7} & 70.7 &\textbf{ 83.4} &\textbf{82.3} & 92.5 \\
\hline
Мягкие предсказанные метки  & 78.8 & 70.3 & 70.7 & 81.7 & 81.7 & 92.5 \\
\hline
Жесткие предсказанные метки & 79.1 & 71.3 &71.1 & 81.7 & 81.4 & 92.6 \\
\hline
\begin{tabular}[c]{@{}l@{}}Независимые метки,\\замороженная голова\end{tabular}   & 78.2 & 66.9& \textbf{71.8} & 82.6 & 81.8 & 91.9 \\
\hline
\begin{tabular}[c]{@{}l@{}}Мягкие независимые метки,\\замороженная голова\end{tabular}  &79.1 & 70.0 & 71.5 & 83.0 &\textbf{ 82.3} & 92.4 \\
\hline
\end{tabular}
}
\end{table}

\section{Выводы}\label{ch:pseudolabel/sect5}

Как можно видеть, разные способы дают результаты, похожие на результаты оригинальной модели BERT. Или даже более высокие результаты, если описывается задача RTE. Причина данного преимущества для задачи RTE --  его схожесть с остальными задачами из GLUE, для которых данных гораздо больше, чем для RTE.

 Отсутствие роста показателей для QQP при объединении меток показывает, что объединение меток для этой задачи было слишком грубым. Это показывает ограничения для предложенных способов объединения меток.
 
На задачах, которые больше всего похожи друг на друга -- RTE и MNLI -- лучше всего показывает себя метод \textbf{Мягкое вероятностное предположение}. Это доказывает оправданность объединения меток при решении похожих задач. Этот метод превосходит результаты для RTE из оригинальной статьи и отстает от результатов для других задач только на 0.5-1.2\%.

На задачах, не похожих на остальные, таких, как SST и QQP, лучше всего показывает себя метод \textbf{Дополненные независимые метки}, что объясняется эффектом переноса знаний. Этот метод должен работать лучше всех вышеописанных методов в условиях, когда задачи достаточно сильно отличаются друг от друга. Именно этот подход применялся при работе над многозадачными моделями в диалоговой платформе DeepPavlov Dream.

Главный \textbf{вывод} из этой главы -- \textbf{псевдоразметка данных с использованием однозадачных моделей улучшает метрики многозадачных моделей. При этом объединение классов оправдывает себя только для задач, достаточно сильно похожих друг на друга.}

Эксперименты, связанные с внедрением многозадачных моделей в диалоговую платформу DeepPavlov Dream, подробно описаны в Главе~\ref{ch:mtldream}.
