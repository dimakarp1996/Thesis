
\chapter*{Словарь терминов} % Заголовок
%TOREMOVE SENTIMENT TOXICITY RETRIEVAL SKILL - РАНЖИРУЮЩИЙ CONVERSATION SKILL - РАЗГОВОРНЫЙ TF-IDF RETRIEVAL embedding skill selector response selector
\addcontentsline{toc}{chapter}{Словарь терминов} % Добавляем его в оглавление
\textbf{Open-source} : находящийся в открытом доступе 
\textbf{Бенчмарк} : набор задач, используемый для оценки качества программного решения. 
\textbf{Конкатенация} : операция объединения двух векто
ров (последовательностей) в один, в результате которой элементы одного из векторов добавляются в конец другого. Происходит от английского слова concatenate. 
\textbf{Корреляция} : Статистическая взаимосвязь двух или более величин, которые можно с некой степенью точности считать случайной, такая, что изменениям одной(одних) из величин сопутствуют изменения другой(других) величин. 
\textbf{Тональность} : эмоциональная окраска текста. Обычно выде
ляют позитивную, негативную илинейтральную.
\textbf{Токсичность} : вид негативной характеристики текста, обычно
означает наличие в тексте нецензурных выражений, оскорблений, непристойно
стей, личностной ненависти и пр.
\textbf{Токен} : текстовая единица, представляющая из себя слово целиком или N-грамму символов.
\textbf{Токенизация} : процесс разбиения текста на токены. Например, разбиение
текста по пробелам и символам пунктуации
\textbf{Языковая модель} : нейросетевая модель, обученная для решения задачи моделирования языка, то есть предсказания следующего
слова/токена в тексте.
 \textbf{Разговорный навык} : модель, производящая по
заданном контексту реплику-гипотезу, которая может являться продолжением диалога.
\textbf{Ранжирующий навык} : модель, извлекающая реплику гипотезу по заданному контексту из заданного набора возможных реплик методом ранжирования.
\textbf{Ранжирующий навык TF-IDF} - ранжирующий навык, использующий меток ранжирования, основанный на алгоритме TF-IDF~\cite{tfidf} 
\textbf{Векторные представления слов/текстов} : представление слова/текста в виде вектора фиксированной длины с вещественными
значениями.  
\textbf{Векторизация слов/текстов} : получение их векторных представлений. 
\textbf{Аннотаторы} : модели для обработки естественного языка, которые получает на вход текст реплики(возможно, с состоянием диалога), и возвращают некие автоматическим образом полученные характеристики для этой реплики. 
\textbf{Выборщик навыков} : компонента, формирующая список навыков, которые будут вызваны для генерации реплик-кандидатов.
\textbf{Выборщик ответа} : компонента, использующая со
стояние диалога, реплики-кандидаты и их аннотации для выбора финального
ответа, возвращаемого пользователю.
сэмплирование
многоязычный
BERT
навык
трансформер-агностичный : работающий для любых типов трансформеров
датасет (набор данных) :
тренировочная выборка (разбиение) :
валидационная выборка (разбиение) :
тестовая выборка (разбиение) :
Дообучение : fine-tuning, обучение модели на новом наборе данных (воз
можно на другой задаче), которая была ранее уже обучена на другом наборе
данных
Предобучение : процесс предварительного обучения модели, который
применяется перед обучением модели на целевом наборе данных (или задаче)
Перенос знаний : transfer learning, использование знаний, полученных во
время обучения на одной задаче (и/или домене) для обучения модели на другой
задаче (и/или домене). Более общее понятие, чем дообучение и предобучение.
100
Конкатенация : от англ. concatenate, операция объединения двух векто
ров (последовательностей) в один, в которой элементы одного из векторов добавляются в конец другого.
корреляция: Статистическая взаимосвязь двух или более случайных
величин (либо величин, которые можно с некоторой допустимой степенью точ
ности считать таковыми).
языковая модель (Language Model) : Модель обученная для предсказа
ния слова/токена в тексте.
learning rate: Коэффициент скорости обучения это гиперпараметр, опре
деляющий порядок как корректировать весы нейронной сети с учётом функции
потерь.
batch size : Количество обучающих примеров за одну итерацию.
dropout : Метод регуляризации, позволяющий уменьшить переобучение
модели за счет предупреждения коадаптаций нейронов на тренировочных дан
ных в процессе обучения.
интент : Намерение, выражаемые одной из сторон диалога
