\chapter*{Заключение}                       % Заголовок
\addcontentsline{toc}{chapter}{Заключение}  % Добавляем его в оглавление

%% Согласно ГОСТ Р 7.0.11-2011:
%% 5.3.3 В заключении диссертации излагают итоги выполненного исследования, рекомендации, перспективы дальнейшей разработки темы.
%% 9.2.3 В заключении автореферата диссертации излагают итоги данного исследования, рекомендации и перспективы дальнейшей разработки темы.
%% Поэтому имеет смысл сделать эту часть общей и загрузить из одного файла в автореферат и в диссертацию:

Основные результаты работы заключаются в следующем:
%% Согласно ГОСТ Р 7.0.11-2011:
%% 5.3.3 В заключении диссертации излагают итоги выполненного исследования, рекомендации, перспективы дальнейшей разработки темы.
%% 9.2.3 В заключении автореферата диссертации излагают итоги данного исследования, рекомендации и перспективы дальнейшей разработки темы.
%ПРОСТО СКОПИРОВАЛ ПОЛОЖЕНИЯ ВЫНОСИМЫЕ ДЛЯ ЗАЩИТЫ
\begin{enumerate}
  \item {Созданная при существенном вкладе автора платформа DREAM, находящаяся в открытом доступе, пригодна для изучения прикладного применения многозадачных нейросетевых моделей.}
  %\item {На примере прикладной задачи по созданию сервиса для работы с текстами texter-ocr-cv-microservice показана применимость технологий, использованных в диалоговой платформе DREAM, за пределами этой диалоговой платформы.}
  \item {Псевдоразметка данных при помощи однозадачных моделей улучшает метрики многозадачных моделей. При этом объединение классов оправдывает себя только для задач, достаточно сильно похожих друг на друга.}
  \item {Многозадачные трансформер-агностичные нейросетевые модели показывают себя не хуже ряда других, более сложных архитектур, а предложенный метод сэмплирования - не хуже ряда других методов сэмплирования. При этом многозадачные трансформер-агностичные модели по данным проведенных экспериментов дают среднюю просадку не более 1 процента по сравнению с однозадачными моделями. При достаточной степени похожести задач друг на друга модели за счет таких задач в среднем даже превосходят однозадачные модели.}
  \item {Для достаточно малых данных многозадачные трансформер-агностичных модели начинают превосходить по своей средней точности однозадачные, в особенности - за счет задач с наименьшим объемом данных.}
  \item {Если в основе многозадачной трансформер-агностичной модели лежит многоязычный BERT, то добавление английских данных к русским при соответствующей номенклатуре классов позволяет улучшить метрики на 1-5\%. Чем меньше изначально русскоязычных данных, тем улучшение сильнее. Этот же вывод справедлив и для однозадачных моделей.}
  \item {Русскоязычные модели, обученные на 6 классах из русскоязычного набора тематических данных YAQTopics, показывают точность около 85\% на наборе данных MASSIVE. Это оправдывает использование набора данных YAQTopics для решения задачи русскоязычной тематической классификации и фундаментальных задач исследования переноса знаний.}
  \item {Для многоязычных нейросетевых моделей качество переноса знаний на разные языки на тематических данных сильно коррелирует с размером предобучающей выборки для каждого языка, но при этом не коррелирует с генеалогической близостью этого языка к языку дообучения.}
  \item {Рассмотренные многозадачные нейросетевые архитектуры пригодны для практического применения в диалоговых платформах и в рамках open-source библиотек. При этом предложенные автором трансформер-агностичные нейросетевые модели выигрывают у моделей типа PAL-BERT за счет трансформер-агностичности, а у моделей с одним линейным слоем - за счёт большей гибкости, отсутствия необходимости в псевдоразметке и как следствие - меньшей склонности к переобучению.}
\end{enumerate}


В заключение хочу выразить особую благодарность следующим людям - 
\begin{enumerate}
\item Научному руководителю Бурцеву~М.\:С. - за все предоставленные условия, существенную помощь в работе над~\cite{dream1,dream1_trudy,dream2,pseudolabel,mtldream}, над диалоговой системой DREAM и научное руководство.
\item Научному консультанту Коновалову~В.\:П. - за существенную помощь в работе над~\cite{rumtl,rutopics}, регулярное консультирование и поддержку на разных этапах работы.
\item Матвееву~И.\:А. - за регулярное консультирование на этапе оформления результатов выполненной работы в виде диссертации.
\item Попову~А. - за консультирование на этапе финальной подачи результатов работы~\cite{rutopics}.
\item Жариковой(Баймурзиной)~Д.\:Р., Кузнецову~Д.\:П., Куратову~Ю.\:М. и Юсупову~И.\:Ф. - за регулярное рецензирование моего кода в рамках работы над диалоговой системой DREAM и платформой DeepPavlov Dream, которое помогло мне существенно улучшить навыки программирования.
\item Помимо них, также всем остальным членам команды DREAM в конкурсах Alexa Prize(Дмитриевскому~А., Евсееву~Д.\:А., Ермаковой~Е.\:С., Игнатову~Ф.\:С., Корневу~Д.\:А., Кумейко Ле~Т.\:А., Остяковой~Л.\:Н., Пеганову~А.\:О., Пугину~П.\:Ю., Сагировой~А.\:Р., Серикову~О.\:А., Чернявскому~Д.\:В.) - за все совместно достигнутые результаты в рамках работы над диалоговой системой DREAM и платформой DeepPavlov Dream.
\item Игнатову~Ф.\:С., помимо предыдущего пункта - за существенную помощь в интеграции моего кода в библиотеку DeepPavlov.
\item Левчуку~П. за сбор данных, использованных при конструировании набора данных RuQTopics.
%\item Базановой~Е.\:М. - за существенную помощь в коррекции английского языка для работы~\cite{enmtl}.
\item Чижиковой~А.\:П. - за существенную помощь в коррекции английского языка для работ~\cite{rumtl,rutopics}.
\item Дуплякину~В.\:О. - за предоставленную возможность выполнения работ, защищенных патентом~\cite{Дуплякин_Дмитрий_Ондар_Ушаков_2021}. Кроме него, также Адыгжы~\:О. и Ушакову~\:А. за совместно достигнутые результаты в рамках этих работ. 
\item Бобе~А.\:С. - за предоставленную возможность выполнения работы~\cite{Болотин_Карпов_Рашков_Шкурак_2019}. Кроме него, также Болотину~Д.\:И., Рашкову~Г.\:В и Шкураку~И.\:Н. за совместно достигнутые результаты в рамках этой работы. 
\item Своим родителям: Карпову~А.\:И. и Приступа~И.\:Г. за то, сколько всего они в меня вложили, и вообще за всё.
\item Игнатьеву~П.\:К., Карповой~Н.\:В.(своей бабушке), Прасолову~Д.\:Ю., Санаевой~А.\:О. за регулярную моральную поддержку, которая существенно повлияла на моё моральное состояние и продуктивность во время выполнения работы.
\item Своему покойному дедушке Карпову~И.\:С. за то, что во многом был для меня примером.\footnote{\url{https://ru.wikipedia.org/wiki/\%D0\%9A\%D0\%B0\%D1\%80\%D0\%BF\%D0\%BE\%D0\%B2,_\%D0\%98\%D0\%B2\%D0\%B0\%D0\%BD_\%D0\%A1\%D0\%B5\%D0\%BC\%D1\%91\%D0\%BD\%D0\%BE\%D0\%B2\%D0\%B8\%D1\%87}}
\end{enumerate}


